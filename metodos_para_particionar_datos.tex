\documentclass{article}
\usepackage{amsmath}

\title{Métodos para Particionar Datos}
\date{}

\begin{document}

\maketitle

\section{Regla de Sturges}
\begin{equation}
k = 1 + \log_2(n)
\end{equation}
Es simple y funciona bien para conjuntos de datos pequeños y moderados. Asume una distribución normal de los datos. \\
Desventajas: Tiende a subestimar el número de clases para conjuntos de datos grandes.

\section{Regla de Scott}
\begin{equation}
h = 3.5\sigma n^{-\frac{1}{3}}
\end{equation}
Minimiza la varianza de la estimación del histograma. Es útil para distribuciones aproximadamente normales. \\
Desventajas: Puede no funcionar bien con datos que tienen una distribución sesgada.

\section{Regla de Freedman-Diaconis}
\begin{equation}
h = 2 \times \frac{IQR}{n^{\frac{1}{3}}}
\end{equation}
Utiliza el rango intercuartílico (IQR) para reducir el impacto de los valores atípicos, siendo más robusto que la regla de Scott. \\
Desventajas: Puede resultar en un número excesivo de intervalos si el conjunto de datos es pequeño.

\section{Regla de Doane}
\begin{equation}
k = 1 + \log_2(n) + \log_2\left(1 + \frac{|g_1|}{\sqrt{\frac{6(n-2)}{(n+1)(n+3)}}}\right)
\end{equation}
Es una extensión de la regla de Sturges que corrige para el sesgo en la distribución de los datos. Es útil para distribuciones no normales. \\
Desventajas: A veces puede sobreestimar el número de clases.

\section{Regla de Rice}
\begin{equation}
k = 2 \times n^{\frac{1}{3}}
\end{equation}
Es una regla empírica simple y más conservadora que la de Sturges. Se usa a menudo como una alternativa rápida y simple. \\
Desventajas: Tiende a sobreestimar el número de intervalos para conjuntos de datos pequeños.

\section{Conclusión}
La elección del método depende de la naturaleza de los datos. Para datos con distribución normal, la regla de Scott o Sturges podría ser suficiente. Para datos con valores atípicos o distribuciones sesgadas, Freedman-Diaconis es más robusta. En general, es recomendable probar diferentes métodos y elegir el que mejor refleje la distribución de los datos.

\end{document}
