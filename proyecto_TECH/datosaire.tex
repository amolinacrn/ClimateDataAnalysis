\documentclass[11pt]{article}

    \usepackage[breakable]{tcolorbox}
    \usepackage{parskip} % Stop auto-indenting (to mimic markdown behaviour)
    

    % Basic figure setup, for now with no caption control since it's done
    % automatically by Pandoc (which extracts ![](path) syntax from Markdown).
    \usepackage{graphicx}
    % Keep aspect ratio if custom image width or height is specified
    \setkeys{Gin}{keepaspectratio}
    % Maintain compatibility with old templates. Remove in nbconvert 6.0
    \let\Oldincludegraphics\includegraphics
    % Ensure that by default, figures have no caption (until we provide a
    % proper Figure object with a Caption API and a way to capture that
    % in the conversion process - todo).
    \usepackage{caption}
    \DeclareCaptionFormat{nocaption}{}
    \captionsetup{format=nocaption,aboveskip=0pt,belowskip=0pt}

    \usepackage{float}
    \floatplacement{figure}{H} % forces figures to be placed at the correct location
    \usepackage{xcolor} % Allow colors to be defined
    \usepackage{enumerate} % Needed for markdown enumerations to work
    \usepackage{geometry} % Used to adjust the document margins
    \usepackage{amsmath} % Equations
    \usepackage{amssymb} % Equations
    \usepackage{textcomp} % defines textquotesingle
    % Hack from http://tex.stackexchange.com/a/47451/13684:
    \AtBeginDocument{%
        \def\PYZsq{\textquotesingle}% Upright quotes in Pygmentized code
    }
    \usepackage{upquote} % Upright quotes for verbatim code
    \usepackage{eurosym} % defines \euro

    \usepackage{iftex}
    \ifPDFTeX
        \usepackage[T1]{fontenc}
        \IfFileExists{alphabeta.sty}{
              \usepackage{alphabeta}
          }{
              \usepackage[mathletters]{ucs}
              \usepackage[utf8x]{inputenc}
          }
    \else
        \usepackage{fontspec}
        \usepackage{unicode-math}
    \fi

    \usepackage{fancyvrb} % verbatim replacement that allows latex
    \usepackage{grffile} % extends the file name processing of package graphics
                         % to support a larger range
    \makeatletter % fix for old versions of grffile with XeLaTeX
    \@ifpackagelater{grffile}{2019/11/01}
    {
      % Do nothing on new versions
    }
    {
      \def\Gread@@xetex#1{%
        \IfFileExists{"\Gin@base".bb}%
        {\Gread@eps{\Gin@base.bb}}%
        {\Gread@@xetex@aux#1}%
      }
    }
    \makeatother
    \usepackage[Export]{adjustbox} % Used to constrain images to a maximum size
    \adjustboxset{max size={0.9\linewidth}{0.9\paperheight}}

    % The hyperref package gives us a pdf with properly built
    % internal navigation ('pdf bookmarks' for the table of contents,
    % internal cross-reference links, web links for URLs, etc.)
    \usepackage{hyperref}
    % The default LaTeX title has an obnoxious amount of whitespace. By default,
    % titling removes some of it. It also provides customization options.
    \usepackage{titling}
    \usepackage{longtable} % longtable support required by pandoc >1.10
    \usepackage{booktabs}  % table support for pandoc > 1.12.2
    \usepackage{array}     % table support for pandoc >= 2.11.3
    \usepackage{calc}      % table minipage width calculation for pandoc >= 2.11.1
    \usepackage[inline]{enumitem} % IRkernel/repr support (it uses the enumerate* environment)
    \usepackage[normalem]{ulem} % ulem is needed to support strikethroughs (\sout)
                                % normalem makes italics be italics, not underlines
    \usepackage{soul}      % strikethrough (\st) support for pandoc >= 3.0.0
    \usepackage{mathrsfs}
    

    
    % Colors for the hyperref package
    \definecolor{urlcolor}{rgb}{0,.145,.698}
    \definecolor{linkcolor}{rgb}{.71,0.21,0.01}
    \definecolor{citecolor}{rgb}{.12,.54,.11}

    % ANSI colors
    \definecolor{ansi-black}{HTML}{3E424D}
    \definecolor{ansi-black-intense}{HTML}{282C36}
    \definecolor{ansi-red}{HTML}{E75C58}
    \definecolor{ansi-red-intense}{HTML}{B22B31}
    \definecolor{ansi-green}{HTML}{00A250}
    \definecolor{ansi-green-intense}{HTML}{007427}
    \definecolor{ansi-yellow}{HTML}{DDB62B}
    \definecolor{ansi-yellow-intense}{HTML}{B27D12}
    \definecolor{ansi-blue}{HTML}{208FFB}
    \definecolor{ansi-blue-intense}{HTML}{0065CA}
    \definecolor{ansi-magenta}{HTML}{D160C4}
    \definecolor{ansi-magenta-intense}{HTML}{A03196}
    \definecolor{ansi-cyan}{HTML}{60C6C8}
    \definecolor{ansi-cyan-intense}{HTML}{258F8F}
    \definecolor{ansi-white}{HTML}{C5C1B4}
    \definecolor{ansi-white-intense}{HTML}{A1A6B2}
    \definecolor{ansi-default-inverse-fg}{HTML}{FFFFFF}
    \definecolor{ansi-default-inverse-bg}{HTML}{000000}

    % common color for the border for error outputs.
    \definecolor{outerrorbackground}{HTML}{FFDFDF}

    % commands and environments needed by pandoc snippets
    % extracted from the output of `pandoc -s`
    \providecommand{\tightlist}{%
      \setlength{\itemsep}{0pt}\setlength{\parskip}{0pt}}
    \DefineVerbatimEnvironment{Highlighting}{Verbatim}{commandchars=\\\{\}}
    % Add ',fontsize=\small' for more characters per line
    \newenvironment{Shaded}{}{}
    \newcommand{\KeywordTok}[1]{\textcolor[rgb]{0.00,0.44,0.13}{\textbf{{#1}}}}
    \newcommand{\DataTypeTok}[1]{\textcolor[rgb]{0.56,0.13,0.00}{{#1}}}
    \newcommand{\DecValTok}[1]{\textcolor[rgb]{0.25,0.63,0.44}{{#1}}}
    \newcommand{\BaseNTok}[1]{\textcolor[rgb]{0.25,0.63,0.44}{{#1}}}
    \newcommand{\FloatTok}[1]{\textcolor[rgb]{0.25,0.63,0.44}{{#1}}}
    \newcommand{\CharTok}[1]{\textcolor[rgb]{0.25,0.44,0.63}{{#1}}}
    \newcommand{\StringTok}[1]{\textcolor[rgb]{0.25,0.44,0.63}{{#1}}}
    \newcommand{\CommentTok}[1]{\textcolor[rgb]{0.38,0.63,0.69}{\textit{{#1}}}}
    \newcommand{\OtherTok}[1]{\textcolor[rgb]{0.00,0.44,0.13}{{#1}}}
    \newcommand{\AlertTok}[1]{\textcolor[rgb]{1.00,0.00,0.00}{\textbf{{#1}}}}
    \newcommand{\FunctionTok}[1]{\textcolor[rgb]{0.02,0.16,0.49}{{#1}}}
    \newcommand{\RegionMarkerTok}[1]{{#1}}
    \newcommand{\ErrorTok}[1]{\textcolor[rgb]{1.00,0.00,0.00}{\textbf{{#1}}}}
    \newcommand{\NormalTok}[1]{{#1}}

    % Additional commands for more recent versions of Pandoc
    \newcommand{\ConstantTok}[1]{\textcolor[rgb]{0.53,0.00,0.00}{{#1}}}
    \newcommand{\SpecialCharTok}[1]{\textcolor[rgb]{0.25,0.44,0.63}{{#1}}}
    \newcommand{\VerbatimStringTok}[1]{\textcolor[rgb]{0.25,0.44,0.63}{{#1}}}
    \newcommand{\SpecialStringTok}[1]{\textcolor[rgb]{0.73,0.40,0.53}{{#1}}}
    \newcommand{\ImportTok}[1]{{#1}}
    \newcommand{\DocumentationTok}[1]{\textcolor[rgb]{0.73,0.13,0.13}{\textit{{#1}}}}
    \newcommand{\AnnotationTok}[1]{\textcolor[rgb]{0.38,0.63,0.69}{\textbf{\textit{{#1}}}}}
    \newcommand{\CommentVarTok}[1]{\textcolor[rgb]{0.38,0.63,0.69}{\textbf{\textit{{#1}}}}}
    \newcommand{\VariableTok}[1]{\textcolor[rgb]{0.10,0.09,0.49}{{#1}}}
    \newcommand{\ControlFlowTok}[1]{\textcolor[rgb]{0.00,0.44,0.13}{\textbf{{#1}}}}
    \newcommand{\OperatorTok}[1]{\textcolor[rgb]{0.40,0.40,0.40}{{#1}}}
    \newcommand{\BuiltInTok}[1]{{#1}}
    \newcommand{\ExtensionTok}[1]{{#1}}
    \newcommand{\PreprocessorTok}[1]{\textcolor[rgb]{0.74,0.48,0.00}{{#1}}}
    \newcommand{\AttributeTok}[1]{\textcolor[rgb]{0.49,0.56,0.16}{{#1}}}
    \newcommand{\InformationTok}[1]{\textcolor[rgb]{0.38,0.63,0.69}{\textbf{\textit{{#1}}}}}
    \newcommand{\WarningTok}[1]{\textcolor[rgb]{0.38,0.63,0.69}{\textbf{\textit{{#1}}}}}


    % Define a nice break command that doesn't care if a line doesn't already
    % exist.
    \def\br{\hspace*{\fill} \\* }
    % Math Jax compatibility definitions
    \def\gt{>}
    \def\lt{<}
    \let\Oldtex\TeX
    \let\Oldlatex\LaTeX
    \renewcommand{\TeX}{\textrm{\Oldtex}}
    \renewcommand{\LaTeX}{\textrm{\Oldlatex}}
    % Document parameters
    % Document title
    \title{datosaire}
    
    
    
    
    
    
    
% Pygments definitions
\makeatletter
\def\PY@reset{\let\PY@it=\relax \let\PY@bf=\relax%
    \let\PY@ul=\relax \let\PY@tc=\relax%
    \let\PY@bc=\relax \let\PY@ff=\relax}
\def\PY@tok#1{\csname PY@tok@#1\endcsname}
\def\PY@toks#1+{\ifx\relax#1\empty\else%
    \PY@tok{#1}\expandafter\PY@toks\fi}
\def\PY@do#1{\PY@bc{\PY@tc{\PY@ul{%
    \PY@it{\PY@bf{\PY@ff{#1}}}}}}}
\def\PY#1#2{\PY@reset\PY@toks#1+\relax+\PY@do{#2}}

\@namedef{PY@tok@w}{\def\PY@tc##1{\textcolor[rgb]{0.73,0.73,0.73}{##1}}}
\@namedef{PY@tok@c}{\let\PY@it=\textit\def\PY@tc##1{\textcolor[rgb]{0.24,0.48,0.48}{##1}}}
\@namedef{PY@tok@cp}{\def\PY@tc##1{\textcolor[rgb]{0.61,0.40,0.00}{##1}}}
\@namedef{PY@tok@k}{\let\PY@bf=\textbf\def\PY@tc##1{\textcolor[rgb]{0.00,0.50,0.00}{##1}}}
\@namedef{PY@tok@kp}{\def\PY@tc##1{\textcolor[rgb]{0.00,0.50,0.00}{##1}}}
\@namedef{PY@tok@kt}{\def\PY@tc##1{\textcolor[rgb]{0.69,0.00,0.25}{##1}}}
\@namedef{PY@tok@o}{\def\PY@tc##1{\textcolor[rgb]{0.40,0.40,0.40}{##1}}}
\@namedef{PY@tok@ow}{\let\PY@bf=\textbf\def\PY@tc##1{\textcolor[rgb]{0.67,0.13,1.00}{##1}}}
\@namedef{PY@tok@nb}{\def\PY@tc##1{\textcolor[rgb]{0.00,0.50,0.00}{##1}}}
\@namedef{PY@tok@nf}{\def\PY@tc##1{\textcolor[rgb]{0.00,0.00,1.00}{##1}}}
\@namedef{PY@tok@nc}{\let\PY@bf=\textbf\def\PY@tc##1{\textcolor[rgb]{0.00,0.00,1.00}{##1}}}
\@namedef{PY@tok@nn}{\let\PY@bf=\textbf\def\PY@tc##1{\textcolor[rgb]{0.00,0.00,1.00}{##1}}}
\@namedef{PY@tok@ne}{\let\PY@bf=\textbf\def\PY@tc##1{\textcolor[rgb]{0.80,0.25,0.22}{##1}}}
\@namedef{PY@tok@nv}{\def\PY@tc##1{\textcolor[rgb]{0.10,0.09,0.49}{##1}}}
\@namedef{PY@tok@no}{\def\PY@tc##1{\textcolor[rgb]{0.53,0.00,0.00}{##1}}}
\@namedef{PY@tok@nl}{\def\PY@tc##1{\textcolor[rgb]{0.46,0.46,0.00}{##1}}}
\@namedef{PY@tok@ni}{\let\PY@bf=\textbf\def\PY@tc##1{\textcolor[rgb]{0.44,0.44,0.44}{##1}}}
\@namedef{PY@tok@na}{\def\PY@tc##1{\textcolor[rgb]{0.41,0.47,0.13}{##1}}}
\@namedef{PY@tok@nt}{\let\PY@bf=\textbf\def\PY@tc##1{\textcolor[rgb]{0.00,0.50,0.00}{##1}}}
\@namedef{PY@tok@nd}{\def\PY@tc##1{\textcolor[rgb]{0.67,0.13,1.00}{##1}}}
\@namedef{PY@tok@s}{\def\PY@tc##1{\textcolor[rgb]{0.73,0.13,0.13}{##1}}}
\@namedef{PY@tok@sd}{\let\PY@it=\textit\def\PY@tc##1{\textcolor[rgb]{0.73,0.13,0.13}{##1}}}
\@namedef{PY@tok@si}{\let\PY@bf=\textbf\def\PY@tc##1{\textcolor[rgb]{0.64,0.35,0.47}{##1}}}
\@namedef{PY@tok@se}{\let\PY@bf=\textbf\def\PY@tc##1{\textcolor[rgb]{0.67,0.36,0.12}{##1}}}
\@namedef{PY@tok@sr}{\def\PY@tc##1{\textcolor[rgb]{0.64,0.35,0.47}{##1}}}
\@namedef{PY@tok@ss}{\def\PY@tc##1{\textcolor[rgb]{0.10,0.09,0.49}{##1}}}
\@namedef{PY@tok@sx}{\def\PY@tc##1{\textcolor[rgb]{0.00,0.50,0.00}{##1}}}
\@namedef{PY@tok@m}{\def\PY@tc##1{\textcolor[rgb]{0.40,0.40,0.40}{##1}}}
\@namedef{PY@tok@gh}{\let\PY@bf=\textbf\def\PY@tc##1{\textcolor[rgb]{0.00,0.00,0.50}{##1}}}
\@namedef{PY@tok@gu}{\let\PY@bf=\textbf\def\PY@tc##1{\textcolor[rgb]{0.50,0.00,0.50}{##1}}}
\@namedef{PY@tok@gd}{\def\PY@tc##1{\textcolor[rgb]{0.63,0.00,0.00}{##1}}}
\@namedef{PY@tok@gi}{\def\PY@tc##1{\textcolor[rgb]{0.00,0.52,0.00}{##1}}}
\@namedef{PY@tok@gr}{\def\PY@tc##1{\textcolor[rgb]{0.89,0.00,0.00}{##1}}}
\@namedef{PY@tok@ge}{\let\PY@it=\textit}
\@namedef{PY@tok@gs}{\let\PY@bf=\textbf}
\@namedef{PY@tok@ges}{\let\PY@bf=\textbf\let\PY@it=\textit}
\@namedef{PY@tok@gp}{\let\PY@bf=\textbf\def\PY@tc##1{\textcolor[rgb]{0.00,0.00,0.50}{##1}}}
\@namedef{PY@tok@go}{\def\PY@tc##1{\textcolor[rgb]{0.44,0.44,0.44}{##1}}}
\@namedef{PY@tok@gt}{\def\PY@tc##1{\textcolor[rgb]{0.00,0.27,0.87}{##1}}}
\@namedef{PY@tok@err}{\def\PY@bc##1{{\setlength{\fboxsep}{\string -\fboxrule}\fcolorbox[rgb]{1.00,0.00,0.00}{1,1,1}{\strut ##1}}}}
\@namedef{PY@tok@kc}{\let\PY@bf=\textbf\def\PY@tc##1{\textcolor[rgb]{0.00,0.50,0.00}{##1}}}
\@namedef{PY@tok@kd}{\let\PY@bf=\textbf\def\PY@tc##1{\textcolor[rgb]{0.00,0.50,0.00}{##1}}}
\@namedef{PY@tok@kn}{\let\PY@bf=\textbf\def\PY@tc##1{\textcolor[rgb]{0.00,0.50,0.00}{##1}}}
\@namedef{PY@tok@kr}{\let\PY@bf=\textbf\def\PY@tc##1{\textcolor[rgb]{0.00,0.50,0.00}{##1}}}
\@namedef{PY@tok@bp}{\def\PY@tc##1{\textcolor[rgb]{0.00,0.50,0.00}{##1}}}
\@namedef{PY@tok@fm}{\def\PY@tc##1{\textcolor[rgb]{0.00,0.00,1.00}{##1}}}
\@namedef{PY@tok@vc}{\def\PY@tc##1{\textcolor[rgb]{0.10,0.09,0.49}{##1}}}
\@namedef{PY@tok@vg}{\def\PY@tc##1{\textcolor[rgb]{0.10,0.09,0.49}{##1}}}
\@namedef{PY@tok@vi}{\def\PY@tc##1{\textcolor[rgb]{0.10,0.09,0.49}{##1}}}
\@namedef{PY@tok@vm}{\def\PY@tc##1{\textcolor[rgb]{0.10,0.09,0.49}{##1}}}
\@namedef{PY@tok@sa}{\def\PY@tc##1{\textcolor[rgb]{0.73,0.13,0.13}{##1}}}
\@namedef{PY@tok@sb}{\def\PY@tc##1{\textcolor[rgb]{0.73,0.13,0.13}{##1}}}
\@namedef{PY@tok@sc}{\def\PY@tc##1{\textcolor[rgb]{0.73,0.13,0.13}{##1}}}
\@namedef{PY@tok@dl}{\def\PY@tc##1{\textcolor[rgb]{0.73,0.13,0.13}{##1}}}
\@namedef{PY@tok@s2}{\def\PY@tc##1{\textcolor[rgb]{0.73,0.13,0.13}{##1}}}
\@namedef{PY@tok@sh}{\def\PY@tc##1{\textcolor[rgb]{0.73,0.13,0.13}{##1}}}
\@namedef{PY@tok@s1}{\def\PY@tc##1{\textcolor[rgb]{0.73,0.13,0.13}{##1}}}
\@namedef{PY@tok@mb}{\def\PY@tc##1{\textcolor[rgb]{0.40,0.40,0.40}{##1}}}
\@namedef{PY@tok@mf}{\def\PY@tc##1{\textcolor[rgb]{0.40,0.40,0.40}{##1}}}
\@namedef{PY@tok@mh}{\def\PY@tc##1{\textcolor[rgb]{0.40,0.40,0.40}{##1}}}
\@namedef{PY@tok@mi}{\def\PY@tc##1{\textcolor[rgb]{0.40,0.40,0.40}{##1}}}
\@namedef{PY@tok@il}{\def\PY@tc##1{\textcolor[rgb]{0.40,0.40,0.40}{##1}}}
\@namedef{PY@tok@mo}{\def\PY@tc##1{\textcolor[rgb]{0.40,0.40,0.40}{##1}}}
\@namedef{PY@tok@ch}{\let\PY@it=\textit\def\PY@tc##1{\textcolor[rgb]{0.24,0.48,0.48}{##1}}}
\@namedef{PY@tok@cm}{\let\PY@it=\textit\def\PY@tc##1{\textcolor[rgb]{0.24,0.48,0.48}{##1}}}
\@namedef{PY@tok@cpf}{\let\PY@it=\textit\def\PY@tc##1{\textcolor[rgb]{0.24,0.48,0.48}{##1}}}
\@namedef{PY@tok@c1}{\let\PY@it=\textit\def\PY@tc##1{\textcolor[rgb]{0.24,0.48,0.48}{##1}}}
\@namedef{PY@tok@cs}{\let\PY@it=\textit\def\PY@tc##1{\textcolor[rgb]{0.24,0.48,0.48}{##1}}}

\def\PYZbs{\char`\\}
\def\PYZus{\char`\_}
\def\PYZob{\char`\{}
\def\PYZcb{\char`\}}
\def\PYZca{\char`\^}
\def\PYZam{\char`\&}
\def\PYZlt{\char`\<}
\def\PYZgt{\char`\>}
\def\PYZsh{\char`\#}
\def\PYZpc{\char`\%}
\def\PYZdl{\char`\$}
\def\PYZhy{\char`\-}
\def\PYZsq{\char`\'}
\def\PYZdq{\char`\"}
\def\PYZti{\char`\~}
% for compatibility with earlier versions
\def\PYZat{@}
\def\PYZlb{[}
\def\PYZrb{]}
\makeatother


    % For linebreaks inside Verbatim environment from package fancyvrb.
    \makeatletter
        \newbox\Wrappedcontinuationbox
        \newbox\Wrappedvisiblespacebox
        \newcommand*\Wrappedvisiblespace {\textcolor{red}{\textvisiblespace}}
        \newcommand*\Wrappedcontinuationsymbol {\textcolor{red}{\llap{\tiny$\m@th\hookrightarrow$}}}
        \newcommand*\Wrappedcontinuationindent {3ex }
        \newcommand*\Wrappedafterbreak {\kern\Wrappedcontinuationindent\copy\Wrappedcontinuationbox}
        % Take advantage of the already applied Pygments mark-up to insert
        % potential linebreaks for TeX processing.
        %        {, <, #, %, $, ' and ": go to next line.
        %        _, }, ^, &, >, - and ~: stay at end of broken line.
        % Use of \textquotesingle for straight quote.
        \newcommand*\Wrappedbreaksatspecials {%
            \def\PYGZus{\discretionary{\char`\_}{\Wrappedafterbreak}{\char`\_}}%
            \def\PYGZob{\discretionary{}{\Wrappedafterbreak\char`\{}{\char`\{}}%
            \def\PYGZcb{\discretionary{\char`\}}{\Wrappedafterbreak}{\char`\}}}%
            \def\PYGZca{\discretionary{\char`\^}{\Wrappedafterbreak}{\char`\^}}%
            \def\PYGZam{\discretionary{\char`\&}{\Wrappedafterbreak}{\char`\&}}%
            \def\PYGZlt{\discretionary{}{\Wrappedafterbreak\char`\<}{\char`\<}}%
            \def\PYGZgt{\discretionary{\char`\>}{\Wrappedafterbreak}{\char`\>}}%
            \def\PYGZsh{\discretionary{}{\Wrappedafterbreak\char`\#}{\char`\#}}%
            \def\PYGZpc{\discretionary{}{\Wrappedafterbreak\char`\%}{\char`\%}}%
            \def\PYGZdl{\discretionary{}{\Wrappedafterbreak\char`\$}{\char`\$}}%
            \def\PYGZhy{\discretionary{\char`\-}{\Wrappedafterbreak}{\char`\-}}%
            \def\PYGZsq{\discretionary{}{\Wrappedafterbreak\textquotesingle}{\textquotesingle}}%
            \def\PYGZdq{\discretionary{}{\Wrappedafterbreak\char`\"}{\char`\"}}%
            \def\PYGZti{\discretionary{\char`\~}{\Wrappedafterbreak}{\char`\~}}%
        }
        % Some characters . , ; ? ! / are not pygmentized.
        % This macro makes them "active" and they will insert potential linebreaks
        \newcommand*\Wrappedbreaksatpunct {%
            \lccode`\~`\.\lowercase{\def~}{\discretionary{\hbox{\char`\.}}{\Wrappedafterbreak}{\hbox{\char`\.}}}%
            \lccode`\~`\,\lowercase{\def~}{\discretionary{\hbox{\char`\,}}{\Wrappedafterbreak}{\hbox{\char`\,}}}%
            \lccode`\~`\;\lowercase{\def~}{\discretionary{\hbox{\char`\;}}{\Wrappedafterbreak}{\hbox{\char`\;}}}%
            \lccode`\~`\:\lowercase{\def~}{\discretionary{\hbox{\char`\:}}{\Wrappedafterbreak}{\hbox{\char`\:}}}%
            \lccode`\~`\?\lowercase{\def~}{\discretionary{\hbox{\char`\?}}{\Wrappedafterbreak}{\hbox{\char`\?}}}%
            \lccode`\~`\!\lowercase{\def~}{\discretionary{\hbox{\char`\!}}{\Wrappedafterbreak}{\hbox{\char`\!}}}%
            \lccode`\~`\/\lowercase{\def~}{\discretionary{\hbox{\char`\/}}{\Wrappedafterbreak}{\hbox{\char`\/}}}%
            \catcode`\.\active
            \catcode`\,\active
            \catcode`\;\active
            \catcode`\:\active
            \catcode`\?\active
            \catcode`\!\active
            \catcode`\/\active
            \lccode`\~`\~
        }
    \makeatother

    \let\OriginalVerbatim=\Verbatim
    \makeatletter
    \renewcommand{\Verbatim}[1][1]{%
        %\parskip\z@skip
        \sbox\Wrappedcontinuationbox {\Wrappedcontinuationsymbol}%
        \sbox\Wrappedvisiblespacebox {\FV@SetupFont\Wrappedvisiblespace}%
        \def\FancyVerbFormatLine ##1{\hsize\linewidth
            \vtop{\raggedright\hyphenpenalty\z@\exhyphenpenalty\z@
                \doublehyphendemerits\z@\finalhyphendemerits\z@
                \strut ##1\strut}%
        }%
        % If the linebreak is at a space, the latter will be displayed as visible
        % space at end of first line, and a continuation symbol starts next line.
        % Stretch/shrink are however usually zero for typewriter font.
        \def\FV@Space {%
            \nobreak\hskip\z@ plus\fontdimen3\font minus\fontdimen4\font
            \discretionary{\copy\Wrappedvisiblespacebox}{\Wrappedafterbreak}
            {\kern\fontdimen2\font}%
        }%

        % Allow breaks at special characters using \PYG... macros.
        \Wrappedbreaksatspecials
        % Breaks at punctuation characters . , ; ? ! and / need catcode=\active
        \OriginalVerbatim[#1,codes*=\Wrappedbreaksatpunct]%
    }
    \makeatother

    % Exact colors from NB
    \definecolor{incolor}{HTML}{303F9F}
    \definecolor{outcolor}{HTML}{D84315}
    \definecolor{cellborder}{HTML}{CFCFCF}
    \definecolor{cellbackground}{HTML}{F7F7F7}

    % prompt
    \makeatletter
    \newcommand{\boxspacing}{\kern\kvtcb@left@rule\kern\kvtcb@boxsep}
    \makeatother
    \newcommand{\prompt}[4]{
        {\ttfamily\llap{{\color{#2}[#3]:\hspace{3pt}#4}}\vspace{-\baselineskip}}
    }
    

    
    % Prevent overflowing lines due to hard-to-break entities
    \sloppy
    % Setup hyperref package
    \hypersetup{
      breaklinks=true,  % so long urls are correctly broken across lines
      colorlinks=true,
      urlcolor=urlcolor,
      linkcolor=linkcolor,
      citecolor=citecolor,
      }
    % Slightly bigger margins than the latex defaults
    
    \geometry{verbose,tmargin=1in,bmargin=1in,lmargin=1in,rmargin=1in}
    
    

\begin{document}
    
    \maketitle
    
    

    
    \hypertarget{criticidad-y-correlaciones-de-largo-alcance-entre-eventos-extremos-de-series-temporales-climuxe1ticas}{%
\section{Criticidad y correlaciones de largo alcance entre eventos
extremos de series temporales
climáticas}\label{criticidad-y-correlaciones-de-largo-alcance-entre-eventos-extremos-de-series-temporales-climuxe1ticas}}

    \hypertarget{introducciuxf3n}{%
\subsection{Introducción}\label{introducciuxf3n}}

    La criticidad en series temporales climáticas se refiere a la tendencia
de los sistemas climáticos a estar cerca de puntos críticos o
transiciones de fase, donde pequeños cambios pueden llevar a grandes
efectos. Este fenómeno es típico de sistemas complejos que, como el
clima, exhiben propiedades emergentes. En estos puntos críticos, los
sistemas climáticos muestran fluctuaciones extremas y comportamientos
inusuales, como eventos de calor extremo, sequías o tormentas intensas.
La criticidad también puede dar lugar a correlaciones de largo alcance,
donde los eventos en una parte del sistema pueden influir en otras
partes distantes, tanto en tiempo como en espacio.

Las correlaciones de largo alcance en el contexto climático implican que
los cambios en variables como la temperatura, la presión o la
precipitación en una región pueden estar conectados con eventos que
ocurren a cientos o miles de kilómetros de distancia o a largo plazo.
Esto se observa en fenómenos como El Niño, donde el calentamiento en el
Pacífico puede afectar los patrones climáticos en todo el mundo. Este
tipo de conexiones es crucial para entender cómo se propagan los
impactos climáticos, lo que a su vez requiere un enfoque de análisis de
series temporales que capte no solo las correlaciones inmediatas, sino
también las dependencias de largo plazo.

El análisis de series temporales climáticas, por lo tanto, debe abordar
tanto las dinámicas lineales como las no lineales. Las dinámicas
lineales son más fáciles de modelar y suelen describir fenómenos
climáticos más estables y predecibles, donde los cambios en las
variables tienen efectos proporcionales y controlables. Por otro lado,
las dinámicas no lineales son características de sistemas con
comportamientos más complejos, donde las pequeñas variaciones pueden
amplificarse a través de retroalimentaciones positivas o negativas, lo
que puede resultar en eventos inesperados y caóticos. En estos sistemas
no lineales, fenómenos como el caos climático o la aparición de patrones
fractales hacen que el pronóstico y el control del clima sean mucho más
difíciles.

La importancia de analizar tanto las dinámicas lineales como no lineales
radica en la capacidad de mejorar la precisión de los modelos climáticos
y detectar señales tempranas de eventos extremos o transiciones
críticas. Los enfoques tradicionales que solo se centran en la
linealidad pueden perder información valiosa sobre los mecanismos no
lineales que impulsan fenómenos importantes como la amplificación
ártica, los monzones, o la formación de huracanes. Al integrar ambas
perspectivas, se puede mejorar la capacidad de predicción y gestión de
riesgos climáticos, optimizando las políticas de adaptación y mitigación
frente al cambio climático.

Además, comprender las correlaciones de largo alcance puede ser clave
para desarrollar estrategias regionales y globales. Por ejemplo, si se
pueden identificar patrones críticos de advertencia temprana en una
región, otras áreas distantes podrían beneficiarse de esa información
para prepararse mejor. En este sentido, el análisis de series temporales
permite no solo una mejor comprensión del clima, sino también una
herramienta poderosa para la toma de decisiones en áreas como la
agricultura, la planificación urbana y la gestión de recursos hídricos.

En resumen, el estudio de la criticidad y las correlaciones de largo
alcance en las series temporales climáticas, junto con la identificación
de dinámicas lineales y no lineales, es crucial para una comprensión
profunda de los sistemas climáticos. Esto no solo mejora la precisión de
los modelos predictivos, sino que también proporciona información
valiosa para mitigar los impactos de eventos climáticos extremos y
adaptarse a un clima cambiante.

    \hypertarget{segmentaciuxf3n-de-datos-climuxe1ticos-por-municipio}{%
\section{Segmentación de datos climáticos por
municipio}\label{segmentaciuxf3n-de-datos-climuxe1ticos-por-municipio}}

    Antes de iniciar el análisis de datos, es fundamental segmentar los
datos en grupos específicos para facilitar los procesos computacionales.
La segmentación permite dividir grandes volúmenes de datos en
subconjuntos más manejables, lo que optimiza el uso de recursos
computacionales y reduce el tiempo de procesamiento. Al agrupar los
datos según criterios relevantes, como variables o ubicaciones, se
pueden aplicar técnicas de análisis de manera más eficiente y enfocada.
Esto no solo mejora el rendimiento del sistema al evitar la saturación
de memoria, sino que también simplifica la gestión y el análisis,
permitiendo una identificación más precisa de patrones y tendencias. En
resumen, la segmentación previa es una estrategia clave para asegurar un
análisis de datos más fluido, rápido y efectivo.

    \hypertarget{cargando-libreruxedas}{%
\subsection{Cargando librerías}\label{cargando-libreruxedas}}

    El punto de partida para analizar un conjunto de datos climáticos es
cargar las librerías necesarias para realizar gráficos, cálculos
estadísticos y ajustes avanzados. En Python, se utilizan herramientas
como \textbf{pandas} para la manipulación de datos, \textbf{matplotlib}
y \textbf{seaborn} para la visualización gráfica, y \textbf{numpy} para
realizar cálculos numéricos. Para análisis estadísticos más detallados,
se recurre a \textbf{scipy}, que ofrece pruebas estadísticas, y
\textbf{statsmodels} o \textbf{sklearn}, útiles para crear modelos de
regresión y otros análisis estadísticos.

    \begin{tcolorbox}[breakable, size=fbox, boxrule=1pt, pad at break*=1mm,colback=cellbackground, colframe=cellborder]
\prompt{In}{incolor}{2}{\boxspacing}
\begin{Verbatim}[commandchars=\\\{\}]
\PY{k+kn}{import} \PY{n+nn}{sys}
\PY{n}{sys}\PY{o}{.}\PY{n}{path}\PY{o}{.}\PY{n}{append}\PY{p}{(}\PY{l+s+s1}{\PYZsq{}}\PY{l+s+s1}{libmolina}\PY{l+s+s1}{\PYZsq{}}\PY{p}{)}
\PY{k+kn}{import} \PY{n+nn}{pandas} \PY{k}{as} \PY{n+nn}{pd}
\PY{k+kn}{import} \PY{n+nn}{os}
\PY{k+kn}{import} \PY{n+nn}{numpy} \PY{k}{as} \PY{n+nn}{np}
\PY{k+kn}{from} \PY{n+nn}{matplotlib}\PY{n+nn}{.}\PY{n+nn}{pylab} \PY{k+kn}{import} \PY{n}{plt}
\PY{k+kn}{import} \PY{n+nn}{seaborn} \PY{k}{as} \PY{n+nn}{sns}
\PY{k+kn}{import} \PY{n+nn}{matplotlib}
\PY{k+kn}{import} \PY{n+nn}{matplotlib}\PY{n+nn}{.}\PY{n+nn}{gridspec} \PY{k}{as} \PY{n+nn}{gridspec}
\PY{k+kn}{import} \PY{n+nn}{matplotlib} \PY{k}{as} \PY{n+nn}{mpl}
\PY{k+kn}{from} \PY{n+nn}{pathlib} \PY{k+kn}{import} \PY{n}{Path}
\PY{k+kn}{from} \PY{n+nn}{libmolina} \PY{k+kn}{import} \PY{o}{*} 
\PY{k+kn}{from} \PY{n+nn}{datetime} \PY{k+kn}{import} \PY{n}{datetime}
\PY{k+kn}{from} \PY{n+nn}{iminuit} \PY{k+kn}{import} \PY{n}{Minuit}
\PY{k+kn}{from} \PY{n+nn}{probfit} \PY{k+kn}{import} \PY{n}{Chi2Regression}
\PY{k+kn}{import} \PY{n+nn}{seaborn} \PY{k}{as} \PY{n+nn}{sns}
\PY{k+kn}{import} \PY{n+nn}{matplotlib}\PY{n+nn}{.}\PY{n+nn}{ticker} \PY{k}{as} \PY{n+nn}{mticker}
\PY{k+kn}{import} \PY{n+nn}{matplotlib}\PY{n+nn}{.}\PY{n+nn}{ticker} \PY{k}{as} \PY{n+nn}{ticker}
\PY{k+kn}{import} \PY{n+nn}{matplotlib}\PY{n+nn}{.}\PY{n+nn}{dates} \PY{k}{as} \PY{n+nn}{mdates}
\PY{k+kn}{from} \PY{n+nn}{PIL} \PY{k+kn}{import} \PY{n}{Image}
\PY{k+kn}{import} \PY{n+nn}{dataframe\PYZus{}image} \PY{k}{as} \PY{n+nn}{dfi}
\PY{k+kn}{from} \PY{n+nn}{scipy}\PY{n+nn}{.}\PY{n+nn}{stats} \PY{k+kn}{import} \PY{n}{pearsonr}
\PY{k+kn}{import} \PY{n+nn}{random}

\PY{c+c1}{\PYZsh{} import sklearn}
\PY{c+c1}{\PYZsh{} import statsmodels.formula.api as smf}
\end{Verbatim}
\end{tcolorbox}

    La opción \texttt{matplotlib.rcParams{[}"text.usetex"{]}\ =\ True}
permite usar LaTeX para renderizar texto en gráficos, lo que mejora la
calidad tipográfica y facilita la inclusión de ecuaciones matemáticas.
Esto es ideal para gráficos científicos que requieren una presentación
clara y precisa de fórmulas. Sin embargo, requiere tener instalado un
compilador de LaTeX y puede aumentar ligeramente el tiempo de generación
del gráfico debido a la compilación del texto.

    \begin{tcolorbox}[breakable, size=fbox, boxrule=1pt, pad at break*=1mm,colback=cellbackground, colframe=cellborder]
\prompt{In}{incolor}{3}{\boxspacing}
\begin{Verbatim}[commandchars=\\\{\}]
\PY{n}{matplotlib}\PY{o}{.}\PY{n}{rcParams}\PY{p}{[}\PY{l+s+s2}{\PYZdq{}}\PY{l+s+s2}{text.usetex}\PY{l+s+s2}{\PYZdq{}}\PY{p}{]} \PY{o}{=} \PY{k+kc}{True} \PY{c+c1}{\PYZsh{}Cargamos librerías LaTex}
\end{Verbatim}
\end{tcolorbox}

    \hypertarget{cargando-datos-por-bloques}{%
\subsection{Cargando datos por
bloques}\label{cargando-datos-por-bloques}}

    Usamos el código
\texttt{conjunto\_de\_datos\ =\ pd.read\_csv("datos\_aire.csv",\ dtype=str,\ chunksize=5000000)}
para cargar eficientemente un archivo con más de 20 millones de filas y
17 columnas de datos. Debido al gran tamaño del conjunto de datos,
optamos por cargarlo en \textbf{lotes} de 5 millones de filas,
utilizando la opción \texttt{chunksize}. Este enfoque es crucial porque
evita que la memoria física y el área de intercambio (swap) de nuestro
sistema se saturen, lo cual podría causar ralentizaciones o incluso
fallos en el procesamiento. Al procesar los datos en bloques,
optimizamos el uso de los recursos del sistema, permitiendo manejar
grandes volúmenes de datos sin comprometer su rendimiento.

Además, usamos el método \texttt{groupby("Variable")} para dividir los
datos según la ``Variable'' de estudio. Este proceso facilita la carga
ordenada y segmentada de los datos, permitiendo que cada subconjunto de
datos correspondiente a cada variable se almacene en un diccionario,
\texttt{grupos\_acumulados}. Este diccionario organiza los datos de
manera eficiente, agrupando las observaciones por variable, lo que nos
permite acceder fácilmente a los datos de cada categoría cuando los
necesitemos.

    \begin{tcolorbox}[breakable, size=fbox, boxrule=1pt, pad at break*=1mm,colback=cellbackground, colframe=cellborder]
\prompt{In}{incolor}{ }{\boxspacing}
\begin{Verbatim}[commandchars=\\\{\}]
\PY{n}{conjunto\PYZus{}de\PYZus{}datos} \PY{o}{=} \PY{n}{pd}\PY{o}{.}\PY{n}{read\PYZus{}csv}\PY{p}{(}
    \PY{l+s+s2}{\PYZdq{}}\PY{l+s+s2}{datos\PYZus{}aire.csv}\PY{l+s+s2}{\PYZdq{}}\PY{p}{,}
    \PY{n}{dtype}\PY{o}{=}\PY{n+nb}{str}\PY{p}{,}
    \PY{n}{chunksize}\PY{o}{=}\PY{l+m+mi}{5000000}\PY{p}{,}
\PY{p}{)}
\PY{n}{grupos\PYZus{}acumulados} \PY{o}{=} \PY{p}{\PYZob{}}\PY{p}{\PYZcb{}}
\PY{k}{for} \PY{n}{chunk} \PY{o+ow}{in} \PY{n}{conjunto\PYZus{}de\PYZus{}datos}\PY{p}{:}
    \PY{n}{grouped} \PY{o}{=} \PY{n}{chunk}\PY{o}{.}\PY{n}{groupby}\PY{p}{(}\PY{l+s+s2}{\PYZdq{}}\PY{l+s+s2}{Variable}\PY{l+s+s2}{\PYZdq{}}\PY{p}{)}
    \PY{k}{for} \PY{n}{nombre\PYZus{}grupo}\PY{p}{,} \PY{n}{grupo} \PY{o+ow}{in} \PY{n}{grouped}\PY{p}{:}
        \PY{k}{if} \PY{n}{nombre\PYZus{}grupo} \PY{o+ow}{in} \PY{n}{grupos\PYZus{}acumulados}\PY{p}{:}
            \PY{n}{grupos\PYZus{}acumulados}\PY{p}{[}\PY{n}{nombre\PYZus{}grupo}\PY{p}{]} \PY{o}{=} \PY{n}{pd}\PY{o}{.}\PY{n}{concat}\PY{p}{(}
                \PY{p}{[}\PY{n}{grupos\PYZus{}acumulados}\PY{p}{[}\PY{n}{nombre\PYZus{}grupo}\PY{p}{]}\PY{p}{,} \PY{n}{grupo}\PY{p}{]}
            \PY{p}{)}
        \PY{k}{else}\PY{p}{:}
            \PY{n}{grupos\PYZus{}acumulados}\PY{p}{[}\PY{n}{nombre\PYZus{}grupo}\PY{p}{]} \PY{o}{=} \PY{n}{grupo}
\end{Verbatim}
\end{tcolorbox}

    \hypertarget{variables-para-analizar-en-el-conjunto-de-datos-climuxe1ticos}{%
\subsection{Variables para analizar en el conjunto de datos
climáticos}\label{variables-para-analizar-en-el-conjunto-de-datos-climuxe1ticos}}

    Es fundamental extraer las variables presentes en los datos antes de
analizarlos porque estas representan las características o atributos
esenciales que definen el fenómeno que se estudia. Identificar y
entender las variables permite estructurar adecuadamente el análisis,
estableciendo relaciones entre ellas y determinando cuáles son
relevantes para las hipótesis que se desean probar. Además, conocer y
extraer las variables brinda un mejor control sobre el conjunto de
datos, permitiendo una gestión más eficiente de la información y
facilitando tareas como la limpieza de datos, la detección de valores
atípicos y la transformación de formatos. Asimismo, las variables de
estudio se pueden agrupar en un diccionario
\texttt{variables\_contaminacion} para organizar la información de
manera clara y permitir un acceso rápido y eficiente a estas variables
cuando se necesiten, lo que facilita su uso en posteriores análisis o
modelamientos.

    \begin{tcolorbox}[breakable, size=fbox, boxrule=1pt, pad at break*=1mm,colback=cellbackground, colframe=cellborder]
\prompt{In}{incolor}{4}{\boxspacing}
\begin{Verbatim}[commandchars=\\\{\}]
\PY{n}{variables\PYZus{}contaminacion} \PY{o}{=} \PY{p}{\PYZob{}}
    \PY{l+s+s2}{\PYZdq{}}\PY{l+s+s2}{co\PYZus{}concentracion}\PY{l+s+s2}{\PYZdq{}}\PY{p}{:} \PY{p}{\PYZob{}}\PY{l+s+s2}{\PYZdq{}}\PY{l+s+s2}{variable}\PY{l+s+s2}{\PYZdq{}}\PY{p}{:} \PY{l+s+s2}{\PYZdq{}}\PY{l+s+s2}{CO}\PY{l+s+s2}{\PYZdq{}}\PY{p}{,} \PY{l+s+s2}{\PYZdq{}}\PY{l+s+s2}{unidad}\PY{l+s+s2}{\PYZdq{}}\PY{p}{:} \PY{l+s+s2}{\PYZdq{}}\PY{l+s+s2}{\PYZdl{}}\PY{l+s+s2}{\PYZbs{}}\PY{l+s+s2}{mu\PYZdl{}g\PYZdl{}/\PYZdl{}m\PYZdl{}\PYZca{}3\PYZdl{}}\PY{l+s+s2}{\PYZdq{}}\PY{p}{,} \PY{l+s+s2}{\PYZdq{}}\PY{l+s+s2}{abreviatura}\PY{l+s+s2}{\PYZdq{}}\PY{p}{:} \PY{l+s+s2}{\PYZdq{}}\PY{l+s+s2}{\PYZdl{}CO\PYZdl{}}\PY{l+s+s2}{\PYZdq{}}\PY{p}{\PYZcb{}}\PY{p}{,}
    \PY{l+s+s2}{\PYZdq{}}\PY{l+s+s2}{direccion\PYZus{}viento}\PY{l+s+s2}{\PYZdq{}}\PY{p}{:} \PY{p}{\PYZob{}}\PY{l+s+s2}{\PYZdq{}}\PY{l+s+s2}{variable}\PY{l+s+s2}{\PYZdq{}}\PY{p}{:} \PY{l+s+s2}{\PYZdq{}}\PY{l+s+s2}{Dirección del Viento}\PY{l+s+s2}{\PYZdq{}}\PY{p}{,} \PY{l+s+s2}{\PYZdq{}}\PY{l+s+s2}{unidad}\PY{l+s+s2}{\PYZdq{}}\PY{p}{:} \PY{l+s+s2}{\PYZdq{}}\PY{l+s+s2}{\PYZdl{}\PYZca{}}\PY{l+s+s2}{\PYZob{}}\PY{l+s+s2}{\PYZbs{}}\PY{l+s+s2}{circ\PYZcb{}\PYZdl{}}\PY{l+s+s2}{\PYZdq{}}\PY{p}{,} \PY{l+s+s2}{\PYZdq{}}\PY{l+s+s2}{abreviatura}\PY{l+s+s2}{\PYZdq{}}\PY{p}{:} \PY{l+s+s2}{\PYZdq{}}\PY{l+s+s2}{\PYZdl{}WD\PYZdl{}}\PY{l+s+s2}{\PYZdq{}}\PY{p}{\PYZcb{}}\PY{p}{,}
    \PY{l+s+s2}{\PYZdq{}}\PY{l+s+s2}{humedad\PYZus{}relativa}\PY{l+s+s2}{\PYZdq{}}\PY{p}{:} \PY{p}{\PYZob{}}\PY{l+s+s2}{\PYZdq{}}\PY{l+s+s2}{variable}\PY{l+s+s2}{\PYZdq{}}\PY{p}{:} \PY{l+s+s2}{\PYZdq{}}\PY{l+s+s2}{Humedad Relativa}\PY{l+s+s2}{\PYZdq{}}\PY{p}{,} \PY{l+s+s2}{\PYZdq{}}\PY{l+s+s2}{unidad}\PY{l+s+s2}{\PYZdq{}}\PY{p}{:} \PY{l+s+s2}{\PYZdq{}}\PY{l+s+s2}{\PYZbs{}}\PY{l+s+s2}{\PYZpc{}}\PY{l+s+s2}{\PYZdq{}}\PY{p}{,} \PY{l+s+s2}{\PYZdq{}}\PY{l+s+s2}{abreviatura}\PY{l+s+s2}{\PYZdq{}}\PY{p}{:} \PY{l+s+s2}{\PYZdq{}}\PY{l+s+s2}{\PYZdl{}RH\PYZdl{}}\PY{l+s+s2}{\PYZdq{}}\PY{p}{\PYZcb{}}\PY{p}{,}
    \PY{l+s+s2}{\PYZdq{}}\PY{l+s+s2}{humedad\PYZus{}relativa\PYZus{}10m}\PY{l+s+s2}{\PYZdq{}}\PY{p}{:} \PY{p}{\PYZob{}}\PY{l+s+s2}{\PYZdq{}}\PY{l+s+s2}{variable}\PY{l+s+s2}{\PYZdq{}}\PY{p}{:} \PY{l+s+s2}{\PYZdq{}}\PY{l+s+s2}{Humedad Relativa 10 m}\PY{l+s+s2}{\PYZdq{}}\PY{p}{,} \PY{l+s+s2}{\PYZdq{}}\PY{l+s+s2}{unidad}\PY{l+s+s2}{\PYZdq{}}\PY{p}{:} \PY{l+s+s2}{\PYZdq{}}\PY{l+s+s2}{\PYZbs{}}\PY{l+s+s2}{\PYZpc{}}\PY{l+s+s2}{\PYZdq{}}\PY{p}{,} \PY{l+s+s2}{\PYZdq{}}\PY{l+s+s2}{abreviatura}\PY{l+s+s2}{\PYZdq{}}\PY{p}{:} \PY{l+s+s2}{\PYZdq{}}\PY{l+s+s2}{\PYZdl{}RH\PYZus{}}\PY{l+s+si}{\PYZob{}10\PYZcb{}}\PY{l+s+s2}{\PYZdl{}}\PY{l+s+s2}{\PYZdq{}}\PY{p}{\PYZcb{}}\PY{p}{,}
    \PY{l+s+s2}{\PYZdq{}}\PY{l+s+s2}{humedad\PYZus{}relativa\PYZus{}2m}\PY{l+s+s2}{\PYZdq{}}\PY{p}{:} \PY{p}{\PYZob{}}\PY{l+s+s2}{\PYZdq{}}\PY{l+s+s2}{variable}\PY{l+s+s2}{\PYZdq{}}\PY{p}{:} \PY{l+s+s2}{\PYZdq{}}\PY{l+s+s2}{Humedad Relativa 2 m}\PY{l+s+s2}{\PYZdq{}}\PY{p}{,} \PY{l+s+s2}{\PYZdq{}}\PY{l+s+s2}{unidad}\PY{l+s+s2}{\PYZdq{}}\PY{p}{:} \PY{l+s+s2}{\PYZdq{}}\PY{l+s+s2}{\PYZbs{}}\PY{l+s+s2}{\PYZpc{}}\PY{l+s+s2}{\PYZdq{}}\PY{p}{,} \PY{l+s+s2}{\PYZdq{}}\PY{l+s+s2}{abreviatura}\PY{l+s+s2}{\PYZdq{}}\PY{p}{:} \PY{l+s+s2}{\PYZdq{}}\PY{l+s+s2}{\PYZdl{}RH\PYZus{}}\PY{l+s+si}{\PYZob{}2\PYZcb{}}\PY{l+s+s2}{\PYZdl{}}\PY{l+s+s2}{\PYZdq{}}\PY{p}{\PYZcb{}}\PY{p}{,}
    \PY{l+s+s2}{\PYZdq{}}\PY{l+s+s2}{no\PYZus{}concentracion}\PY{l+s+s2}{\PYZdq{}}\PY{p}{:} \PY{p}{\PYZob{}}\PY{l+s+s2}{\PYZdq{}}\PY{l+s+s2}{variable}\PY{l+s+s2}{\PYZdq{}}\PY{p}{:} \PY{l+s+s2}{\PYZdq{}}\PY{l+s+s2}{NO}\PY{l+s+s2}{\PYZdq{}}\PY{p}{,} \PY{l+s+s2}{\PYZdq{}}\PY{l+s+s2}{unidad}\PY{l+s+s2}{\PYZdq{}}\PY{p}{:} \PY{l+s+s2}{\PYZdq{}}\PY{l+s+s2}{\PYZdl{}}\PY{l+s+s2}{\PYZbs{}}\PY{l+s+s2}{mu\PYZdl{}g\PYZdl{}/\PYZdl{}m\PYZdl{}\PYZca{}3\PYZdl{}}\PY{l+s+s2}{\PYZdq{}}\PY{p}{,} \PY{l+s+s2}{\PYZdq{}}\PY{l+s+s2}{abreviatura}\PY{l+s+s2}{\PYZdq{}}\PY{p}{:} \PY{l+s+s2}{\PYZdq{}}\PY{l+s+s2}{\PYZdl{}NO\PYZdl{}}\PY{l+s+s2}{\PYZdq{}}\PY{p}{\PYZcb{}}\PY{p}{,}
    \PY{l+s+s2}{\PYZdq{}}\PY{l+s+s2}{no2\PYZus{}concentracion}\PY{l+s+s2}{\PYZdq{}}\PY{p}{:} \PY{p}{\PYZob{}}\PY{l+s+s2}{\PYZdq{}}\PY{l+s+s2}{variable}\PY{l+s+s2}{\PYZdq{}}\PY{p}{:} \PY{l+s+s2}{\PYZdq{}}\PY{l+s+s2}{NO2}\PY{l+s+s2}{\PYZdq{}}\PY{p}{,} \PY{l+s+s2}{\PYZdq{}}\PY{l+s+s2}{unidad}\PY{l+s+s2}{\PYZdq{}}\PY{p}{:} \PY{l+s+s2}{\PYZdq{}}\PY{l+s+s2}{\PYZdl{}}\PY{l+s+s2}{\PYZbs{}}\PY{l+s+s2}{mu\PYZdl{}g\PYZdl{}/\PYZdl{}m\PYZdl{}\PYZca{}3\PYZdl{}}\PY{l+s+s2}{\PYZdq{}}\PY{p}{,} \PY{l+s+s2}{\PYZdq{}}\PY{l+s+s2}{abreviatura}\PY{l+s+s2}{\PYZdq{}}\PY{p}{:} \PY{l+s+s2}{\PYZdq{}}\PY{l+s+s2}{\PYZdl{}NO\PYZus{}2\PYZdl{}}\PY{l+s+s2}{\PYZdq{}}\PY{p}{\PYZcb{}}\PY{p}{,}
    \PY{l+s+s2}{\PYZdq{}}\PY{l+s+s2}{o3\PYZus{}concentracion}\PY{l+s+s2}{\PYZdq{}}\PY{p}{:} \PY{p}{\PYZob{}}\PY{l+s+s2}{\PYZdq{}}\PY{l+s+s2}{variable}\PY{l+s+s2}{\PYZdq{}}\PY{p}{:} \PY{l+s+s2}{\PYZdq{}}\PY{l+s+s2}{O3}\PY{l+s+s2}{\PYZdq{}}\PY{p}{,} \PY{l+s+s2}{\PYZdq{}}\PY{l+s+s2}{unidad}\PY{l+s+s2}{\PYZdq{}}\PY{p}{:} \PY{l+s+s2}{\PYZdq{}}\PY{l+s+s2}{\PYZdl{}}\PY{l+s+s2}{\PYZbs{}}\PY{l+s+s2}{mu\PYZdl{}g\PYZdl{}/\PYZdl{}m\PYZdl{}\PYZca{}3\PYZdl{}}\PY{l+s+s2}{\PYZdq{}}\PY{p}{,} \PY{l+s+s2}{\PYZdq{}}\PY{l+s+s2}{abreviatura}\PY{l+s+s2}{\PYZdq{}}\PY{p}{:} \PY{l+s+s2}{\PYZdq{}}\PY{l+s+s2}{\PYZdl{}O\PYZus{}3\PYZdl{}}\PY{l+s+s2}{\PYZdq{}}\PY{p}{\PYZcb{}}\PY{p}{,}
    \PY{l+s+s2}{\PYZdq{}}\PY{l+s+s2}{pm10\PYZus{}concentracion}\PY{l+s+s2}{\PYZdq{}}\PY{p}{:} \PY{p}{\PYZob{}}\PY{l+s+s2}{\PYZdq{}}\PY{l+s+s2}{variable}\PY{l+s+s2}{\PYZdq{}}\PY{p}{:} \PY{l+s+s2}{\PYZdq{}}\PY{l+s+s2}{PM10}\PY{l+s+s2}{\PYZdq{}}\PY{p}{,} \PY{l+s+s2}{\PYZdq{}}\PY{l+s+s2}{unidad}\PY{l+s+s2}{\PYZdq{}}\PY{p}{:} \PY{l+s+s2}{\PYZdq{}}\PY{l+s+s2}{\PYZdl{}}\PY{l+s+s2}{\PYZbs{}}\PY{l+s+s2}{mu\PYZdl{}g\PYZdl{}/\PYZdl{}m\PYZdl{}\PYZca{}3\PYZdl{}}\PY{l+s+s2}{\PYZdq{}}\PY{p}{,} \PY{l+s+s2}{\PYZdq{}}\PY{l+s+s2}{abreviatura}\PY{l+s+s2}{\PYZdq{}}\PY{p}{:} \PY{l+s+s2}{\PYZdq{}}\PY{l+s+s2}{\PYZdl{}PM\PYZus{}}\PY{l+s+si}{\PYZob{}10\PYZcb{}}\PY{l+s+s2}{\PYZdl{}}\PY{l+s+s2}{\PYZdq{}}\PY{p}{\PYZcb{}}\PY{p}{,}
    \PY{l+s+s2}{\PYZdq{}}\PY{l+s+s2}{pm25\PYZus{}concentracion}\PY{l+s+s2}{\PYZdq{}}\PY{p}{:} \PY{p}{\PYZob{}}\PY{l+s+s2}{\PYZdq{}}\PY{l+s+s2}{variable}\PY{l+s+s2}{\PYZdq{}}\PY{p}{:} \PY{l+s+s2}{\PYZdq{}}\PY{l+s+s2}{PM2.5}\PY{l+s+s2}{\PYZdq{}}\PY{p}{,} \PY{l+s+s2}{\PYZdq{}}\PY{l+s+s2}{unidad}\PY{l+s+s2}{\PYZdq{}}\PY{p}{:} \PY{l+s+s2}{\PYZdq{}}\PY{l+s+s2}{\PYZdl{}}\PY{l+s+s2}{\PYZbs{}}\PY{l+s+s2}{mu\PYZdl{}g\PYZdl{}/\PYZdl{}m\PYZdl{}\PYZca{}3\PYZdl{}}\PY{l+s+s2}{\PYZdq{}}\PY{p}{,} \PY{l+s+s2}{\PYZdq{}}\PY{l+s+s2}{abreviatura}\PY{l+s+s2}{\PYZdq{}}\PY{p}{:} \PY{l+s+s2}{\PYZdq{}}\PY{l+s+s2}{\PYZdl{}PM\PYZus{}}\PY{l+s+si}{\PYZob{}2.5\PYZcb{}}\PY{l+s+s2}{\PYZdl{}}\PY{l+s+s2}{\PYZdq{}}\PY{p}{\PYZcb{}}\PY{p}{,}
    \PY{l+s+s2}{\PYZdq{}}\PY{l+s+s2}{pst\PYZus{}concentracion}\PY{l+s+s2}{\PYZdq{}}\PY{p}{:} \PY{p}{\PYZob{}}\PY{l+s+s2}{\PYZdq{}}\PY{l+s+s2}{variable}\PY{l+s+s2}{\PYZdq{}}\PY{p}{:} \PY{l+s+s2}{\PYZdq{}}\PY{l+s+s2}{PST}\PY{l+s+s2}{\PYZdq{}}\PY{p}{,} \PY{l+s+s2}{\PYZdq{}}\PY{l+s+s2}{unidad}\PY{l+s+s2}{\PYZdq{}}\PY{p}{:} \PY{l+s+s2}{\PYZdq{}}\PY{l+s+s2}{\PYZdl{}}\PY{l+s+s2}{\PYZbs{}}\PY{l+s+s2}{mu\PYZdl{}g\PYZdl{}/\PYZdl{}m\PYZdl{}\PYZca{}3\PYZdl{}}\PY{l+s+s2}{\PYZdq{}}\PY{p}{,} \PY{l+s+s2}{\PYZdq{}}\PY{l+s+s2}{abreviatura}\PY{l+s+s2}{\PYZdq{}}\PY{p}{:} \PY{l+s+s2}{\PYZdq{}}\PY{l+s+s2}{\PYZdl{}PST\PYZdl{}}\PY{l+s+s2}{\PYZdq{}}\PY{p}{\PYZcb{}}\PY{p}{,}
    \PY{l+s+s2}{\PYZdq{}}\PY{l+s+s2}{precipitacion\PYZus{}liquida}\PY{l+s+s2}{\PYZdq{}}\PY{p}{:} \PY{p}{\PYZob{}}\PY{l+s+s2}{\PYZdq{}}\PY{l+s+s2}{variable}\PY{l+s+s2}{\PYZdq{}}\PY{p}{:} \PY{l+s+s2}{\PYZdq{}}\PY{l+s+s2}{Precipitación Líquida}\PY{l+s+s2}{\PYZdq{}}\PY{p}{,} \PY{l+s+s2}{\PYZdq{}}\PY{l+s+s2}{unidad}\PY{l+s+s2}{\PYZdq{}}\PY{p}{:} \PY{l+s+s2}{\PYZdq{}}\PY{l+s+s2}{mm}\PY{l+s+s2}{\PYZdq{}}\PY{p}{,} \PY{l+s+s2}{\PYZdq{}}\PY{l+s+s2}{abreviatura}\PY{l+s+s2}{\PYZdq{}}\PY{p}{:} \PY{l+s+s2}{\PYZdq{}}\PY{l+s+s2}{\PYZdl{}P\PYZus{}}\PY{l+s+si}{\PYZob{}liq\PYZcb{}}\PY{l+s+s2}{\PYZdl{}}\PY{l+s+s2}{\PYZdq{}}\PY{p}{\PYZcb{}}\PY{p}{,}
    \PY{l+s+s2}{\PYZdq{}}\PY{l+s+s2}{presion\PYZus{}atmosferica}\PY{l+s+s2}{\PYZdq{}}\PY{p}{:} \PY{p}{\PYZob{}}\PY{l+s+s2}{\PYZdq{}}\PY{l+s+s2}{variable}\PY{l+s+s2}{\PYZdq{}}\PY{p}{:} \PY{l+s+s2}{\PYZdq{}}\PY{l+s+s2}{Presión Atmosférica}\PY{l+s+s2}{\PYZdq{}}\PY{p}{,} \PY{l+s+s2}{\PYZdq{}}\PY{l+s+s2}{unidad}\PY{l+s+s2}{\PYZdq{}}\PY{p}{:} \PY{l+s+s2}{\PYZdq{}}\PY{l+s+s2}{mmHg}\PY{l+s+s2}{\PYZdq{}}\PY{p}{,} \PY{l+s+s2}{\PYZdq{}}\PY{l+s+s2}{abreviatura}\PY{l+s+s2}{\PYZdq{}}\PY{p}{:} \PY{l+s+s2}{\PYZdq{}}\PY{l+s+s2}{\PYZdl{}P\PYZus{}}\PY{l+s+si}{\PYZob{}atm\PYZcb{}}\PY{l+s+s2}{\PYZdl{}}\PY{l+s+s2}{\PYZdq{}}\PY{p}{\PYZcb{}}\PY{p}{,}
    \PY{l+s+s2}{\PYZdq{}}\PY{l+s+s2}{radiacion\PYZus{}solar\PYZus{}global}\PY{l+s+s2}{\PYZdq{}}\PY{p}{:} \PY{p}{\PYZob{}}
        \PY{l+s+s2}{\PYZdq{}}\PY{l+s+s2}{variable}\PY{l+s+s2}{\PYZdq{}}\PY{p}{:} \PY{l+s+s2}{\PYZdq{}}\PY{l+s+s2}{Radiación Solar Global}\PY{l+s+s2}{\PYZdq{}}\PY{p}{,} \PY{l+s+s2}{\PYZdq{}}\PY{l+s+s2}{unidad}\PY{l+s+s2}{\PYZdq{}}\PY{p}{:} \PY{l+s+s2}{\PYZdq{}}\PY{l+s+s2}{W\PYZdl{}/\PYZdl{}m\PYZdl{}\PYZca{}2\PYZdl{}}\PY{l+s+s2}{\PYZdq{}}\PY{p}{,} \PY{l+s+s2}{\PYZdq{}}\PY{l+s+s2}{abreviatura}\PY{l+s+s2}{\PYZdq{}}\PY{p}{:} \PY{l+s+s2}{\PYZdq{}}\PY{l+s+s2}{\PYZdl{}R\PYZus{}}\PY{l+s+si}{\PYZob{}solar\PYZcb{}}\PY{l+s+s2}{\PYZdl{}}\PY{l+s+s2}{\PYZdq{}}
    \PY{p}{\PYZcb{}}\PY{p}{,}
    \PY{l+s+s2}{\PYZdq{}}\PY{l+s+s2}{radiacion\PYZus{}uvb}\PY{l+s+s2}{\PYZdq{}}\PY{p}{:} \PY{p}{\PYZob{}}\PY{l+s+s2}{\PYZdq{}}\PY{l+s+s2}{variable}\PY{l+s+s2}{\PYZdq{}}\PY{p}{:} \PY{l+s+s2}{\PYZdq{}}\PY{l+s+s2}{Radiación UVB}\PY{l+s+s2}{\PYZdq{}}\PY{p}{,} \PY{l+s+s2}{\PYZdq{}}\PY{l+s+s2}{unidad}\PY{l+s+s2}{\PYZdq{}}\PY{p}{:} \PY{l+s+s2}{\PYZdq{}}\PY{l+s+s2}{MED\PYZdl{}/\PYZdl{}h}\PY{l+s+s2}{\PYZdq{}}\PY{p}{,} \PY{l+s+s2}{\PYZdq{}}\PY{l+s+s2}{abreviatura}\PY{l+s+s2}{\PYZdq{}}\PY{p}{:} \PY{l+s+s2}{\PYZdq{}}\PY{l+s+s2}{\PYZdl{}UVB\PYZdl{}}\PY{l+s+s2}{\PYZdq{}}\PY{p}{\PYZcb{}}\PY{p}{,}
    \PY{l+s+s2}{\PYZdq{}}\PY{l+s+s2}{so2\PYZus{}concentracion}\PY{l+s+s2}{\PYZdq{}}\PY{p}{:} \PY{p}{\PYZob{}}\PY{l+s+s2}{\PYZdq{}}\PY{l+s+s2}{variable}\PY{l+s+s2}{\PYZdq{}}\PY{p}{:} \PY{l+s+s2}{\PYZdq{}}\PY{l+s+s2}{SO2}\PY{l+s+s2}{\PYZdq{}}\PY{p}{,} \PY{l+s+s2}{\PYZdq{}}\PY{l+s+s2}{unidad}\PY{l+s+s2}{\PYZdq{}}\PY{p}{:} \PY{l+s+s2}{\PYZdq{}}\PY{l+s+s2}{\PYZdl{}}\PY{l+s+s2}{\PYZbs{}}\PY{l+s+s2}{mu\PYZdl{}g\PYZdl{}/\PYZdl{}m\PYZdl{}\PYZca{}3\PYZdl{}}\PY{l+s+s2}{\PYZdq{}}\PY{p}{,} \PY{l+s+s2}{\PYZdq{}}\PY{l+s+s2}{abreviatura}\PY{l+s+s2}{\PYZdq{}}\PY{p}{:} \PY{l+s+s2}{\PYZdq{}}\PY{l+s+s2}{\PYZdl{}SO\PYZus{}2\PYZdl{}}\PY{l+s+s2}{\PYZdq{}}\PY{p}{\PYZcb{}}\PY{p}{,}
    \PY{l+s+s2}{\PYZdq{}}\PY{l+s+s2}{temperatura}\PY{l+s+s2}{\PYZdq{}}\PY{p}{:} \PY{p}{\PYZob{}}\PY{l+s+s2}{\PYZdq{}}\PY{l+s+s2}{variable}\PY{l+s+s2}{\PYZdq{}}\PY{p}{:} \PY{l+s+s2}{\PYZdq{}}\PY{l+s+s2}{Temperatura}\PY{l+s+s2}{\PYZdq{}}\PY{p}{,} \PY{l+s+s2}{\PYZdq{}}\PY{l+s+s2}{unidad}\PY{l+s+s2}{\PYZdq{}}\PY{p}{:} \PY{l+s+s2}{\PYZdq{}}\PY{l+s+s2}{\PYZdl{}\PYZca{}}\PY{l+s+s2}{\PYZob{}}\PY{l+s+s2}{\PYZbs{}}\PY{l+s+s2}{circ\PYZcb{}\PYZdl{}C}\PY{l+s+s2}{\PYZdq{}}\PY{p}{,} \PY{l+s+s2}{\PYZdq{}}\PY{l+s+s2}{abreviatura}\PY{l+s+s2}{\PYZdq{}}\PY{p}{:} \PY{l+s+s2}{\PYZdq{}}\PY{l+s+s2}{\PYZdl{}T\PYZdl{}}\PY{l+s+s2}{\PYZdq{}}\PY{p}{\PYZcb{}}\PY{p}{,}
    \PY{l+s+s2}{\PYZdq{}}\PY{l+s+s2}{temperatura\PYZus{}10m}\PY{l+s+s2}{\PYZdq{}}\PY{p}{:} \PY{p}{\PYZob{}}\PY{l+s+s2}{\PYZdq{}}\PY{l+s+s2}{variable}\PY{l+s+s2}{\PYZdq{}}\PY{p}{:} \PY{l+s+s2}{\PYZdq{}}\PY{l+s+s2}{Temperatura a 10 m}\PY{l+s+s2}{\PYZdq{}}\PY{p}{,} \PY{l+s+s2}{\PYZdq{}}\PY{l+s+s2}{unidad}\PY{l+s+s2}{\PYZdq{}}\PY{p}{:} \PY{l+s+s2}{\PYZdq{}}\PY{l+s+s2}{\PYZdl{}\PYZca{}}\PY{l+s+s2}{\PYZob{}}\PY{l+s+s2}{\PYZbs{}}\PY{l+s+s2}{circ\PYZcb{}\PYZdl{}C}\PY{l+s+s2}{\PYZdq{}}\PY{p}{,} \PY{l+s+s2}{\PYZdq{}}\PY{l+s+s2}{abreviatura}\PY{l+s+s2}{\PYZdq{}}\PY{p}{:} \PY{l+s+s2}{\PYZdq{}}\PY{l+s+s2}{\PYZdl{}T\PYZus{}}\PY{l+s+si}{\PYZob{}10\PYZcb{}}\PY{l+s+s2}{\PYZdl{}}\PY{l+s+s2}{\PYZdq{}}\PY{p}{\PYZcb{}}\PY{p}{,}
    \PY{l+s+s2}{\PYZdq{}}\PY{l+s+s2}{temperatura\PYZus{}2m}\PY{l+s+s2}{\PYZdq{}}\PY{p}{:} \PY{p}{\PYZob{}}\PY{l+s+s2}{\PYZdq{}}\PY{l+s+s2}{variable}\PY{l+s+s2}{\PYZdq{}}\PY{p}{:} \PY{l+s+s2}{\PYZdq{}}\PY{l+s+s2}{Temperatura a 2 m}\PY{l+s+s2}{\PYZdq{}}\PY{p}{,} \PY{l+s+s2}{\PYZdq{}}\PY{l+s+s2}{unidad}\PY{l+s+s2}{\PYZdq{}}\PY{p}{:} \PY{l+s+s2}{\PYZdq{}}\PY{l+s+s2}{\PYZdl{}\PYZca{}}\PY{l+s+s2}{\PYZob{}}\PY{l+s+s2}{\PYZbs{}}\PY{l+s+s2}{circ\PYZcb{}\PYZdl{}C}\PY{l+s+s2}{\PYZdq{}}\PY{p}{,} \PY{l+s+s2}{\PYZdq{}}\PY{l+s+s2}{abreviatura}\PY{l+s+s2}{\PYZdq{}}\PY{p}{:} \PY{l+s+s2}{\PYZdq{}}\PY{l+s+s2}{\PYZdl{}T\PYZus{}}\PY{l+s+si}{\PYZob{}2\PYZcb{}}\PY{l+s+s2}{\PYZdl{}}\PY{l+s+s2}{\PYZdq{}}\PY{p}{\PYZcb{}}\PY{p}{,}
    \PY{l+s+s2}{\PYZdq{}}\PY{l+s+s2}{velocidad\PYZus{}viento}\PY{l+s+s2}{\PYZdq{}}\PY{p}{:} \PY{p}{\PYZob{}}\PY{l+s+s2}{\PYZdq{}}\PY{l+s+s2}{variable}\PY{l+s+s2}{\PYZdq{}}\PY{p}{:} \PY{l+s+s2}{\PYZdq{}}\PY{l+s+s2}{Velocidad del Viento}\PY{l+s+s2}{\PYZdq{}}\PY{p}{,} \PY{l+s+s2}{\PYZdq{}}\PY{l+s+s2}{unidad}\PY{l+s+s2}{\PYZdq{}}\PY{p}{:} \PY{l+s+s2}{\PYZdq{}}\PY{l+s+s2}{m\PYZdl{}/\PYZdl{}s}\PY{l+s+s2}{\PYZdq{}}\PY{p}{,} \PY{l+s+s2}{\PYZdq{}}\PY{l+s+s2}{abreviatura}\PY{l+s+s2}{\PYZdq{}}\PY{p}{:} \PY{l+s+s2}{\PYZdq{}}\PY{l+s+s2}{\PYZdl{}WS\PYZdl{}}\PY{l+s+s2}{\PYZdq{}}\PY{p}{\PYZcb{}}
\PY{p}{\PYZcb{}}
\end{Verbatim}
\end{tcolorbox}

    \hypertarget{estandarizando-fechas-en-los-datos}{%
\subsection{Estandarizando fechas en los
datos}\label{estandarizando-fechas-en-los-datos}}

    Una vez cargadas las casi 3.5G de datos, notamos que las mediciones
temporales presentan varios problemas en la columna de fechas. No solo
están mezcladas entre el formato de 24 horas y el de 12 horas (AM/PM),
sino que además están escritas de forma incorrecta, lo que las hace
incompatibles con la función \texttt{to\_datetime()} de pandas. Estas
inconsistencias incluyen fechas mal formateadas o en un orden que no es
estándar, lo que impide que pandas las procese correctamente.

Para resolver este problema, es necesario desarrollar un código que
estandarice las fechas, asegurando que todas sigan un formato coherente.
Esto implica corregir los errores de formato y unificar los datos entre
los sistemas de 12 y 24 horas usando \texttt{pd.to\_datetime()}. Esta
conversión es esencial porque el formato mezclado limita la capacidad de
realizar operaciones temporales avanzadas, como la agrupación, la
comparación y la manipulación de fechas. Al transformar la columna a un
solo formato \texttt{datetime}, se habilitan funciones clave para el
análisis temporal, como la identificación de tendencias, la agrupación
por intervalos de tiempo, y la detección de eventos anómalos. Las 53
lineas de código que se exponen a continuación, muestra en detalle el
proceso:

    \begin{tcolorbox}[breakable, size=fbox, boxrule=1pt, pad at break*=1mm,colback=cellbackground, colframe=cellborder]
\prompt{In}{incolor}{ }{\boxspacing}
\begin{Verbatim}[commandchars=\\\{\}]
\PY{n}{grupos\PYZus{}acumulados\PYZus{}condicional} \PY{o}{=} \PY{p}{\PYZob{}}\PY{p}{\PYZcb{}}
\PY{k}{for} \PY{n}{us}\PY{p}{,} \PY{n}{ot} \PY{o+ow}{in} \PY{n+nb}{enumerate}\PY{p}{(}\PY{n+nb}{list}\PY{p}{(}\PY{n}{variables\PYZus{}contaminacion}\PY{o}{.}\PY{n}{keys}\PY{p}{(}\PY{p}{)}\PY{p}{)}\PY{p}{)}\PY{p}{:}
    \PY{n}{vrclima} \PY{o}{=} \PY{n}{variables\PYZus{}contaminacion}\PY{p}{[}\PY{n}{ot}\PY{p}{]}\PY{p}{[}\PY{l+s+s2}{\PYZdq{}}\PY{l+s+s2}{variable}\PY{l+s+s2}{\PYZdq{}}\PY{p}{]}
    \PY{n}{varPM} \PY{o}{=} \PY{n}{grupos\PYZus{}acumulados}\PY{p}{[}\PY{n}{vrclima}\PY{p}{]}\PY{p}{[}\PY{l+s+s2}{\PYZdq{}}\PY{l+s+s2}{Fecha}\PY{l+s+s2}{\PYZdq{}}\PY{p}{]}\PY{o}{.}\PY{n}{str}\PY{o}{.}\PY{n}{replace}\PY{p}{(}\PY{l+s+s2}{\PYZdq{}}\PY{l+s+s2}{p. m.}\PY{l+s+s2}{\PYZdq{}}\PY{p}{,} \PY{l+s+s2}{\PYZdq{}}\PY{l+s+s2}{PM}\PY{l+s+s2}{\PYZdq{}}\PY{p}{)}
    \PY{n}{grupos\PYZus{}acumulados}\PY{p}{[}\PY{n}{vrclima}\PY{p}{]}\PY{p}{[}\PY{l+s+s2}{\PYZdq{}}\PY{l+s+s2}{Fecha}\PY{l+s+s2}{\PYZdq{}}\PY{p}{]} \PY{o}{=} \PY{n}{varPM}
    \PY{n}{varAM} \PY{o}{=} \PY{n}{grupos\PYZus{}acumulados}\PY{p}{[}\PY{n}{vrclima}\PY{p}{]}\PY{p}{[}\PY{l+s+s2}{\PYZdq{}}\PY{l+s+s2}{Fecha}\PY{l+s+s2}{\PYZdq{}}\PY{p}{]}\PY{o}{.}\PY{n}{str}\PY{o}{.}\PY{n}{replace}\PY{p}{(}\PY{l+s+s2}{\PYZdq{}}\PY{l+s+s2}{a. m.}\PY{l+s+s2}{\PYZdq{}}\PY{p}{,} \PY{l+s+s2}{\PYZdq{}}\PY{l+s+s2}{AM}\PY{l+s+s2}{\PYZdq{}}\PY{p}{)}
    \PY{n}{grupos\PYZus{}acumulados}\PY{p}{[}\PY{n}{vrclima}\PY{p}{]}\PY{p}{[}\PY{l+s+s2}{\PYZdq{}}\PY{l+s+s2}{Fecha}\PY{l+s+s2}{\PYZdq{}}\PY{p}{]} \PY{o}{=} \PY{n}{varAM}
    \PY{n}{vrcondicion} \PY{o}{=} \PY{n}{grupos\PYZus{}acumulados}\PY{p}{[}\PY{n}{vrclima}\PY{p}{]}\PY{p}{[}\PY{l+s+s2}{\PYZdq{}}\PY{l+s+s2}{Fecha}\PY{l+s+s2}{\PYZdq{}}\PY{p}{]}
    \PY{n}{grupos\PYZus{}acumulados}\PY{p}{[}\PY{n}{vrclima}\PY{p}{]}\PY{p}{[}\PY{l+s+s2}{\PYZdq{}}\PY{l+s+s2}{Condicional}\PY{l+s+s2}{\PYZdq{}}\PY{p}{]} \PY{o}{=} \PY{n}{vrcondicion}\PY{o}{.}\PY{n}{str}\PY{o}{.}\PY{n}{contains}\PY{p}{(}\PY{l+s+s2}{\PYZdq{}}\PY{l+s+s2}{M}\PY{l+s+s2}{\PYZdq{}}\PY{p}{)}
    \PY{n}{remplazofv} \PY{o}{=} \PY{n}{grupos\PYZus{}acumulados}\PY{p}{[}\PY{n}{vrclima}\PY{p}{]}\PY{p}{[}\PY{l+s+s2}{\PYZdq{}}\PY{l+s+s2}{Condicional}\PY{l+s+s2}{\PYZdq{}}\PY{p}{]}\PY{o}{.}\PY{n}{map}\PY{p}{(}
        \PY{p}{\PYZob{}}\PY{k+kc}{True}\PY{p}{:} \PY{l+s+s2}{\PYZdq{}}\PY{l+s+s2}{Fhora\PYZus{}AMPM}\PY{l+s+s2}{\PYZdq{}}\PY{p}{,} \PY{k+kc}{False}\PY{p}{:} \PY{l+s+s2}{\PYZdq{}}\PY{l+s+s2}{Fhora\PYZus{}24HH}\PY{l+s+s2}{\PYZdq{}}\PY{p}{\PYZcb{}}
    \PY{p}{)}
    \PY{n}{grupos\PYZus{}acumulados}\PY{p}{[}\PY{n}{vrclima}\PY{p}{]}\PY{p}{[}\PY{l+s+s2}{\PYZdq{}}\PY{l+s+s2}{Condicional}\PY{l+s+s2}{\PYZdq{}}\PY{p}{]} \PY{o}{=} \PY{n}{remplazofv}
    \PY{n}{grupo\PYZus{}condicional} \PY{o}{=} \PY{n}{grupos\PYZus{}acumulados}\PY{p}{[}\PY{n}{vrclima}\PY{p}{]}\PY{o}{.}\PY{n}{groupby}\PY{p}{(}\PY{l+s+s2}{\PYZdq{}}\PY{l+s+s2}{Condicional}\PY{l+s+s2}{\PYZdq{}}\PY{p}{)}
    \PY{k}{for} \PY{n}{nombre\PYZus{}grupo}\PY{p}{,} \PY{n}{grupo} \PY{o+ow}{in} \PY{n}{grupo\PYZus{}condicional}\PY{p}{:}
        \PY{k}{if} \PY{n}{nombre\PYZus{}grupo} \PY{o+ow}{in} \PY{n}{grupos\PYZus{}acumulados\PYZus{}condicional}\PY{p}{:}
            \PY{n}{grupos\PYZus{}acumulados\PYZus{}condicional}\PY{p}{[}\PY{n}{nombre\PYZus{}grupo}\PY{p}{]} \PY{o}{=} \PY{n}{pd}\PY{o}{.}\PY{n}{concat}\PY{p}{(}
                \PY{p}{[}\PY{n}{grupos\PYZus{}acumulados\PYZus{}condicional}\PY{p}{[}\PY{n}{nombre\PYZus{}grupo}\PY{p}{]}\PY{p}{,} \PY{n}{grupo}\PY{p}{]}
            \PY{p}{)}
        \PY{k}{else}\PY{p}{:}
            \PY{n}{grupos\PYZus{}acumulados\PYZus{}condicional}\PY{p}{[}\PY{n}{nombre\PYZus{}grupo}\PY{p}{]} \PY{o}{=} \PY{n}{grupo}
\PY{n}{grupo\PYZus{}condicional\PYZus{}acumulados} \PY{o}{=} \PY{n}{pd}\PY{o}{.}\PY{n}{concat}\PY{p}{(}\PY{n}{grupos\PYZus{}acumulados\PYZus{}condicional}\PY{o}{.}\PY{n}{values}\PY{p}{(}\PY{p}{)}\PY{p}{)}
\PY{n}{grup\PYZus{}condicional} \PY{o}{=} \PY{n}{grupo\PYZus{}condicional\PYZus{}acumulados}\PY{o}{.}\PY{n}{groupby}\PY{p}{(}\PY{l+s+s2}{\PYZdq{}}\PY{l+s+s2}{Condicional}\PY{l+s+s2}{\PYZdq{}}\PY{p}{)}
\PY{n}{grupos\PYZus{}formatoHora\PYZus{}12h\PYZus{}24H} \PY{o}{=} \PY{p}{\PYZob{}}\PY{p}{\PYZcb{}}
\PY{k}{for} \PY{n}{nombre\PYZus{}grupo}\PY{p}{,} \PY{n}{grupo} \PY{o+ow}{in} \PY{n}{grup\PYZus{}condicional}\PY{p}{:}
    \PY{k}{if} \PY{n}{nombre\PYZus{}grupo} \PY{o+ow}{in} \PY{n}{grupos\PYZus{}formatoHora\PYZus{}12h\PYZus{}24H}\PY{p}{:}
        \PY{n}{grupos\PYZus{}formatoHora\PYZus{}12h\PYZus{}24H}\PY{p}{[}\PY{n}{nombre\PYZus{}grupo}\PY{p}{]} \PY{o}{=} \PY{n}{pd}\PY{o}{.}\PY{n}{concat}\PY{p}{(}
            \PY{p}{[}\PY{n}{grupos\PYZus{}formatoHora\PYZus{}12h\PYZus{}24H}\PY{p}{[}\PY{n}{nombre\PYZus{}grupo}\PY{p}{]}\PY{p}{,} \PY{n}{grupo}\PY{p}{]}
        \PY{p}{)}
    \PY{k}{else}\PY{p}{:}
        \PY{n}{grupos\PYZus{}formatoHora\PYZus{}12h\PYZus{}24H}\PY{p}{[}\PY{n}{nombre\PYZus{}grupo}\PY{p}{]} \PY{o}{=} \PY{n}{grupo}

\PY{n}{grupos\PYZus{}formatoHora\PYZus{}12h\PYZus{}24H}\PY{p}{[}\PY{l+s+s2}{\PYZdq{}}\PY{l+s+s2}{Fhora\PYZus{}AMPM}\PY{l+s+s2}{\PYZdq{}}\PY{p}{]}\PY{p}{[}\PY{l+s+s2}{\PYZdq{}}\PY{l+s+s2}{Fecha\PYZus{}24h}\PY{l+s+s2}{\PYZdq{}}\PY{p}{]} \PY{o}{=} \PY{n}{pd}\PY{o}{.}\PY{n}{to\PYZus{}datetime}\PY{p}{(}
    \PY{n}{grupos\PYZus{}formatoHora\PYZus{}12h\PYZus{}24H}\PY{p}{[}\PY{l+s+s2}{\PYZdq{}}\PY{l+s+s2}{Fhora\PYZus{}AMPM}\PY{l+s+s2}{\PYZdq{}}\PY{p}{]}\PY{p}{[}\PY{l+s+s2}{\PYZdq{}}\PY{l+s+s2}{Fecha}\PY{l+s+s2}{\PYZdq{}}\PY{p}{]}\PY{p}{,}
    \PY{n+nb}{format}\PY{o}{=}\PY{l+s+s2}{\PYZdq{}}\PY{l+s+si}{\PYZpc{}d}\PY{l+s+s2}{/}\PY{l+s+s2}{\PYZpc{}}\PY{l+s+s2}{m/}\PY{l+s+s2}{\PYZpc{}}\PY{l+s+s2}{Y }\PY{l+s+s2}{\PYZpc{}}\PY{l+s+s2}{I:}\PY{l+s+s2}{\PYZpc{}}\PY{l+s+s2}{M:}\PY{l+s+s2}{\PYZpc{}}\PY{l+s+s2}{S }\PY{l+s+s2}{\PYZpc{}}\PY{l+s+s2}{p}\PY{l+s+s2}{\PYZdq{}}\PY{p}{,}
    \PY{n}{errors}\PY{o}{=}\PY{l+s+s2}{\PYZdq{}}\PY{l+s+s2}{coerce}\PY{l+s+s2}{\PYZdq{}}\PY{p}{,}
    \PY{n}{dayfirst}\PY{o}{=}\PY{k+kc}{True}\PY{p}{,}
\PY{p}{)}

\PY{n}{fecha\PYZus{}minima} \PY{o}{=} \PY{n}{grupos\PYZus{}formatoHora\PYZus{}12h\PYZus{}24H}\PY{p}{[}\PY{l+s+s2}{\PYZdq{}}\PY{l+s+s2}{Fhora\PYZus{}AMPM}\PY{l+s+s2}{\PYZdq{}}\PY{p}{]}\PY{p}{[}\PY{l+s+s2}{\PYZdq{}}\PY{l+s+s2}{Fecha\PYZus{}24h}\PY{l+s+s2}{\PYZdq{}}\PY{p}{]}\PY{o}{.}\PY{n}{min}\PY{p}{(}\PY{p}{)}
\PY{n}{fecha\PYZus{}referencia} \PY{o}{=} \PY{n}{pd}\PY{o}{.}\PY{n}{to\PYZus{}datetime}\PY{p}{(}
    \PY{n}{fecha\PYZus{}minima}\PY{p}{,}
    \PY{n+nb}{format}\PY{o}{=}\PY{l+s+s2}{\PYZdq{}}\PY{l+s+si}{\PYZpc{}d}\PY{l+s+s2}{/}\PY{l+s+s2}{\PYZpc{}}\PY{l+s+s2}{m/}\PY{l+s+s2}{\PYZpc{}}\PY{l+s+s2}{Y }\PY{l+s+s2}{\PYZpc{}}\PY{l+s+s2}{H:}\PY{l+s+s2}{\PYZpc{}}\PY{l+s+s2}{M:}\PY{l+s+s2}{\PYZpc{}}\PY{l+s+s2}{S}\PY{l+s+s2}{\PYZdq{}}\PY{p}{,}
    \PY{n}{errors}\PY{o}{=}\PY{l+s+s2}{\PYZdq{}}\PY{l+s+s2}{coerce}\PY{l+s+s2}{\PYZdq{}}\PY{p}{,}
    \PY{n}{dayfirst}\PY{o}{=}\PY{k+kc}{True}\PY{p}{,}
\PY{p}{)}
\PY{c+c1}{\PYZsh{} Calcular la diferencia en horas entre cada fecha y la hora de referencia}
\PY{n}{grupos\PYZus{}formatoHora\PYZus{}12h\PYZus{}24H}\PY{p}{[}\PY{l+s+s2}{\PYZdq{}}\PY{l+s+s2}{Fhora\PYZus{}AMPM}\PY{l+s+s2}{\PYZdq{}}\PY{p}{]}\PY{p}{[}\PY{l+s+s2}{\PYZdq{}}\PY{l+s+s2}{Fecha\PYZus{}horas}\PY{l+s+s2}{\PYZdq{}}\PY{p}{]} \PY{o}{=} \PY{p}{(}
    \PY{n}{grupos\PYZus{}formatoHora\PYZus{}12h\PYZus{}24H}\PY{p}{[}\PY{l+s+s2}{\PYZdq{}}\PY{l+s+s2}{Fhora\PYZus{}AMPM}\PY{l+s+s2}{\PYZdq{}}\PY{p}{]}\PY{p}{[}\PY{l+s+s2}{\PYZdq{}}\PY{l+s+s2}{Fecha\PYZus{}24h}\PY{l+s+s2}{\PYZdq{}}\PY{p}{]} \PY{o}{\PYZhy{}} \PY{n}{fecha\PYZus{}referencia}
\PY{p}{)}\PY{o}{.}\PY{n}{dt}\PY{o}{.}\PY{n}{total\PYZus{}seconds}\PY{p}{(}\PY{p}{)} \PY{o}{/} \PY{l+m+mi}{3600}
\PY{n}{grupos\PYZus{}formatoHora\PYZus{}12h\PYZus{}24H}\PY{p}{[}\PY{l+s+s2}{\PYZdq{}}\PY{l+s+s2}{Fhora\PYZus{}AMPM}\PY{l+s+s2}{\PYZdq{}}\PY{p}{]}\PY{p}{[}\PY{l+s+s2}{\PYZdq{}}\PY{l+s+s2}{Fecha\PYZus{}24h}\PY{l+s+s2}{\PYZdq{}}\PY{p}{]} \PY{o}{=} \PY{n}{grupos\PYZus{}formatoHora\PYZus{}12h\PYZus{}24H}\PY{p}{[}
    \PY{l+s+s2}{\PYZdq{}}\PY{l+s+s2}{Fhora\PYZus{}AMPM}\PY{l+s+s2}{\PYZdq{}}
\PY{p}{]}\PY{p}{[}\PY{l+s+s2}{\PYZdq{}}\PY{l+s+s2}{Fecha\PYZus{}24h}\PY{l+s+s2}{\PYZdq{}}\PY{p}{]}\PY{o}{.}\PY{n}{dt}\PY{o}{.}\PY{n}{strftime}\PY{p}{(}\PY{l+s+s2}{\PYZdq{}}\PY{l+s+s2}{\PYZpc{}}\PY{l+s+s2}{Y/}\PY{l+s+s2}{\PYZpc{}}\PY{l+s+s2}{m/}\PY{l+s+si}{\PYZpc{}d}\PY{l+s+s2}{ }\PY{l+s+s2}{\PYZpc{}}\PY{l+s+s2}{H:}\PY{l+s+s2}{\PYZpc{}}\PY{l+s+s2}{M:}\PY{l+s+s2}{\PYZpc{}}\PY{l+s+s2}{S}\PY{l+s+s2}{\PYZdq{}}\PY{p}{)}
\end{Verbatim}
\end{tcolorbox}

    \hypertarget{guardando-datos-estandarizados}{%
\subsection{Guardando datos
estandarizados}\label{guardando-datos-estandarizados}}

    Debido a la limitación de memoria disponible, con solo 8 GB de RAM y 20
GB de swap, se hizo necesario almacenar previamente los conjuntos de
datos en los archivos `Fhora\_AMPM.csv' y `Fhora\_24HH.csv' antes de
proceder con su carga y división en subconjuntos de datos de interés.
Esta estrategia fue crucial para evitar problemas de saturación de
memoria durante el procesamiento de los grandes volúmenes de datos.
Además, se añadió al conjunto de datos, una columna adicional con el
formato del tiempo en horas para facilitar el cálculo de posibles
correlaciones temporales, especialmente en eventos outliers dentro de
las series climáticas. Los outliers en datos de clima son de particular
interés porque representan eventos extremos que pueden indicar cambios
significativos en el clima, como fenómenos meteorológicos severos o
patrones inusuales. Analizar estos eventos es esencial para comprender
mejor las fluctuaciones climáticas y para prever posibles impactos
futuros, ya que estos extremos pueden tener implicaciones importantes en
la planificación y la adaptación a cambios climáticos.

    \begin{tcolorbox}[breakable, size=fbox, boxrule=1pt, pad at break*=1mm,colback=cellbackground, colframe=cellborder]
\prompt{In}{incolor}{ }{\boxspacing}
\begin{Verbatim}[commandchars=\\\{\}]
\PY{n}{grupos\PYZus{}formatoHora\PYZus{}12h\PYZus{}24H}\PY{p}{[}\PY{l+s+s2}{\PYZdq{}}\PY{l+s+s2}{Fhora\PYZus{}AMPM}\PY{l+s+s2}{\PYZdq{}}\PY{p}{]}\PY{o}{.}\PY{n}{to\PYZus{}csv}\PY{p}{(}\PY{l+s+s2}{\PYZdq{}}\PY{l+s+s2}{Fhora\PYZus{}AMPM.csv}\PY{l+s+s2}{\PYZdq{}}\PY{p}{,} \PY{n}{index}\PY{o}{=}\PY{k+kc}{False}\PY{p}{)}

\PY{n}{grupos\PYZus{}formatoHora\PYZus{}12h\PYZus{}24H}\PY{p}{[}\PY{l+s+s2}{\PYZdq{}}\PY{l+s+s2}{Fhora\PYZus{}24HH}\PY{l+s+s2}{\PYZdq{}}\PY{p}{]}\PY{p}{[}\PY{l+s+s2}{\PYZdq{}}\PY{l+s+s2}{Fecha\PYZus{}24h}\PY{l+s+s2}{\PYZdq{}}\PY{p}{]} \PY{o}{=} \PY{n}{pd}\PY{o}{.}\PY{n}{to\PYZus{}datetime}\PY{p}{(}
    \PY{n}{grupos\PYZus{}formatoHora\PYZus{}12h\PYZus{}24H}\PY{p}{[}\PY{l+s+s2}{\PYZdq{}}\PY{l+s+s2}{Fhora\PYZus{}24HH}\PY{l+s+s2}{\PYZdq{}}\PY{p}{]}\PY{p}{[}\PY{l+s+s2}{\PYZdq{}}\PY{l+s+s2}{Fecha}\PY{l+s+s2}{\PYZdq{}}\PY{p}{]}\PY{p}{,}
    \PY{n+nb}{format}\PY{o}{=}\PY{l+s+s2}{\PYZdq{}}\PY{l+s+si}{\PYZpc{}d}\PY{l+s+s2}{/}\PY{l+s+s2}{\PYZpc{}}\PY{l+s+s2}{m/}\PY{l+s+s2}{\PYZpc{}}\PY{l+s+s2}{Y }\PY{l+s+s2}{\PYZpc{}}\PY{l+s+s2}{H:}\PY{l+s+s2}{\PYZpc{}}\PY{l+s+s2}{M}\PY{l+s+s2}{\PYZdq{}}\PY{p}{,}
    \PY{n}{errors}\PY{o}{=}\PY{l+s+s2}{\PYZdq{}}\PY{l+s+s2}{coerce}\PY{l+s+s2}{\PYZdq{}}\PY{p}{,}
\PY{p}{)}

\PY{n}{grupos\PYZus{}formatoHora\PYZus{}12h\PYZus{}24H}\PY{p}{[}\PY{l+s+s2}{\PYZdq{}}\PY{l+s+s2}{Fhora\PYZus{}24HH}\PY{l+s+s2}{\PYZdq{}}\PY{p}{]}\PY{p}{[}\PY{l+s+s2}{\PYZdq{}}\PY{l+s+s2}{Fecha\PYZus{}horas}\PY{l+s+s2}{\PYZdq{}}\PY{p}{]} \PY{o}{=} \PY{p}{(}
    \PY{n}{grupos\PYZus{}formatoHora\PYZus{}12h\PYZus{}24H}\PY{p}{[}\PY{l+s+s2}{\PYZdq{}}\PY{l+s+s2}{Fhora\PYZus{}24HH}\PY{l+s+s2}{\PYZdq{}}\PY{p}{]}\PY{p}{[}\PY{l+s+s2}{\PYZdq{}}\PY{l+s+s2}{Fecha\PYZus{}24h}\PY{l+s+s2}{\PYZdq{}}\PY{p}{]} \PY{o}{\PYZhy{}} \PY{n}{fecha\PYZus{}referencia}
\PY{p}{)}\PY{o}{.}\PY{n}{dt}\PY{o}{.}\PY{n}{total\PYZus{}seconds}\PY{p}{(}\PY{p}{)} \PY{o}{/} \PY{l+m+mi}{3600}

\PY{n}{grupos\PYZus{}formatoHora\PYZus{}12h\PYZus{}24H}\PY{p}{[}\PY{l+s+s2}{\PYZdq{}}\PY{l+s+s2}{Fhora\PYZus{}24HH}\PY{l+s+s2}{\PYZdq{}}\PY{p}{]}\PY{p}{[}\PY{l+s+s2}{\PYZdq{}}\PY{l+s+s2}{Fecha\PYZus{}24h}\PY{l+s+s2}{\PYZdq{}}\PY{p}{]} \PY{o}{=} \PY{n}{grupos\PYZus{}formatoHora\PYZus{}12h\PYZus{}24H}\PY{p}{[}
    \PY{l+s+s2}{\PYZdq{}}\PY{l+s+s2}{Fhora\PYZus{}24HH}\PY{l+s+s2}{\PYZdq{}}
\PY{p}{]}\PY{p}{[}\PY{l+s+s2}{\PYZdq{}}\PY{l+s+s2}{Fecha\PYZus{}24h}\PY{l+s+s2}{\PYZdq{}}\PY{p}{]}\PY{o}{.}\PY{n}{dt}\PY{o}{.}\PY{n}{strftime}\PY{p}{(}\PY{l+s+s2}{\PYZdq{}}\PY{l+s+s2}{\PYZpc{}}\PY{l+s+s2}{Y/}\PY{l+s+s2}{\PYZpc{}}\PY{l+s+s2}{m/}\PY{l+s+si}{\PYZpc{}d}\PY{l+s+s2}{ }\PY{l+s+s2}{\PYZpc{}}\PY{l+s+s2}{H:}\PY{l+s+s2}{\PYZpc{}}\PY{l+s+s2}{M:}\PY{l+s+s2}{\PYZpc{}}\PY{l+s+s2}{S}\PY{l+s+s2}{\PYZdq{}}\PY{p}{)}

\PY{n}{grupos\PYZus{}formatoHora\PYZus{}12h\PYZus{}24H}\PY{p}{[}\PY{l+s+s2}{\PYZdq{}}\PY{l+s+s2}{Fhora\PYZus{}24HH}\PY{l+s+s2}{\PYZdq{}}\PY{p}{]}\PY{o}{.}\PY{n}{to\PYZus{}csv}\PY{p}{(}\PY{l+s+s2}{\PYZdq{}}\PY{l+s+s2}{Fhora\PYZus{}24HH.csv}\PY{l+s+s2}{\PYZdq{}}\PY{p}{,} \PY{n}{index}\PY{o}{=}\PY{k+kc}{False}\PY{p}{)}
\end{Verbatim}
\end{tcolorbox}

    \hypertarget{segmentaciuxf3n-y-guardado-de-datos-estuxe1ndar-por-municipios}{%
\subsection{Segmentación y guardado de datos estándar por
municipios}\label{segmentaciuxf3n-y-guardado-de-datos-estuxe1ndar-por-municipios}}

    Una vez que los datos ha sigo guardados en un formato correcto, se
cargan los archivos `Fhora\_AMPM.csv', `Fhora\_24HH.csv' y cuyos datos
se agrupan por ``Variable''. Esto permite separar el conjunto de datos
en subconjuntos específicos según cada variable de estudio. A
continuación, dentro de cada grupo correspondiente a una variable, se
realiza una segunda agrupación por ``Municipio''. Este proceso permite
extraer y organizar los datos de manera más detallada, dividiendo la
información según los municipios individuales.

Posteriormente, los datos correspondientes a cada combinación de
variable y municipio se almacenan en archivos separados. Estos archivos
se guardan en un directorio local, donde cada variable tiene su propio
conjunto de archivos, uno para cada municipio. Esta estructura facilita
la gestión de la información, permitiendo un acceso más eficiente y un
análisis especializado. Al organizar los datos de esta manera, no solo
se mejora la capacidad de manejo del conjunto de datos, sino que también
se optimiza la velocidad y la eficiencia de los análisis posteriores,
permitiendo enfoques más detallados y específicos para cada variable y
cada región. Es importante destacar que en este trabajo, se han
eliminado con el método
\texttt{dropna(axis=0,\ subset={[}"Concentración"{]})}, las filas que
contienen valores \texttt{NaN} en la columna que almacena los datos de
la \texttt{Concentración} atmosférica.

El siguiente código de 46 lineas ilustra cómo se lleva a cabo todo este
proceso.

    \begin{tcolorbox}[breakable, size=fbox, boxrule=1pt, pad at break*=1mm,colback=cellbackground, colframe=cellborder]
\prompt{In}{incolor}{ }{\boxspacing}
\begin{Verbatim}[commandchars=\\\{\}]
\PY{n}{grupos\PYZus{}acumulados} \PY{o}{=} \PY{p}{\PYZob{}}\PY{p}{\PYZcb{}}
\PY{k}{for} \PY{n}{datsclima} \PY{o+ow}{in} \PY{p}{[}\PY{l+s+s2}{\PYZdq{}}\PY{l+s+s2}{Fhora\PYZus{}AMPM}\PY{l+s+s2}{\PYZdq{}}\PY{p}{,} \PY{l+s+s2}{\PYZdq{}}\PY{l+s+s2}{Fhora\PYZus{}24HH}\PY{l+s+s2}{\PYZdq{}}\PY{p}{]}\PY{p}{:}
    \PY{n}{conjunto\PYZus{}de\PYZus{}datos} \PY{o}{=} \PY{n}{pd}\PY{o}{.}\PY{n}{read\PYZus{}csv}\PY{p}{(}
        \PY{n}{datsclima} \PY{o}{+} \PY{l+s+s2}{\PYZdq{}}\PY{l+s+s2}{.csv}\PY{l+s+s2}{\PYZdq{}}\PY{p}{,}
        \PY{n}{dtype}\PY{o}{=}\PY{n+nb}{str}\PY{p}{,}
        \PY{n}{chunksize}\PY{o}{=}\PY{l+m+mi}{5000000}\PY{p}{,}
    \PY{p}{)}

    \PY{k}{for} \PY{n}{chunk} \PY{o+ow}{in} \PY{n}{conjunto\PYZus{}de\PYZus{}datos}\PY{p}{:}
        \PY{n}{grouped} \PY{o}{=} \PY{n}{chunk}\PY{o}{.}\PY{n}{groupby}\PY{p}{(}\PY{l+s+s2}{\PYZdq{}}\PY{l+s+s2}{Variable}\PY{l+s+s2}{\PYZdq{}}\PY{p}{)}
        \PY{k}{for} \PY{n}{nombre\PYZus{}grupo}\PY{p}{,} \PY{n}{grupo} \PY{o+ow}{in} \PY{n}{grouped}\PY{p}{:}
            \PY{k}{if} \PY{n}{nombre\PYZus{}grupo} \PY{o+ow}{in} \PY{n}{grupos\PYZus{}acumulados}\PY{p}{:}
                \PY{n}{grupos\PYZus{}acumulados}\PY{p}{[}\PY{n}{nombre\PYZus{}grupo}\PY{p}{]} \PY{o}{=} \PY{n}{pd}\PY{o}{.}\PY{n}{concat}\PY{p}{(}
                    \PY{p}{[}\PY{n}{grupos\PYZus{}acumulados}\PY{p}{[}\PY{n}{nombre\PYZus{}grupo}\PY{p}{]}\PY{p}{,} \PY{n}{grupo}\PY{p}{]}
                \PY{p}{)}
            \PY{k}{else}\PY{p}{:}
                \PY{n}{grupos\PYZus{}acumulados}\PY{p}{[}\PY{n}{nombre\PYZus{}grupo}\PY{p}{]} \PY{o}{=} \PY{n}{grupo}

\PY{k}{for} \PY{n}{us}\PY{p}{,} \PY{n}{ot} \PY{o+ow}{in} \PY{n+nb}{enumerate}\PY{p}{(}\PY{n+nb}{list}\PY{p}{(}\PY{n}{variables\PYZus{}contaminacion}\PY{o}{.}\PY{n}{keys}\PY{p}{(}\PY{p}{)}\PY{p}{)}\PY{p}{)}\PY{p}{:}
    \PY{n}{vrclima} \PY{o}{=} \PY{n}{variables\PYZus{}contaminacion}\PY{p}{[}\PY{n}{ot}\PY{p}{]}\PY{p}{[}\PY{l+s+s2}{\PYZdq{}}\PY{l+s+s2}{variable}\PY{l+s+s2}{\PYZdq{}}\PY{p}{]}
    \PY{n}{df\PYZus{}colconcet\PYZus{}noNulos} \PY{o}{=} \PY{p}{(}
        \PY{n}{grupos\PYZus{}acumulados}\PY{p}{[}\PY{n}{vrclima}\PY{p}{]}\PY{o}{.}\PY{n}{dropna}\PY{p}{(}\PY{n}{axis}\PY{o}{=}\PY{l+m+mi}{0}\PY{p}{,} \PY{n}{subset}\PY{o}{=}\PY{p}{[}\PY{l+s+s2}{\PYZdq{}}\PY{l+s+s2}{Concentración}\PY{l+s+s2}{\PYZdq{}}\PY{p}{]}\PY{p}{)}\PY{o}{.}\PY{n}{copy}\PY{p}{(}\PY{p}{)}
    \PY{p}{)}
    \PY{n}{df\PYZus{}noNulos\PYZus{}ordenados} \PY{o}{=} \PY{n}{df\PYZus{}colconcet\PYZus{}noNulos}\PY{o}{.}\PY{n}{sort\PYZus{}values}\PY{p}{(}\PY{n}{by}\PY{o}{=}\PY{l+s+s2}{\PYZdq{}}\PY{l+s+s2}{Fecha\PYZus{}horas}\PY{l+s+s2}{\PYZdq{}}\PY{p}{)}\PY{o}{.}\PY{n}{copy}\PY{p}{(}\PY{p}{)}
    \PY{n}{df\PYZus{}noNulos\PYZus{}ordenados}\PY{p}{[}\PY{l+s+s2}{\PYZdq{}}\PY{l+s+s2}{Concentración}\PY{l+s+s2}{\PYZdq{}}\PY{p}{]} \PY{o}{=} \PY{n}{df\PYZus{}noNulos\PYZus{}ordenados}\PY{p}{[}
        \PY{l+s+s2}{\PYZdq{}}\PY{l+s+s2}{Concentración}\PY{l+s+s2}{\PYZdq{}}
    \PY{p}{]}\PY{o}{.}\PY{n}{str}\PY{o}{.}\PY{n}{replace}\PY{p}{(}\PY{l+s+s2}{\PYZdq{}}\PY{l+s+s2}{,}\PY{l+s+s2}{\PYZdq{}}\PY{p}{,} \PY{l+s+s2}{\PYZdq{}}\PY{l+s+s2}{\PYZdq{}}\PY{p}{)}

    \PY{n}{df\PYZus{}noNulos\PYZus{}ordenados}\PY{p}{[}\PY{l+s+s2}{\PYZdq{}}\PY{l+s+s2}{Concentración}\PY{l+s+s2}{\PYZdq{}}\PY{p}{]} \PY{o}{=} \PY{n}{df\PYZus{}noNulos\PYZus{}ordenados}\PY{p}{[}
        \PY{l+s+s2}{\PYZdq{}}\PY{l+s+s2}{Concentración}\PY{l+s+s2}{\PYZdq{}}
    \PY{p}{]}\PY{o}{.}\PY{n}{astype}\PY{p}{(}\PY{n+nb}{float}\PY{p}{)}

    \PY{n}{grouped\PYZus{}id\PYZus{}municipio} \PY{o}{=} \PY{n}{df\PYZus{}noNulos\PYZus{}ordenados}\PY{o}{.}\PY{n}{groupby}\PY{p}{(}\PY{l+s+s2}{\PYZdq{}}\PY{l+s+s2}{Código del municipio}\PY{l+s+s2}{\PYZdq{}}\PY{p}{)}
    \PY{n}{grupos\PYZus{}acumulados\PYZus{}municipio} \PY{o}{=} \PY{p}{\PYZob{}}\PY{p}{\PYZcb{}}
    \PY{k}{for} \PY{n}{nombre\PYZus{}grupo}\PY{p}{,} \PY{n}{grupo} \PY{o+ow}{in} \PY{n}{grouped\PYZus{}id\PYZus{}municipio}\PY{p}{:}
        \PY{k}{if} \PY{n}{nombre\PYZus{}grupo} \PY{o+ow}{in} \PY{n}{grupos\PYZus{}acumulados\PYZus{}municipio}\PY{p}{:}
            \PY{n}{grupos\PYZus{}acumulados\PYZus{}municipio}\PY{p}{[}\PY{n}{nombre\PYZus{}grupo}\PY{p}{]} \PY{o}{=} \PY{n}{pd}\PY{o}{.}\PY{n}{concat}\PY{p}{(}
                \PY{p}{[}\PY{n}{grupos\PYZus{}acumulados\PYZus{}municipio}\PY{p}{[}\PY{n}{nombre\PYZus{}grupo}\PY{p}{]}\PY{p}{,} \PY{n}{grupo}\PY{p}{]}
            \PY{p}{)}
        \PY{k}{else}\PY{p}{:}
            \PY{n}{grupos\PYZus{}acumulados\PYZus{}municipio}\PY{p}{[}\PY{n}{nombre\PYZus{}grupo}\PY{p}{]} \PY{o}{=} \PY{n}{grupo}

    \PY{k}{for} \PY{n}{nombre\PYZus{}grupo}\PY{p}{,} \PY{n}{grupo} \PY{o+ow}{in} \PY{n}{grupos\PYZus{}acumulados\PYZus{}municipio}\PY{o}{.}\PY{n}{items}\PY{p}{(}\PY{p}{)}\PY{p}{:}
        \PY{n}{nombre\PYZus{}archivo} \PY{o}{=} \PY{l+s+sa}{f}\PY{l+s+s2}{\PYZdq{}}\PY{l+s+s2}{datos\PYZus{}segmentados/}\PY{l+s+si}{\PYZob{}}\PY{n}{ot}\PY{l+s+si}{\PYZcb{}}\PY{l+s+s2}{/}\PY{l+s+si}{\PYZob{}}\PY{n}{nombre\PYZus{}grupo}\PY{l+s+si}{\PYZcb{}}\PY{l+s+s2}{.csv}\PY{l+s+s2}{\PYZdq{}}
        \PY{n}{grupo}\PY{o}{.}\PY{n}{to\PYZus{}csv}\PY{p}{(}\PY{n}{nombre\PYZus{}archivo}\PY{p}{,} \PY{n}{index}\PY{o}{=}\PY{k+kc}{False}\PY{p}{)}
        \PY{n+nb}{print}\PY{p}{(}\PY{l+s+sa}{f}\PY{l+s+s2}{\PYZdq{}}\PY{l+s+s2}{Guardado: }\PY{l+s+si}{\PYZob{}}\PY{n}{nombre\PYZus{}archivo}\PY{l+s+si}{\PYZcb{}}\PY{l+s+s2}{\PYZdq{}}\PY{p}{)}
\end{Verbatim}
\end{tcolorbox}

    Es importante destacar que el proceso de segmentación de los datos por
municipio en este trabajo podría realizarse de manera mucho más
eficiente si se contara con una computadora de mayor capacidad de
memoria, superior a 32 GB. Con una memoria RAM más amplia, se podrían
manejar y procesar mayores volúmenes de datos simultáneamente,
reduciendo significativamente la necesidad de almacenamiento temporal y
evitando la saturación de la memoria. Esto permitiría una ejecución más
rápida de los procesos y una gestión más fluida de los datos,
optimizando el análisis y la segmentación sin las limitaciones impuestas
por la capacidad actual de memoria.

La Figura 1 ilustra el conjunto de datos generado para cada variable
climática. En cada directorio, se encuentran los datos correspondientes
a los municipios en los que se realizó la medición de esa variable
climática.

    \begin{tcolorbox}[breakable, size=fbox, boxrule=1pt, pad at break*=1mm,colback=cellbackground, colframe=cellborder]
\prompt{In}{incolor}{5}{\boxspacing}
\begin{Verbatim}[commandchars=\\\{\}]
\PY{n}{imshow\PYZus{}plots}\PY{p}{(}\PY{l+s+s2}{\PYZdq{}}\PY{l+s+s2}{img.png}\PY{l+s+s2}{\PYZdq{}}\PY{p}{)}
\end{Verbatim}
\end{tcolorbox}

    \begin{center}
    \adjustimage{max size={0.9\linewidth}{0.9\paperheight}}{datosaire_files/datosaire_26_0.png}
    \end{center}
    { \hspace*{\fill} \\}
    
    \hypertarget{analizando-dinuxe1micas-lineales-en-datos-climuxe1ticos}{%
\section{Analizando dinámicas lineales en datos
climáticos}\label{analizando-dinuxe1micas-lineales-en-datos-climuxe1ticos}}

    El estudio de las dinámicas lineales a partir de un conjunto de datos
climáticos implica un enfoque metodológico que busca identificar
relaciones proporcionales y predecibles entre las variables. El primer
paso es la recopilación y organización de los datos, asegurando que sean
consistentes y de buena calidad. Esto puede incluir mediciones de
temperatura, precipitación, presión atmosférica o concentraciones de
gases de efecto invernadero, entre otros. A continuación, es esencial
realizar una limpieza de los datos, eliminando valores atípicos o
errores que puedan sesgar los resultados.

Una vez que los datos están listos, se procede a aplicar técnicas
estadísticas que permitan identificar posibles relaciones lineales entre
las variables. Uno de los métodos más utilizados es la regresión lineal,
que busca ajustar una línea recta que explique cómo una variable
dependiente (como la temperatura) cambia en función de otra variable
independiente (como la concentración de dióxido de carbono). Esta
relación lineal es útil para predecir tendencias futuras y entender cómo
las variables se afectan mutuamente en condiciones controladas.

Después de la regresión lineal, se evalúa la significancia de los
resultados. Esto se puede hacer calculando coeficientes de correlación,
que miden la fuerza y dirección de la relación lineal entre dos
variables. Además, se emplean pruebas estadísticas como el p-valor para
determinar si las correlaciones observadas son estadísticamente
significativas. Si los resultados indican una relación lineal fuerte y
significativa, es posible modelar el comportamiento futuro del sistema
climático bajo el supuesto de que estas relaciones se mantendrán
constantes en el tiempo.

    \hypertarget{visualizaciuxf3n-de-series-temporales}{%
\subsection{Visualización de series
temporales}\label{visualizaciuxf3n-de-series-temporales}}

    Para iniciar el análisis, primero debemos cargar los datos correctamente
en el entorno de trabajo. Esto implica importar los datos desde su
ubicación actual, donde fueron almacenados en archivos \texttt{.CSV},
siguiendo el procedimiento detallado en las secciones anteriores. Esta
etapa es crucial para garantizar que la información esté disponible en
un formato adecuado para el análisis posterior. Para comenzar con el
análisis, iniciaré con una visualización de las series temporales por
varible. Las gráficas se crear con las 59 lineas de cogido mostradas a
continuación.

    \begin{tcolorbox}[breakable, size=fbox, boxrule=1pt, pad at break*=1mm,colback=cellbackground, colframe=cellborder]
\prompt{In}{incolor}{ }{\boxspacing}
\begin{Verbatim}[commandchars=\\\{\}]
\PY{n}{f24h} \PY{o}{=} \PY{l+s+s2}{\PYZdq{}}\PY{l+s+s2}{Fecha\PYZus{}24h}\PY{l+s+s2}{\PYZdq{}}
\PY{n}{dens} \PY{o}{=} \PY{l+s+s2}{\PYZdq{}}\PY{l+s+s2}{Concentración}\PY{l+s+s2}{\PYZdq{}}
\PY{n}{ui} \PY{o}{=} \PY{n}{np}\PY{o}{.}\PY{n}{array}\PY{p}{(}\PY{p}{[}\PY{p}{[}\PY{n}{i}\PY{p}{,} \PY{n}{j}\PY{p}{]} \PY{k}{for} \PY{n}{i} \PY{o+ow}{in} \PY{n+nb}{range}\PY{p}{(}\PY{l+m+mi}{1}\PY{p}{)} \PY{k}{for} \PY{n}{j} \PY{o+ow}{in} \PY{n+nb}{range}\PY{p}{(}\PY{l+m+mi}{2}\PY{p}{)}\PY{p}{]}\PY{p}{)}

\PY{k}{def} \PY{n+nf}{df\PYZus{}umbr}\PY{p}{(}\PY{n}{dataframe}\PY{p}{)}\PY{p}{:}
    \PY{n}{dataframe}\PY{o}{.}\PY{n}{set\PYZus{}index}\PY{p}{(}\PY{n}{f24h}\PY{p}{,} \PY{n}{inplace}\PY{o}{=}\PY{k+kc}{True}\PY{p}{)}
    \PY{n}{dataframe}\PY{p}{[}\PY{l+s+s2}{\PYZdq{}}\PY{l+s+s2}{Periodo}\PY{l+s+s2}{\PYZdq{}}\PY{p}{]} \PY{o}{=} \PY{n}{dataframe}\PY{o}{.}\PY{n}{index}\PY{o}{.}\PY{n}{to\PYZus{}period}\PY{p}{(}\PY{l+s+s2}{\PYZdq{}}\PY{l+s+s2}{D}\PY{l+s+s2}{\PYZdq{}}\PY{p}{)}\PY{o}{.}\PY{n}{copy}\PY{p}{(}\PY{p}{)}
    \PY{n}{df\PYZus{}ubr} \PY{o}{=} \PY{n}{dataframe}\PY{o}{.}\PY{n}{groupby}\PY{p}{(}\PY{l+s+s2}{\PYZdq{}}\PY{l+s+s2}{Periodo}\PY{l+s+s2}{\PYZdq{}}\PY{p}{)}\PY{p}{[}\PY{n}{dens}\PY{p}{]}\PY{o}{.}\PY{n}{describe}\PY{p}{(}\PY{p}{)}
    \PY{k}{return} \PY{n}{df\PYZus{}ubr}


\PY{k}{for} \PY{n}{ur} \PY{o+ow}{in} \PY{n+nb}{range}\PY{p}{(}\PY{l+m+mi}{0}\PY{p}{,} \PY{n+nb}{len}\PY{p}{(}\PY{n+nb}{list}\PY{p}{(}\PY{n}{variables\PYZus{}contaminacion}\PY{o}{.}\PY{n}{keys}\PY{p}{(}\PY{p}{)}\PY{p}{)}\PY{p}{)}\PY{p}{,} \PY{l+m+mi}{2}\PY{p}{)}\PY{p}{:}
    \PY{n}{fig} \PY{o}{=} \PY{n}{plt}\PY{o}{.}\PY{n}{figure}\PY{p}{(}\PY{n}{figsize}\PY{o}{=}\PY{p}{(}\PY{l+m+mi}{20}\PY{p}{,} \PY{l+m+mi}{5}\PY{p}{)}\PY{p}{,} \PY{n}{tight\PYZus{}layout}\PY{o}{=}\PY{k+kc}{True}\PY{p}{)}
    \PY{n}{gs} \PY{o}{=} \PY{n}{gridspec}\PY{o}{.}\PY{n}{GridSpec}\PY{p}{(}\PY{l+m+mi}{1}\PY{p}{,} \PY{l+m+mi}{2}\PY{p}{)}
    \PY{k}{for} \PY{n}{us} \PY{o+ow}{in} \PY{n+nb}{range}\PY{p}{(}\PY{n}{ur}\PY{p}{,} \PY{n}{ur} \PY{o}{+} \PY{l+m+mi}{2}\PY{p}{)}\PY{p}{:}
        \PY{n}{uz} \PY{o}{=} \PY{n}{us} \PY{o}{\PYZpc{}} \PY{l+m+mi}{2}
        \PY{n}{ot} \PY{o}{=} \PY{n+nb}{list}\PY{p}{(}\PY{n}{variables\PYZus{}contaminacion}\PY{o}{.}\PY{n}{keys}\PY{p}{(}\PY{p}{)}\PY{p}{)}\PY{p}{[}\PY{n}{us}\PY{p}{]}
        \PY{n}{ax0} \PY{o}{=} \PY{n}{fig}\PY{o}{.}\PY{n}{add\PYZus{}subplot}\PY{p}{(}\PY{n}{gs}\PY{p}{[}\PY{n}{ui}\PY{p}{[}\PY{p}{:}\PY{p}{,} \PY{l+m+mi}{0}\PY{p}{]}\PY{p}{[}\PY{n}{uz}\PY{p}{]}\PY{p}{,} \PY{n}{ui}\PY{p}{[}\PY{p}{:}\PY{p}{,} \PY{l+m+mi}{1}\PY{p}{]}\PY{p}{[}\PY{n}{uz}\PY{p}{]}\PY{p}{]}\PY{p}{)}
        \PY{n}{data\PYZus{}frame\PYZus{}subconjunto} \PY{o}{=} \PY{n}{pd}\PY{o}{.}\PY{n}{read\PYZus{}csv}\PY{p}{(}
            \PY{l+s+s2}{\PYZdq{}}\PY{l+s+s2}{datos\PYZus{}segmentados/}\PY{l+s+s2}{\PYZdq{}} \PY{o}{+} \PY{n}{ot} \PY{o}{+} \PY{l+s+s2}{\PYZdq{}}\PY{l+s+s2}{.csv}\PY{l+s+s2}{\PYZdq{}}\PY{p}{,} \PY{n}{dtype}\PY{o}{=}\PY{n+nb}{str}
        \PY{p}{)}

        \PY{n}{data\PYZus{}frame\PYZus{}subconjunto}\PY{p}{[}\PY{n}{dens}\PY{p}{]} \PY{o}{=} \PY{n}{data\PYZus{}frame\PYZus{}subconjunto}\PY{p}{[}\PY{n}{dens}\PY{p}{]}\PY{o}{.}\PY{n}{astype}\PY{p}{(}\PY{n+nb}{float}\PY{p}{)}
        \PY{n}{data\PYZus{}frame\PYZus{}subconjunto}\PY{p}{[}\PY{n}{f24h}\PY{p}{]} \PY{o}{=} \PY{n}{pd}\PY{o}{.}\PY{n}{to\PYZus{}datetime}\PY{p}{(}
            \PY{n}{data\PYZus{}frame\PYZus{}subconjunto}\PY{p}{[}\PY{n}{f24h}\PY{p}{]}\PY{p}{,} \PY{n+nb}{format}\PY{o}{=}\PY{l+s+s2}{\PYZdq{}}\PY{l+s+s2}{\PYZpc{}}\PY{l+s+s2}{Y/}\PY{l+s+s2}{\PYZpc{}}\PY{l+s+s2}{m/}\PY{l+s+si}{\PYZpc{}d}\PY{l+s+s2}{ }\PY{l+s+s2}{\PYZpc{}}\PY{l+s+s2}{H:}\PY{l+s+s2}{\PYZpc{}}\PY{l+s+s2}{M:}\PY{l+s+s2}{\PYZpc{}}\PY{l+s+s2}{S}\PY{l+s+s2}{\PYZdq{}}
        \PY{p}{)}
        \PY{n}{dtframe} \PY{o}{=} \PY{n}{data\PYZus{}frame\PYZus{}subconjunto}\PY{o}{.}\PY{n}{copy}\PY{p}{(}\PY{p}{)}
        \PY{n}{dev\PYZus{}std} \PY{o}{=} \PY{n}{dtframe}\PY{p}{[}\PY{n}{dens}\PY{p}{]}\PY{o}{.}\PY{n}{std}\PY{p}{(}\PY{p}{)}
        \PY{n}{media} \PY{o}{=} \PY{n}{dtframe}\PY{p}{[}\PY{n}{dens}\PY{p}{]}\PY{o}{.}\PY{n}{mean}\PY{p}{(}\PY{p}{)}

        \PY{n}{umb0}\PY{p}{,} \PY{n}{umb1} \PY{o}{=} \PY{n}{umbral\PYZus{}iqr}\PY{p}{(}\PY{n}{dtframe}\PY{p}{,} \PY{n}{dens}\PY{p}{)}

        \PY{n}{df\PYZus{}umb0} \PY{o}{=} \PY{n}{dtframe}\PY{p}{[}\PY{p}{(}\PY{n}{dtframe}\PY{p}{[}\PY{n}{dens}\PY{p}{]} \PY{o}{\PYZgt{}} \PY{n}{umb1}\PY{p}{)} \PY{o}{|} \PY{p}{(}\PY{n}{dtframe}\PY{p}{[}\PY{n}{dens}\PY{p}{]} \PY{o}{\PYZlt{}} \PY{n}{umb0}\PY{p}{)}\PY{p}{]}\PY{o}{.}\PY{n}{copy}\PY{p}{(}\PY{p}{)}
        \PY{n}{df\PYZus{}umbr}\PY{p}{(}\PY{n}{df\PYZus{}umb0}\PY{p}{)}\PY{p}{[}\PY{l+s+s2}{\PYZdq{}}\PY{l+s+s2}{mean}\PY{l+s+s2}{\PYZdq{}}\PY{p}{]}\PY{o}{.}\PY{n}{plot}\PY{p}{(}\PY{n}{style}\PY{o}{=}\PY{l+s+s2}{\PYZdq{}}\PY{l+s+s2}{\PYZhy{}o}\PY{l+s+s2}{\PYZdq{}}\PY{p}{,} \PY{n}{markersize}\PY{o}{=}\PY{l+m+mi}{3}\PY{p}{,} \PY{n}{color}\PY{o}{=}\PY{l+s+s2}{\PYZdq{}}\PY{l+s+s2}{red}\PY{l+s+s2}{\PYZdq{}}\PY{p}{)}

        \PY{n}{df\PYZus{}umb1} \PY{o}{=} \PY{n}{dtframe}\PY{p}{[}\PY{p}{(}\PY{n}{dtframe}\PY{p}{[}\PY{n}{dens}\PY{p}{]} \PY{o}{\PYZlt{}} \PY{n}{umb1}\PY{p}{)} \PY{o}{\PYZam{}} \PY{p}{(}\PY{n}{dtframe}\PY{p}{[}\PY{n}{dens}\PY{p}{]} \PY{o}{\PYZgt{}} \PY{n}{umb0}\PY{p}{)}\PY{p}{]}\PY{o}{.}\PY{n}{copy}\PY{p}{(}\PY{p}{)}
        \PY{n}{df\PYZus{}umbr}\PY{p}{(}\PY{n}{df\PYZus{}umb1}\PY{p}{)}\PY{p}{[}\PY{l+s+s2}{\PYZdq{}}\PY{l+s+s2}{mean}\PY{l+s+s2}{\PYZdq{}}\PY{p}{]}\PY{o}{.}\PY{n}{plot}\PY{p}{(}\PY{n}{style}\PY{o}{=}\PY{l+s+s2}{\PYZdq{}}\PY{l+s+s2}{\PYZhy{}o}\PY{l+s+s2}{\PYZdq{}}\PY{p}{,} \PY{n}{markersize}\PY{o}{=}\PY{l+m+mi}{3}\PY{p}{,} \PY{n}{color}\PY{o}{=}\PY{l+s+s2}{\PYZdq{}}\PY{l+s+s2}{blue}\PY{l+s+s2}{\PYZdq{}}\PY{p}{)}

        \PY{n}{unidad\PYZus{}variable} \PY{o}{=} \PY{n}{variables\PYZus{}contaminacion}\PY{p}{[}\PY{n}{ot}\PY{p}{]}\PY{p}{[}\PY{l+s+s2}{\PYZdq{}}\PY{l+s+s2}{unidad}\PY{l+s+s2}{\PYZdq{}}\PY{p}{]}
        \PY{n}{nombre\PYZus{}variable} \PY{o}{=} \PY{n}{variables\PYZus{}contaminacion}\PY{p}{[}\PY{n}{ot}\PY{p}{]}\PY{p}{[}\PY{l+s+s2}{\PYZdq{}}\PY{l+s+s2}{variable}\PY{l+s+s2}{\PYZdq{}}\PY{p}{]}
        \PY{n}{ax0}\PY{o}{.}\PY{n}{xaxis}\PY{o}{.}\PY{n}{set\PYZus{}tick\PYZus{}params}\PY{p}{(}\PY{n}{labelsize}\PY{o}{=}\PY{l+m+mi}{17}\PY{p}{)}
        \PY{n}{ax0}\PY{o}{.}\PY{n}{yaxis}\PY{o}{.}\PY{n}{set\PYZus{}tick\PYZus{}params}\PY{p}{(}\PY{n}{labelsize}\PY{o}{=}\PY{l+m+mi}{17}\PY{p}{)}
        \PY{n}{ax0}\PY{o}{.}\PY{n}{set\PYZus{}xlabel}\PY{p}{(}\PY{l+s+sa}{f}\PY{l+s+s2}{\PYZdq{}}\PY{l+s+s2}{Tiempo(días)}\PY{l+s+s2}{\PYZdq{}}\PY{p}{,} \PY{n}{fontsize}\PY{o}{=}\PY{l+m+mi}{20}\PY{p}{)}
        \PY{n}{ax0}\PY{o}{.}\PY{n}{set\PYZus{}ylabel}\PY{p}{(}\PY{l+s+sa}{f}\PY{l+s+s2}{\PYZdq{}}\PY{l+s+si}{\PYZob{}}\PY{n}{nombre\PYZus{}variable}\PY{l+s+si}{\PYZcb{}}\PY{l+s+s2}{ (}\PY{l+s+si}{\PYZob{}}\PY{n}{unidad\PYZus{}variable}\PY{l+s+si}{\PYZcb{}}\PY{l+s+s2}{)}\PY{l+s+s2}{\PYZdq{}}\PY{p}{,} \PY{n}{fontsize}\PY{o}{=}\PY{l+m+mi}{18}\PY{p}{)}
        \PY{n}{ax0}\PY{o}{.}\PY{n}{set\PYZus{}title}\PY{p}{(}\PY{l+s+sa}{f}\PY{l+s+s2}{\PYZdq{}}\PY{l+s+si}{\PYZob{}}\PY{n}{nombre\PYZus{}variable}\PY{l+s+si}{\PYZcb{}}\PY{l+s+s2}{ en función del tiempo}\PY{l+s+s2}{\PYZdq{}}\PY{p}{,} \PY{n}{fontsize}\PY{o}{=}\PY{l+m+mi}{20}\PY{p}{)}

    \PY{n}{fig}\PY{o}{.}\PY{n}{tight\PYZus{}layout}\PY{p}{(}\PY{p}{)}
    \PY{n}{plt}\PY{o}{.}\PY{n}{subplots\PYZus{}adjust}\PY{p}{(}\PY{n}{wspace}\PY{o}{=}\PY{l+m+mf}{0.15}\PY{p}{)}
    \PY{n}{ncont} \PY{o}{=} \PY{n+nb}{int}\PY{p}{(}\PY{n}{ur} \PY{o}{/} \PY{l+m+mi}{2}\PY{p}{)}
    \PY{n}{fig}\PY{o}{.}\PY{n}{savefig}\PY{p}{(}\PY{l+s+sa}{f}\PY{l+s+s2}{\PYZdq{}}\PY{l+s+s2}{plots/img\PYZus{}std}\PY{l+s+si}{\PYZob{}}\PY{n}{ncont}\PY{l+s+si}{\PYZcb{}}\PY{l+s+s2}{.png}\PY{l+s+s2}{\PYZdq{}}\PY{p}{)}
    \PY{k}{if} \PY{n}{ur}\PY{o}{==}\PY{l+m+mi}{10}\PY{p}{:}
        \PY{k}{break}
    \PY{c+c1}{\PYZsh{} plt.show()}
\end{Verbatim}
\end{tcolorbox}

    Las gráficas presentadas ilustran el comportamiento de las series
temporales de cuatro variables ambientales, medidas desde enero de 2011
hasta diciembre de 2018: \textbf{concentración de NO (Óxidos de
Nitrógeno)}, \textbf{humedad relativa}, \textbf{humedad relativa a 2
metros} y \textbf{humedad relativa a 10 metros}. La mayoría de las
series temporales fueron registradas con una frecuencia de muestreo de 1
hora, mientras que otras fueron registradas con frecuencias de 24 horas.
Cada punto en las gráficas representa el promedio diario de los datos
recopilados para cada variable, un proceso realizado para mejorar la
visualización de la variabilidad promedio diaria de la serie temporal y
facilitar el análisis de las tendencias a lo largo del tiempo. Estas
visualizaciones proporcionan una visión integral del comportamiento de
estas variables ambientales en función del tiempo.

    \begin{tcolorbox}[breakable, size=fbox, boxrule=1pt, pad at break*=1mm,colback=cellbackground, colframe=cellborder]
\prompt{In}{incolor}{11}{\boxspacing}
\begin{Verbatim}[commandchars=\\\{\}]
\PY{n}{imshow\PYZus{}plots}\PY{p}{(}\PY{l+s+s2}{\PYZdq{}}\PY{l+s+s2}{plots/img2.png}\PY{l+s+s2}{\PYZdq{}}\PY{p}{)}\PY{p}{;} \PY{n}{imshow\PYZus{}plots}\PY{p}{(}\PY{l+s+s2}{\PYZdq{}}\PY{l+s+s2}{plots/img1.png}\PY{l+s+s2}{\PYZdq{}}\PY{p}{)}
\end{Verbatim}
\end{tcolorbox}

    \begin{center}
    \adjustimage{max size={0.9\linewidth}{0.9\paperheight}}{datosaire_files/datosaire_33_0.png}
    \end{center}
    { \hspace*{\fill} \\}
    
    \begin{center}
    \adjustimage{max size={0.9\linewidth}{0.9\paperheight}}{datosaire_files/datosaire_33_1.png}
    \end{center}
    { \hspace*{\fill} \\}
    
    Analizar todos los datos de una variable sin segmentarlos por municipio
puede llevar a conclusiones equivocadas. Las mediciones de diferentes
localidades están influenciadas por condiciones locales específicas,
como la altitud, la proximidad al mar, el uso del suelo y el clima, que
afectan significativamente los datos y generan patrones y escalas
diversas entre los municipios. Al no considerar estas diferencias, se
corre el riesgo de mezclar información de contextos distintos, lo que
puede distorsionar los resultados y dificultar una interpretación mas
adecuada.

Aplicar un umbral único a los datos de las distintas variables sin
considerar las diversas escalas puede generar interpretaciones
incorrectas. Un umbral establecido sin tener en cuenta las diferencias
locales pueden eliminar datos valiosos que, aunque parezcan irrelevantes
en una escala general, son esenciales en su contexto específico. Esto
puede llevar a una sobreestimación en los análisis y a la pérdida de
información clave sobre las variaciones locales.

Por ejemplo, eventos climáticos que son extremos en una región pueden no
serlo en otra. Si se aplican los mismos criterios de umbral, se pueden
omitir detalles importantes que reflejan las diferencias locales,
limitando la comprensión global del fenómeno. A continuación mostramos
un gráfico en donde se observa como se dividen los datos al aplicar un
umbral intercuartílico (\(Q^{\pm} \pm 1.5IQR\), donde \$IQR = Q\^{}\{+\}
- Q\^{}\{-\} \$ y \(Q^{-}\), \(Q^{+}\), son los cuartiles del 25\% y
75\% respectivamente) puede ignorar drásticamente datos de la serie
temporal. En la figura, los datos de color rojo están por fuera de los
\(IRQ\), observe que para el caso del NO la cantidad de datos
considerados como outliers con este criterio es muy grande.

    \begin{tcolorbox}[breakable, size=fbox, boxrule=1pt, pad at break*=1mm,colback=cellbackground, colframe=cellborder]
\prompt{In}{incolor}{39}{\boxspacing}
\begin{Verbatim}[commandchars=\\\{\}]
\PY{n}{imshow\PYZus{}plots}\PY{p}{(}\PY{l+s+s2}{\PYZdq{}}\PY{l+s+s2}{plots/img\PYZus{}std2.png}\PY{l+s+s2}{\PYZdq{}}\PY{p}{)}
\end{Verbatim}
\end{tcolorbox}

    \begin{center}
    \adjustimage{max size={0.9\linewidth}{0.9\paperheight}}{datosaire_files/datosaire_35_0.png}
    \end{center}
    { \hspace*{\fill} \\}
    
    \hypertarget{revisiuxf3n-de-datos-locales}{%
\subsection{Revisión de datos
locales}\label{revisiuxf3n-de-datos-locales}}

    La revisión local de datos es un proceso crucial que consiste en
examinar los datos por municipio en lugar de considerar el conjunto
global. Esta metodología permite una inspección más detallada y
específica, teniendo en cuenta las particularidades y condiciones
locales que pueden influir en las mediciones.

En esta sección, se incluyen tanto análisis estadísticos como gráficos
para una comprensión integral. A través del análisis estadístico,
podemos calcular medidas de tendencia central, dispersión y otros
indicadores que revelan las características específicas de cada
municipio. Los gráficos, por su parte, facilitan la visualización de
patrones y tendencias a nivel local, permitiendo identificar variaciones
y comportamientos que podrían ser invisibles en un análisis global.

    \begin{tcolorbox}[breakable, size=fbox, boxrule=1pt, pad at break*=1mm,colback=cellbackground, colframe=cellborder]
\prompt{In}{incolor}{ }{\boxspacing}
\begin{Verbatim}[commandchars=\\\{\}]
\PY{n}{dtf\PYZus{}csv} \PY{o}{=} \PY{n}{pd}\PY{o}{.}\PY{n}{read\PYZus{}csv}\PY{p}{(}\PY{l+s+s2}{\PYZdq{}}\PY{l+s+s2}{datos\PYZus{}segmentados/temperatura\PYZus{}10m.csv}\PY{l+s+s2}{\PYZdq{}}\PY{p}{,} \PY{n}{dtype}\PY{o}{=}\PY{n+nb}{str}\PY{p}{)}
\PY{n}{dtf\PYZus{}csv}\PY{p}{[}\PY{l+s+s2}{\PYZdq{}}\PY{l+s+s2}{Concentración}\PY{l+s+s2}{\PYZdq{}}\PY{p}{]} \PY{o}{=} \PY{n}{dtf\PYZus{}csv}\PY{p}{[}\PY{l+s+s2}{\PYZdq{}}\PY{l+s+s2}{Concentración}\PY{l+s+s2}{\PYZdq{}}\PY{p}{]}\PY{o}{.}\PY{n}{astype}\PY{p}{(}\PY{n+nb}{float}\PY{p}{)}
\PY{n}{datfram} \PY{o}{=} \PY{n}{dtf\PYZus{}csv}\PY{p}{[}\PY{p}{[}\PY{l+s+s2}{\PYZdq{}}\PY{l+s+s2}{Código del municipio}\PY{l+s+s2}{\PYZdq{}}\PY{p}{,} \PY{l+s+s2}{\PYZdq{}}\PY{l+s+s2}{Nombre del municipio}\PY{l+s+s2}{\PYZdq{}}\PY{p}{,} \PY{l+s+s2}{\PYZdq{}}\PY{l+s+s2}{Concentración}\PY{l+s+s2}{\PYZdq{}}\PY{p}{]}\PY{p}{]}
\PY{n}{data\PYZus{}frame\PYZus{}groupby} \PY{o}{=} \PY{n}{datfram}\PY{o}{.}\PY{n}{groupby}\PY{p}{(}\PY{l+s+s2}{\PYZdq{}}\PY{l+s+s2}{Nombre del municipio}\PY{l+s+s2}{\PYZdq{}}\PY{p}{)}
\PY{n}{unidad\PYZus{}variable} \PY{o}{=} \PY{n}{variables\PYZus{}contaminacion}\PY{p}{[}\PY{l+s+s2}{\PYZdq{}}\PY{l+s+s2}{temperatura\PYZus{}10m}\PY{l+s+s2}{\PYZdq{}}\PY{p}{]}\PY{p}{[}\PY{l+s+s2}{\PYZdq{}}\PY{l+s+s2}{unidad}\PY{l+s+s2}{\PYZdq{}}\PY{p}{]}
\PY{n}{nombre\PYZus{}variable} \PY{o}{=} \PY{n}{variables\PYZus{}contaminacion}\PY{p}{[}\PY{l+s+s2}{\PYZdq{}}\PY{l+s+s2}{temperatura\PYZus{}10m}\PY{l+s+s2}{\PYZdq{}}\PY{p}{]}\PY{p}{[}\PY{l+s+s2}{\PYZdq{}}\PY{l+s+s2}{variable}\PY{l+s+s2}{\PYZdq{}}\PY{p}{]}
\PY{n}{abreviatura\PYZus{}variable} \PY{o}{=} \PY{n}{variables\PYZus{}contaminacion}\PY{p}{[}\PY{l+s+s2}{\PYZdq{}}\PY{l+s+s2}{temperatura\PYZus{}10m}\PY{l+s+s2}{\PYZdq{}}\PY{p}{]}\PY{p}{[}\PY{l+s+s2}{\PYZdq{}}\PY{l+s+s2}{abreviatura}\PY{l+s+s2}{\PYZdq{}}\PY{p}{]}
\PY{n}{df\PYZus{}describe} \PY{o}{=} \PY{n}{data\PYZus{}frame\PYZus{}groupby}\PY{o}{.}\PY{n}{describe}\PY{p}{(}\PY{p}{)}
\PY{n}{datf\PYZus{}cont}\PY{o}{=}\PY{n}{df\PYZus{}describe}\PY{p}{[}\PY{l+s+s2}{\PYZdq{}}\PY{l+s+s2}{Concentración}\PY{l+s+s2}{\PYZdq{}}\PY{p}{]}
\PY{c+c1}{\PYZsh{} err\PYZus{}std = datf\PYZus{}cont[\PYZdq{}std\PYZdq{}] / np.sqrt(datf\PYZus{}cont[\PYZdq{}count\PYZdq{}])}
\PY{c+c1}{\PYZsh{} datf\PYZus{}cont.insert(3, \PYZdq{}Error\PYZdq{}, err\PYZus{}std)}
\PY{c+c1}{\PYZsh{} df\PYZus{}reset = df\PYZus{}describe.reset\PYZus{}index(drop=True)}
\PY{c+c1}{\PYZsh{} dfi.export(datf\PYZus{}cont, f\PYZdq{}tabla\PYZus{}df\PYZus{}\PYZob{}nombre\PYZus{}variable.replace(\PYZsq{} \PYZsq{},\PYZsq{}\PYZus{}\PYZsq{})\PYZcb{}.png\PYZdq{})}
\end{Verbatim}
\end{tcolorbox}

    Como ejemplo: la tabla del DataFrame presentada se generó utilizando el
método \texttt{describe()} aplicado a un DataFrame previamente agrupado
con el método \texttt{groupby()}, a partir de un subconjunto de datos
que corresponden a la variable climática \texttt{temperatura\_10m.csv}.
En este proceso, el método \texttt{groupby()} se utilizó para clasificar
los datos por municipio, segmentando el conjunto de datos en grupos
distintos según la ubicación geográfica. Posteriormente, el método
\texttt{describe()} fue aplicado a cada uno de estos grupos, calculando
y mostrando una serie de estadísticas descriptivas para cada municipio.
Estas estadísticas incluyen el conteo de datos (\texttt{count}), la
media (\texttt{mean}), la desviación estándar (\texttt{std}), entre
otras métricas, proporcionando un resumen detallado de la distribución
de las temperaturas a 10 metros en cada municipio. Este enfoque permite
una visión más granular de los datos, destacando las variaciones y
tendencias específicas para cada localidad.

    \begin{tcolorbox}[breakable, size=fbox, boxrule=1pt, pad at break*=1mm,colback=cellbackground, colframe=cellborder]
\prompt{In}{incolor}{22}{\boxspacing}
\begin{Verbatim}[commandchars=\\\{\}]
\PY{n}{imshow\PYZus{}plots}\PY{p}{(}\PY{l+s+s2}{\PYZdq{}}\PY{l+s+s2}{tabla\PYZus{}df\PYZus{}Temperatura\PYZus{}a\PYZus{}10\PYZus{}m.png}\PY{l+s+s2}{\PYZdq{}}\PY{p}{)}
\end{Verbatim}
\end{tcolorbox}

    \begin{center}
    \adjustimage{max size={0.9\linewidth}{0.9\paperheight}}{datosaire_files/datosaire_40_0.png}
    \end{center}
    { \hspace*{\fill} \\}
    
    A continuación, mostramos el código cuyo objetivo es crear un gráfico
que ilustra el comportamiento de los datos por región del dataframe
anterior. Cabe aclarar que el código presentado es general y puede
generar gráficos para todas las variables disponibles en el conjunto de
datos. Sin embargo, a modo de ejemplo, en éste documento creamos solo un
gráfico, ya que existen un total de 20 variables y mostrar gráficos para
cada una de ellas no sería práctico. El propósito principal aquí es
mostrarle al lector una visualización mas detallada de la variabilidad
de los datos y destacar los problemas asociados con su recolección, ya
que en muchos casos no es uniforme, lo que puede afectar la
interpretación y análisis de los resultados al enfrentarnos con muestras
de diferentes tamaños.

    \begin{tcolorbox}[breakable, size=fbox, boxrule=1pt, pad at break*=1mm,colback=cellbackground, colframe=cellborder]
\prompt{In}{incolor}{ }{\boxspacing}
\begin{Verbatim}[commandchars=\\\{\}]
\PY{n}{fig} \PY{o}{=} \PY{n}{plt}\PY{o}{.}\PY{n}{figure}\PY{p}{(}\PY{n}{figsize}\PY{o}{=}\PY{p}{(}\PY{l+m+mi}{27}\PY{p}{,} \PY{l+m+mi}{17}\PY{p}{)}\PY{p}{)}
\PY{n}{grid} \PY{o}{=} \PY{n}{fig}\PY{o}{.}\PY{n}{add\PYZus{}gridspec}\PY{p}{(}\PY{l+m+mi}{2}\PY{p}{,} \PY{l+m+mi}{2}\PY{p}{,} \PY{n}{width\PYZus{}ratios}\PY{o}{=}\PY{p}{[}\PY{l+m+mi}{1}\PY{p}{,} \PY{l+m+mi}{2}\PY{p}{]}\PY{p}{)}
\PY{n}{ax0} \PY{o}{=} \PY{n}{fig}\PY{o}{.}\PY{n}{add\PYZus{}subplot}\PY{p}{(}\PY{n}{grid}\PY{p}{[}\PY{p}{:}\PY{p}{,} \PY{l+m+mi}{0}\PY{p}{]}\PY{p}{)}
\PY{n}{ax1} \PY{o}{=} \PY{n}{fig}\PY{o}{.}\PY{n}{add\PYZus{}subplot}\PY{p}{(}\PY{n}{grid}\PY{p}{[}\PY{l+m+mi}{0}\PY{p}{,} \PY{l+m+mi}{1}\PY{p}{]}\PY{p}{)}
\PY{n}{ax2} \PY{o}{=} \PY{n}{fig}\PY{o}{.}\PY{n}{add\PYZus{}subplot}\PY{p}{(}\PY{n}{grid}\PY{p}{[}\PY{l+m+mi}{1}\PY{p}{,} \PY{l+m+mi}{1}\PY{p}{]}\PY{p}{)}

\PY{n}{datf\PYZus{}cont}\PY{p}{[}\PY{l+s+s2}{\PYZdq{}}\PY{l+s+s2}{count}\PY{l+s+s2}{\PYZdq{}}\PY{p}{]}\PY{o}{.}\PY{n}{plot}\PY{p}{(}\PY{n}{kind}\PY{o}{=}\PY{l+s+s2}{\PYZdq{}}\PY{l+s+s2}{bar}\PY{l+s+s2}{\PYZdq{}}\PY{p}{,} \PY{n}{linewidth}\PY{o}{=}\PY{l+m+mi}{2}\PY{p}{,}\PY{n}{ax}\PY{o}{=}\PY{n}{ax1}\PY{p}{,} \PY{n}{width}\PY{o}{=}\PY{l+m+mi}{1}\PY{p}{,} \PY{n}{ec}\PY{o}{=}\PY{l+s+s2}{\PYZdq{}}\PY{l+s+s2}{w}\PY{l+s+s2}{\PYZdq{}}\PY{p}{,} \PY{n}{color}\PY{o}{=}\PY{l+s+s2}{\PYZdq{}}\PY{l+s+s2}{\PYZsh{}008B8B}\PY{l+s+s2}{\PYZdq{}}\PY{p}{,} \PY{n}{legend}\PY{o}{=}\PY{k+kc}{False}\PY{p}{)}
\PY{n}{datf\PYZus{}cont}\PY{p}{[}\PY{l+s+s2}{\PYZdq{}}\PY{l+s+s2}{mean}\PY{l+s+s2}{\PYZdq{}}\PY{p}{]}\PY{o}{.}\PY{n}{plot}\PY{p}{(}
    \PY{n}{yerr}\PY{o}{=}\PY{n}{datf\PYZus{}cont}\PY{p}{[}\PY{l+s+s2}{\PYZdq{}}\PY{l+s+s2}{std}\PY{l+s+s2}{\PYZdq{}}\PY{p}{]}\PY{p}{,}
    \PY{n}{kind}\PY{o}{=}\PY{l+s+s2}{\PYZdq{}}\PY{l+s+s2}{bar}\PY{l+s+s2}{\PYZdq{}}\PY{p}{,}
    \PY{n}{capsize}\PY{o}{=}\PY{l+m+mi}{5}\PY{p}{,}
    \PY{n}{ecolor}\PY{o}{=}\PY{l+s+s2}{\PYZdq{}}\PY{l+s+s2}{red}\PY{l+s+s2}{\PYZdq{}}\PY{p}{,}
    \PY{n}{linewidth}\PY{o}{=}\PY{l+m+mi}{2}\PY{p}{,}
    \PY{n}{ax}\PY{o}{=}\PY{n}{ax2}\PY{p}{,}
    \PY{n}{width}\PY{o}{=}\PY{l+m+mi}{1}\PY{p}{,}
    \PY{n}{ec}\PY{o}{=}\PY{l+s+s2}{\PYZdq{}}\PY{l+s+s2}{w}\PY{l+s+s2}{\PYZdq{}}\PY{p}{,}
    \PY{n}{color}\PY{o}{=}\PY{l+s+s2}{\PYZdq{}}\PY{l+s+s2}{\PYZsh{}00008B}\PY{l+s+s2}{\PYZdq{}}\PY{p}{,}
    \PY{n}{legend}\PY{o}{=}\PY{k+kc}{False}\PY{p}{,}
\PY{p}{)}
\PY{n}{df\PYZus{}group\PYZus{}dfi} \PY{o}{=} \PY{n}{datf\PYZus{}cont}\PY{p}{[}\PY{p}{[}\PY{l+s+s2}{\PYZdq{}}\PY{l+s+s2}{count}\PY{l+s+s2}{\PYZdq{}}\PY{p}{,} \PY{l+s+s2}{\PYZdq{}}\PY{l+s+s2}{mean}\PY{l+s+s2}{\PYZdq{}}\PY{p}{]}\PY{p}{]}
\PY{n}{dfi}\PY{o}{.}\PY{n}{export}\PY{p}{(}\PY{n}{df\PYZus{}group\PYZus{}dfi}\PY{p}{,} \PY{l+s+s2}{\PYZdq{}}\PY{l+s+s2}{tabla\PYZus{}df.png}\PY{l+s+s2}{\PYZdq{}}\PY{p}{)}
\PY{n}{img} \PY{o}{=} \PY{n}{np}\PY{o}{.}\PY{n}{asarray}\PY{p}{(}\PY{n}{Image}\PY{o}{.}\PY{n}{open}\PY{p}{(}\PY{l+s+s2}{\PYZdq{}}\PY{l+s+s2}{tabla\PYZus{}df.png}\PY{l+s+s2}{\PYZdq{}}\PY{p}{)}\PY{p}{)}
\PY{n}{ax0}\PY{o}{.}\PY{n}{imshow}\PY{p}{(}\PY{n}{img}\PY{p}{)}
\PY{n}{ax0}\PY{o}{.}\PY{n}{axis}\PY{p}{(}\PY{l+s+s2}{\PYZdq{}}\PY{l+s+s2}{off}\PY{l+s+s2}{\PYZdq{}}\PY{p}{)}
\PY{n}{ax0}\PY{o}{.}\PY{n}{set\PYZus{}title}\PY{p}{(}
    \PY{l+s+sa}{f}\PY{l+s+s2}{\PYZdq{}}\PY{l+s+s2}{ Tabla descriptiva de }\PY{l+s+si}{\PYZob{}}\PY{n}{nombre\PYZus{}variable}\PY{l+s+si}{\PYZcb{}}\PY{l+s+s2}{ (}\PY{l+s+si}{\PYZob{}}\PY{n}{abreviatura\PYZus{}variable}\PY{l+s+si}{\PYZcb{}}\PY{l+s+s2}{)}\PY{l+s+s2}{\PYZdq{}}\PY{p}{,}
    \PY{n}{fontsize}\PY{o}{=}\PY{l+m+mi}{35}\PY{p}{,}
\PY{p}{)}

\PY{n}{ax1}\PY{o}{.}\PY{n}{xaxis}\PY{o}{.}\PY{n}{set\PYZus{}tick\PYZus{}params}\PY{p}{(}\PY{n}{labelsize}\PY{o}{=}\PY{l+m+mi}{22}\PY{p}{)}
\PY{n}{ax1}\PY{o}{.}\PY{n}{yaxis}\PY{o}{.}\PY{n}{set\PYZus{}tick\PYZus{}params}\PY{p}{(}\PY{n}{labelsize}\PY{o}{=}\PY{l+m+mi}{27}\PY{p}{)}
\PY{n}{ax1}\PY{o}{.}\PY{n}{set\PYZus{}xlabel}\PY{p}{(}\PY{l+s+sa}{f}\PY{l+s+s2}{\PYZdq{}}\PY{l+s+s2}{Municipio}\PY{l+s+s2}{\PYZdq{}}\PY{p}{,} \PY{n}{fontsize}\PY{o}{=}\PY{l+m+mi}{32}\PY{p}{)}
\PY{n}{ax1}\PY{o}{.}\PY{n}{set\PYZus{}ylabel}\PY{p}{(}\PY{l+s+sa}{f}\PY{l+s+s2}{\PYZdq{}}\PY{l+s+s2}{Conteo }\PY{l+s+si}{\PYZob{}}\PY{n}{abreviatura\PYZus{}variable}\PY{l+s+si}{\PYZcb{}}\PY{l+s+s2}{ (}\PY{l+s+si}{\PYZob{}}\PY{n}{unidad\PYZus{}variable}\PY{l+s+si}{\PYZcb{}}\PY{l+s+s2}{)}\PY{l+s+s2}{\PYZdq{}}\PY{p}{,} \PY{n}{fontsize}\PY{o}{=}\PY{l+m+mi}{32}\PY{p}{)}
\PY{n}{ax1}\PY{o}{.}\PY{n}{set\PYZus{}title}\PY{p}{(}
    \PY{l+s+sa}{f}\PY{l+s+s2}{\PYZdq{}}\PY{l+s+s2}{Conteo de }\PY{l+s+si}{\PYZob{}}\PY{n}{nombre\PYZus{}variable}\PY{l+s+si}{\PYZcb{}}\PY{l+s+s2}{ (}\PY{l+s+si}{\PYZob{}}\PY{n}{abreviatura\PYZus{}variable}\PY{l+s+si}{\PYZcb{}}\PY{l+s+s2}{) por municipio}\PY{l+s+s2}{\PYZdq{}}\PY{p}{,}
    \PY{n}{fontsize}\PY{o}{=}\PY{l+m+mi}{35}\PY{p}{,}
\PY{p}{)}

\PY{n}{ax2}\PY{o}{.}\PY{n}{xaxis}\PY{o}{.}\PY{n}{set\PYZus{}tick\PYZus{}params}\PY{p}{(}\PY{n}{labelsize}\PY{o}{=}\PY{l+m+mi}{22}\PY{p}{)}
\PY{n}{ax2}\PY{o}{.}\PY{n}{yaxis}\PY{o}{.}\PY{n}{set\PYZus{}tick\PYZus{}params}\PY{p}{(}\PY{n}{labelsize}\PY{o}{=}\PY{l+m+mi}{27}\PY{p}{)}
\PY{n}{ax2}\PY{o}{.}\PY{n}{set\PYZus{}xlabel}\PY{p}{(}\PY{l+s+sa}{f}\PY{l+s+s2}{\PYZdq{}}\PY{l+s+s2}{Municipio}\PY{l+s+s2}{\PYZdq{}}\PY{p}{,} \PY{n}{fontsize}\PY{o}{=}\PY{l+m+mi}{32}\PY{p}{)}
\PY{n}{ax2}\PY{o}{.}\PY{n}{set\PYZus{}ylabel}\PY{p}{(}
    \PY{l+s+sa}{rf}\PY{l+s+s2}{\PYZdq{}}\PY{l+s+s2}{\PYZdl{}}\PY{l+s+s2}{\PYZbs{}}\PY{l+s+s2}{langle\PYZdl{}}\PY{l+s+si}{\PYZob{}}\PY{n}{abreviatura\PYZus{}variable}\PY{l+s+si}{\PYZcb{}}\PY{l+s+s2}{\PYZdl{}}\PY{l+s+s2}{\PYZbs{}}\PY{l+s+s2}{rangle\PYZdl{} (}\PY{l+s+si}{\PYZob{}}\PY{n}{unidad\PYZus{}variable}\PY{l+s+si}{\PYZcb{}}\PY{l+s+s2}{)}\PY{l+s+s2}{\PYZdq{}}\PY{p}{,} \PY{n}{fontsize}\PY{o}{=}\PY{l+m+mi}{32}
\PY{p}{)}
\PY{n}{ax2}\PY{o}{.}\PY{n}{set\PYZus{}title}\PY{p}{(}
    \PY{l+s+sa}{f}\PY{l+s+s2}{\PYZdq{}}\PY{l+s+s2}{Promedio de }\PY{l+s+si}{\PYZob{}}\PY{n}{nombre\PYZus{}variable}\PY{l+s+si}{\PYZcb{}}\PY{l+s+s2}{ (}\PY{l+s+si}{\PYZob{}}\PY{n}{abreviatura\PYZus{}variable}\PY{l+s+si}{\PYZcb{}}\PY{l+s+s2}{) por municipio}\PY{l+s+s2}{\PYZdq{}}\PY{p}{,} \PY{n}{fontsize}\PY{o}{=}\PY{l+m+mi}{35}
\PY{p}{)}

\PY{n}{ax1}\PY{o}{.}\PY{n}{set\PYZus{}yscale}\PY{p}{(}\PY{l+s+s2}{\PYZdq{}}\PY{l+s+s2}{log}\PY{l+s+s2}{\PYZdq{}}\PY{p}{)}
\PY{n}{plt}\PY{o}{.}\PY{n}{setp}\PY{p}{(}\PY{n}{ax2}\PY{o}{.}\PY{n}{get\PYZus{}xticklabels}\PY{p}{(}\PY{p}{)}\PY{p}{,} \PY{n}{rotation}\PY{o}{=}\PY{l+m+mi}{45}\PY{p}{,} \PY{n}{ha}\PY{o}{=}\PY{l+s+s2}{\PYZdq{}}\PY{l+s+s2}{right}\PY{l+s+s2}{\PYZdq{}}\PY{p}{,} \PY{n}{rotation\PYZus{}mode}\PY{o}{=}\PY{l+s+s2}{\PYZdq{}}\PY{l+s+s2}{anchor}\PY{l+s+s2}{\PYZdq{}}\PY{p}{)}
\PY{n}{plt}\PY{o}{.}\PY{n}{setp}\PY{p}{(}\PY{n}{ax1}\PY{o}{.}\PY{n}{get\PYZus{}xticklabels}\PY{p}{(}\PY{p}{)}\PY{p}{,} \PY{n}{rotation}\PY{o}{=}\PY{l+m+mi}{45}\PY{p}{,} \PY{n}{ha}\PY{o}{=}\PY{l+s+s2}{\PYZdq{}}\PY{l+s+s2}{right}\PY{l+s+s2}{\PYZdq{}}\PY{p}{,} \PY{n}{rotation\PYZus{}mode}\PY{o}{=}\PY{l+s+s2}{\PYZdq{}}\PY{l+s+s2}{anchor}\PY{l+s+s2}{\PYZdq{}}\PY{p}{)}
\PY{n}{fig}\PY{o}{.}\PY{n}{tight\PYZus{}layout}\PY{p}{(}\PY{p}{)}
\PY{n}{plt}\PY{o}{.}\PY{n}{subplots\PYZus{}adjust}\PY{p}{(}\PY{n}{wspace}\PY{o}{=}\PY{l+m+mf}{0.15}\PY{p}{)}
\PY{n}{fig}\PY{o}{.}\PY{n}{savefig}\PY{p}{(}\PY{l+s+sa}{f}\PY{l+s+s2}{\PYZdq{}}\PY{l+s+s2}{plots/temperatura\PYZus{}10m.png}\PY{l+s+s2}{\PYZdq{}}\PY{p}{,} \PY{n}{bbox\PYZus{}inches}\PY{o}{=}\PY{l+s+s2}{\PYZdq{}}\PY{l+s+s2}{tight}\PY{l+s+s2}{\PYZdq{}}\PY{p}{)}
\PY{n}{plt}\PY{o}{.}\PY{n}{show}\PY{p}{(}\PY{p}{)}
\end{Verbatim}
\end{tcolorbox}

    La gráfica presentada muestra dos aspectos clave del conjunto de datos
\texttt{temperatura\_10m.csv} relacionados con la temperatura medida a
10 metros desde el punto de referencia: el conteo de los datos y la
media calculada para cada grupo de datos climáticos correspondiente a
cada municipio. El gráfico de barras verde azulado representa el conteo
de datos obtenidos para cada municipio, mientras que el gráfico de
barras azul muestra la media de las temperaturas registradas a 10 metros
\(\langle T_{10} \rangle\), desde el punto de referencia para cada uno
de estos grupos de datos climáticos.

El eje vertical del gráfico de conteo de datos se ha ajustado a una
escala logarítmica para mejorar la visualización, ya que las barras muy
altas en una escala lineal podrían ocultar las barras donde los datos
son menos frecuentes. Este ajuste permite una representación más clara
de las variaciones en el número de mediciones.

Las barras de error en color rojo indican la desviación estándar de los
datos, proporcionando una visualización de la dispersión alrededor de la
media. Se observa que la dispersión en Santa Marta es muy significativa,
sugiriendo una alta variabilidad en las temperaturas registradas,
mientras que en Caucasia la dispersión es notablemente pequeña,
indicando una menor variabilidad.

Además, el conteo de datos en el DataFrame y en el gráfico revela una
discrepancia significativa en el número de mediciones entre municipios.
Esta discrepancia es un factor importante que debe ser considerado al
interpretar los resultados y al concluir los análisis, ya que puede
afectar la representatividad y precisión de las conclusiones sobre las
condiciones climáticas en diferentes localidades.

Para obtener el DataFrame utilizado para crear estas gráficas, se aplicó
el método \texttt{groupby()} para agrupar los datos por municipio,
seguido del método \texttt{describe()} para calcular las estadísticas
descriptivas. Posteriormente, el método \texttt{df.plot()} se empleó
para visualizar tanto el conteo de los datos como la media, junto con
las barras de error, facilitando una comprensión detallada de las
variaciones y tendencias en las temperaturas a 10 metros desde el punto
de referencia a nivel municipal.

    \begin{tcolorbox}[breakable, size=fbox, boxrule=1pt, pad at break*=1mm,colback=cellbackground, colframe=cellborder]
\prompt{In}{incolor}{19}{\boxspacing}
\begin{Verbatim}[commandchars=\\\{\}]
\PY{n}{imshow\PYZus{}plots}\PY{p}{(}\PY{l+s+s2}{\PYZdq{}}\PY{l+s+s2}{plots/temperatura\PYZus{}10m.png}\PY{l+s+s2}{\PYZdq{}}\PY{p}{)}
\end{Verbatim}
\end{tcolorbox}

    \begin{center}
    \adjustimage{max size={0.9\linewidth}{0.9\paperheight}}{datosaire_files/datosaire_44_0.png}
    \end{center}
    { \hspace*{\fill} \\}
    
    \hypertarget{revisiuxf3n-de-datos-locales-y-aplicaciuxf3n-del-criterio-iqr-en-series-temporales-climuxe1ticas}{%
\subsection{Revisión de datos locales y aplicación del criterio IQR en
series temporales
climáticas}\label{revisiuxf3n-de-datos-locales-y-aplicaciuxf3n-del-criterio-iqr-en-series-temporales-climuxe1ticas}}

    Esta sección, vamos a emplear un enfoque basado en el rango
intercuartílico (IQR) para identificar y eliminar los datos que se
consideran outliers excesivamente inusuales. Este criterio es
especialmente útil en el análisis climático, donde los valores extremos
pueden corresponder a eventos atípicos o errores de medición, los
cuales, si no se manejan adecuadamente, pueden distorsionar los
resultados generales.

El IQR, que mide la dispersión de los datos centrales, se calcula como
la diferencia entre el tercer cuartil (\(Q^+\)) y el primer cuartil
(\(Q^-\)). Para identificar los outliers, aplicaremos el rango de
\(Q^{\pm} \pm 1.5IQR\), donde \$IQR = Q\^{}\{+\} - Q\^{}\{-\} \$. Los
valores que caigan fuera de este rango serán considerados outliers y
eliminados, ya que están demasiado alejados de las condiciones normales
y podrían afectar negativamente la interpretación de los patrones
generales.

Este método tiene la ventaja de ser robusto ante la presencia de
distribuciones no normales, lo que lo convierte en una herramienta más
confiable que el uso de desviaciones estándar en conjuntos de datos con
alta variabilidad. Al eliminar estos outliers, nos aseguramos de que el
análisis climático mantenga su integridad, capturando las tendencias
principales sin verse distorsionado por eventos raros que no reflejan el
comportamiento típico del sistema climático.

    Inicialmente vamos a definir algunas variables de interés que serán
necesarias para el siguiente análisis.

    \begin{tcolorbox}[breakable, size=fbox, boxrule=1pt, pad at break*=1mm,colback=cellbackground, colframe=cellborder]
\prompt{In}{incolor}{21}{\boxspacing}
\begin{Verbatim}[commandchars=\\\{\}]
\PY{n}{fecha24h} \PY{o}{=} \PY{l+s+s2}{\PYZdq{}}\PY{l+s+s2}{Fecha\PYZus{}24h}\PY{l+s+s2}{\PYZdq{}}  \PY{c+c1}{\PYZsh{} datos fecha de eventos}
\PY{n}{pathdir} \PY{o}{=} \PY{l+s+s2}{\PYZdq{}}\PY{l+s+s2}{datos\PYZus{}segmentados}\PY{l+s+s2}{\PYZdq{}}  \PY{c+c1}{\PYZsh{} directorio datos agrupados}
\PY{n}{formatofecha} \PY{o}{=} \PY{l+s+s2}{\PYZdq{}}\PY{l+s+s2}{\PYZpc{}}\PY{l+s+s2}{Y/}\PY{l+s+s2}{\PYZpc{}}\PY{l+s+s2}{m/}\PY{l+s+si}{\PYZpc{}d}\PY{l+s+s2}{ }\PY{l+s+s2}{\PYZpc{}}\PY{l+s+s2}{H:}\PY{l+s+s2}{\PYZpc{}}\PY{l+s+s2}{M:}\PY{l+s+s2}{\PYZpc{}}\PY{l+s+s2}{S}\PY{l+s+s2}{\PYZdq{}}  \PY{c+c1}{\PYZsh{} formato para la fecha.}
\PY{n}{fechahoras} \PY{o}{=} \PY{l+s+s2}{\PYZdq{}}\PY{l+s+s2}{Fecha\PYZus{}horas}\PY{l+s+s2}{\PYZdq{}}
\PY{n}{densidad} \PY{o}{=} \PY{l+s+s2}{\PYZdq{}}\PY{l+s+s2}{Concentración}\PY{l+s+s2}{\PYZdq{}}  \PY{c+c1}{\PYZsh{} datos variable admosférica}
\PY{n}{municipio} \PY{o}{=} \PY{l+s+s2}{\PYZdq{}}\PY{l+s+s2}{Nombre del municipio}\PY{l+s+s2}{\PYZdq{}}  \PY{c+c1}{\PYZsh{} datos fecha en formato 24 horas}
\PY{n}{periodo} \PY{o}{=} \PY{l+s+s2}{\PYZdq{}}\PY{l+s+s2}{H}\PY{l+s+s2}{\PYZdq{}}  \PY{c+c1}{\PYZsh{} agrupe los datos por periodo: hora (H), dia(D), semana(W), mensual(M), anual (H)}
\PY{n}{varperiodo} \PY{o}{=} \PY{l+s+s2}{\PYZdq{}}\PY{l+s+s2}{Periodo}\PY{l+s+s2}{\PYZdq{}}  \PY{c+c1}{\PYZsh{} creando columna de datos agrupados por periodo}
\PY{n}{varestadist} \PY{o}{=} \PY{p}{(}\PY{l+s+s2}{\PYZdq{}}\PY{l+s+s2}{mean}\PY{l+s+s2}{\PYZdq{}}\PY{p}{)}  \PY{c+c1}{\PYZsh{} estadística de interés, media(mean) desviación estándar (std), etc.}
\PY{n}{id\PYZus{}municipio} \PY{o}{=} \PY{l+s+s2}{\PYZdq{}}\PY{l+s+s2}{Código del municipio}\PY{l+s+s2}{\PYZdq{}}  \PY{c+c1}{\PYZsh{} Código del municipio}
\PY{n}{unid} \PY{o}{=} \PY{l+s+s2}{\PYZdq{}}\PY{l+s+s2}{unidad}\PY{l+s+s2}{\PYZdq{}}       \PY{c+c1}{\PYZsh{}unidades de cada variable}
\PY{n}{abrev} \PY{o}{=} \PY{l+s+s2}{\PYZdq{}}\PY{l+s+s2}{abreviatura}\PY{l+s+s2}{\PYZdq{}}    \PY{c+c1}{\PYZsh{} símbolo de abreviatura de variable}
\PY{n}{variable}\PY{o}{=} \PY{l+s+s2}{\PYZdq{}}\PY{l+s+s2}{variable}\PY{l+s+s2}{\PYZdq{}}     \PY{c+c1}{\PYZsh{}variable en estudio}
\PY{n}{color} \PY{o}{=} \PY{n}{np}\PY{o}{.}\PY{n}{random}\PY{o}{.}\PY{n}{rand}\PY{p}{(}\PY{l+m+mi}{200}\PY{p}{,} \PY{l+m+mi}{3}\PY{p}{)}
\PY{n}{mcolr} \PY{o}{=} \PY{n}{np}\PY{o}{.}\PY{n}{mean}\PY{p}{(}\PY{n}{color}\PY{p}{,} \PY{n}{axis}\PY{o}{=}\PY{l+m+mi}{1}\PY{p}{)}
\PY{n}{colores} \PY{o}{=} \PY{n}{color}\PY{p}{[}\PY{n}{np}\PY{o}{.}\PY{n}{where}\PY{p}{(}\PY{n}{mcolr} \PY{o}{\PYZlt{}} \PY{l+m+mf}{0.5}\PY{p}{)}\PY{p}{]}
\PY{n}{adres\PYZus{}dir} \PY{o}{=} \PY{n}{os}\PY{o}{.}\PY{n}{listdir}\PY{p}{(}\PY{n}{pathdir}\PY{p}{)}
\PY{n}{pathfil} \PY{o}{=} \PY{n+nb}{list}\PY{p}{(}\PY{n}{variables\PYZus{}contaminacion}\PY{o}{.}\PY{n}{keys}\PY{p}{(}\PY{p}{)}\PY{p}{)}  
\end{Verbatim}
\end{tcolorbox}

    En el siguiente fragmento de código, se generan histogramas de
frecuencia para cada una de las variables climáticas de los municipios
en estudio. Estas variables incluyen factores como la dirección del
viento, la temperatura, la humedad relativa, y la precipitación, entre
otras. Los histogramas permiten visualizar la distribución de los datos,
mostrando cuántas veces ocurren ciertos valores dentro de un rango
específico.

El código no solo genera estos histogramas, sino que también calcula las
correlaciones de Pearson entre los diferentes histogramas de frecuencia
de las variables climáticas. Este análisis de correlación permite
identificar qué tan relacionadas están las distribuciones de dos
variables. Un valor de correlación cercano a 1 indica una relación
positiva fuerte, mientras que un valor cercano a -1 indica una relación
negativa fuerte. Un valor cercano a 0 sugiere que no hay relación
significativa entre las variables.

El propósito de este análisis es evaluar cómo se comportan las distintas
variables climáticas en cada municipio y cómo se relacionan entre sí en
cada región, lo que podría ser útil para identificar patrones climáticos
o detectar comportamientos anómalos en diferentes zonas.

    La función \texttt{function\_df\_clima} está diseñada con el objetivo de
identificar y retirar los valores atípicos (outliers) de los diferentes
conjuntos de datos climáticos de cada municipio.

    \begin{tcolorbox}[breakable, size=fbox, boxrule=1pt, pad at break*=1mm,colback=cellbackground, colframe=cellborder]
\prompt{In}{incolor}{27}{\boxspacing}
\begin{Verbatim}[commandchars=\\\{\}]
\PY{k}{def} \PY{n+nf}{function\PYZus{}df\PYZus{}clima}\PY{p}{(}\PY{n}{var\PYZus{}clima}\PY{p}{)}\PY{p}{:}
    \PY{n+nb}{print}\PY{p}{(}\PY{n}{var\PYZus{}clima}\PY{p}{)}
    \PY{n}{adres\PYZus{}dir} \PY{o}{=} \PY{n}{os}\PY{o}{.}\PY{n}{listdir}\PY{p}{(}\PY{l+s+sa}{f}\PY{l+s+s2}{\PYZdq{}}\PY{l+s+si}{\PYZob{}}\PY{n}{pathdir}\PY{l+s+si}{\PYZcb{}}\PY{l+s+s2}{/}\PY{l+s+si}{\PYZob{}}\PY{n}{var\PYZus{}clima}\PY{l+s+si}{\PYZcb{}}\PY{l+s+s2}{\PYZdq{}}\PY{p}{)}
    \PY{n}{nbins\PYZus{}apped} \PY{o}{=} \PY{p}{[}\PY{p}{]}\PY{p}{;} \PY{n}{df\PYZus{}csv} \PY{o}{=} \PY{p}{[}\PY{p}{]}\PY{p}{;} \PY{n}{dats\PYZus{}min} \PY{o}{=} \PY{p}{[}\PY{p}{]}\PY{p}{;} \PY{n}{df\PYZus{}densidad} \PY{o}{=} \PY{p}{[}\PY{p}{]}\PY{p}{;} \PY{n}{municipios} \PY{o}{=} \PY{p}{[}\PY{p}{]}
    \PY{k}{for} \PY{n}{datsmun} \PY{o+ow}{in} \PY{n}{adres\PYZus{}dir}\PY{p}{:}
        \PY{n}{dfcsv} \PY{o}{=} \PY{n}{pd}\PY{o}{.}\PY{n}{read\PYZus{}csv}\PY{p}{(}\PY{l+s+sa}{f}\PY{l+s+s2}{\PYZdq{}}\PY{l+s+si}{\PYZob{}}\PY{n}{pathdir}\PY{l+s+si}{\PYZcb{}}\PY{l+s+s2}{/}\PY{l+s+si}{\PYZob{}}\PY{n}{var\PYZus{}clima}\PY{l+s+si}{\PYZcb{}}\PY{l+s+s2}{/}\PY{l+s+si}{\PYZob{}}\PY{n}{datsmun}\PY{l+s+si}{\PYZcb{}}\PY{l+s+s2}{\PYZdq{}}\PY{p}{,} \PY{n}{dtype}\PY{o}{=}\PY{n+nb}{str}\PY{p}{)}
        \PY{n}{dats\PYZus{}min}\PY{o}{.}\PY{n}{append}\PY{p}{(}\PY{n+nb}{len}\PY{p}{(}\PY{n}{dfcsv}\PY{p}{[}\PY{n}{densidad}\PY{p}{]}\PY{p}{)}\PY{p}{)}
        \PY{n}{df\PYZus{}csv}\PY{o}{.}\PY{n}{append}\PY{p}{(}\PY{n}{dfcsv}\PY{p}{)}
    \PY{n}{ndats} \PY{o}{=} \PY{n+nb}{pow}\PY{p}{(}\PY{l+m+mi}{10}\PY{p}{,} \PY{n}{np}\PY{o}{.}\PY{n}{floor}\PY{p}{(}\PY{n}{np}\PY{o}{.}\PY{n}{log10}\PY{p}{(}\PY{n}{np}\PY{o}{.}\PY{n}{mean}\PY{p}{(}\PY{n}{dats\PYZus{}min}\PY{p}{)}\PY{p}{)}\PY{p}{)}\PY{p}{)}
    \PY{k}{for} \PY{n}{dfcsv} \PY{o+ow}{in} \PY{n}{df\PYZus{}csv}\PY{p}{:}
        \PY{k}{if} \PY{n+nb}{len}\PY{p}{(}\PY{n}{dfcsv}\PY{p}{[}\PY{n}{densidad}\PY{p}{]}\PY{p}{)} \PY{o}{\PYZgt{}}\PY{o}{=} \PY{n}{ndats}\PY{p}{:}
            \PY{n}{dfcsv}\PY{p}{[}\PY{n}{fecha24h}\PY{p}{]} \PY{o}{=} \PY{n}{pd}\PY{o}{.}\PY{n}{to\PYZus{}datetime}\PY{p}{(}\PY{n}{dfcsv}\PY{p}{[}\PY{n}{fecha24h}\PY{p}{]}\PY{p}{,} \PY{n+nb}{format}\PY{o}{=}\PY{n}{formatofecha}\PY{p}{)}
            \PY{n}{dfcsv} \PY{o}{=} \PY{n}{dfcsv}\PY{o}{.}\PY{n}{set\PYZus{}index}\PY{p}{(}\PY{n}{fecha24h}\PY{p}{)}\PY{o}{.}\PY{n}{copy}\PY{p}{(}\PY{p}{)}
            \PY{n}{dfcsv}\PY{p}{[}\PY{n}{densidad}\PY{p}{]} \PY{o}{=} \PY{n}{dfcsv}\PY{p}{[}\PY{n}{densidad}\PY{p}{]}\PY{o}{.}\PY{n}{astype}\PY{p}{(}\PY{n+nb}{float}\PY{p}{)}
            \PY{n}{umbQ1}\PY{p}{,} \PY{n}{umbQ3} \PY{o}{=} \PY{n}{umbral\PYZus{}iqr}\PY{p}{(}\PY{n}{dfcsv}\PY{p}{,} \PY{n}{densidad}\PY{p}{)}
            \PY{n}{df\PYZus{}loc1} \PY{o}{=} \PY{n}{dfcsv}\PY{p}{[}\PY{p}{(}\PY{n}{dfcsv}\PY{p}{[}\PY{n}{densidad}\PY{p}{]} \PY{o}{\PYZlt{}} \PY{n}{umbQ3}\PY{p}{)} \PY{o}{\PYZam{}} \PY{p}{(}\PY{n}{dfcsv}\PY{p}{[}\PY{n}{densidad}\PY{p}{]} \PY{o}{\PYZgt{}} \PY{n}{umbQ1}\PY{p}{)}\PY{p}{]}
            \PY{k}{if} \PY{n+nb}{len}\PY{p}{(}\PY{n}{df\PYZus{}loc1}\PY{p}{[}\PY{n}{densidad}\PY{p}{]}\PY{p}{)}\PY{o}{!=}\PY{l+m+mi}{0}\PY{p}{:}
                \PY{n}{municipios}\PY{o}{.}\PY{n}{append}\PY{p}{(}\PY{n}{df\PYZus{}loc1}\PY{p}{[}\PY{n}{municipio}\PY{p}{]}\PY{o}{.}\PY{n}{unique}\PY{p}{(}\PY{p}{)}\PY{o}{.}\PY{n}{tolist}\PY{p}{(}\PY{p}{)}\PY{p}{[}\PY{l+m+mi}{0}\PY{p}{]}\PY{p}{)}
                \PY{n}{df\PYZus{}densidad}\PY{o}{.}\PY{n}{append}\PY{p}{(}\PY{n}{df\PYZus{}loc1}\PY{p}{)}
                \PY{n}{k} \PY{o}{=} \PY{n+nb}{len}\PY{p}{(}\PY{n}{np}\PY{o}{.}\PY{n}{histogram\PYZus{}bin\PYZus{}edges}\PY{p}{(}\PY{n}{df\PYZus{}loc1}\PY{p}{[}\PY{n}{densidad}\PY{p}{]}\PY{p}{,} \PY{n}{bins}\PY{o}{=}\PY{l+s+s2}{\PYZdq{}}\PY{l+s+s2}{auto}\PY{l+s+s2}{\PYZdq{}}\PY{p}{)}\PY{p}{)}
                \PY{n}{nbins\PYZus{}apped}\PY{o}{.}\PY{n}{append}\PY{p}{(}\PY{n}{k}\PY{p}{)}
            \PY{k}{else}\PY{p}{:}
                \PY{n+nb}{print}\PY{p}{(}\PY{l+s+s2}{\PYZdq{}}\PY{l+s+s2}{Dataframe vacío}\PY{l+s+s2}{\PYZdq{}}\PY{p}{,} \PY{n+nb}{len}\PY{p}{(}\PY{n}{df\PYZus{}loc1}\PY{p}{[}\PY{n}{densidad}\PY{p}{]}\PY{p}{)}\PY{p}{)}
    \PY{k}{if} \PY{n+nb}{len}\PY{p}{(}\PY{n}{nbins\PYZus{}apped}\PY{p}{)}\PY{o}{!=}\PY{l+m+mi}{0}\PY{p}{:}
        \PY{n}{nbins} \PY{o}{=} \PY{n+nb}{int}\PY{p}{(}\PY{n}{np}\PY{o}{.}\PY{n}{mean}\PY{p}{(}\PY{n}{nbins\PYZus{}apped}\PY{p}{)}\PY{p}{)}
        \PY{k}{return} \PY{n}{df\PYZus{}densidad}\PY{p}{,} \PY{n}{nbins}\PY{p}{,} \PY{n}{municipios}
    \PY{k}{else}\PY{p}{:}
        \PY{k}{return} \PY{p}{[}\PY{p}{]}\PY{p}{,}\PY{p}{[}\PY{p}{]}\PY{p}{,}\PY{p}{[}\PY{p}{]}
\end{Verbatim}
\end{tcolorbox}

    \begin{tcolorbox}[breakable, size=fbox, boxrule=1pt, pad at break*=1mm,colback=cellbackground, colframe=cellborder]
\prompt{In}{incolor}{ }{\boxspacing}
\begin{Verbatim}[commandchars=\\\{\}]
\PY{n}{mx\PYZus{}df}\PY{o}{=}\PY{p}{[}\PY{p}{]}\PY{p}{;}\PY{n}{list\PYZus{}bn}\PY{o}{=}\PY{p}{[}\PY{p}{]}\PY{p}{;}\PY{n}{lista\PYZus{}municipal}\PY{o}{=}\PY{p}{[}\PY{p}{]}
\PY{k}{for} \PY{n}{var\PYZus{}clima} \PY{o+ow}{in} \PY{p}{(}\PY{n}{pathfil}\PY{p}{)}\PY{p}{:}
    \PY{n}{lista\PYZus{}df}\PY{p}{,} \PY{n}{list\PYZus{}nbins}\PY{p}{,}\PY{n}{municipio\PYZus{}dep}\PY{o}{=}\PY{n}{function\PYZus{}df\PYZus{}clima}\PY{p}{(}\PY{n}{var\PYZus{}clima}\PY{p}{)}
    \PY{n}{lista\PYZus{}municipal}\PY{o}{.}\PY{n}{append}\PY{p}{(}\PY{n}{municipio\PYZus{}dep}\PY{p}{)}
    \PY{n}{mx\PYZus{}df}\PY{o}{.}\PY{n}{append}\PY{p}{(}\PY{n}{lista\PYZus{}df}\PY{p}{)}
    \PY{n}{list\PYZus{}bn}\PY{o}{.}\PY{n}{append}\PY{p}{(}\PY{n}{list\PYZus{}nbins}\PY{p}{)}
\end{Verbatim}
\end{tcolorbox}

    \begin{tcolorbox}[breakable, size=fbox, boxrule=1pt, pad at break*=1mm,colback=cellbackground, colframe=cellborder]
\prompt{In}{incolor}{ }{\boxspacing}
\begin{Verbatim}[commandchars=\\\{\}]
\PY{k}{for} \PY{n}{i} \PY{o+ow}{in} \PY{n+nb}{range}\PY{p}{(}\PY{l+m+mi}{0}\PY{p}{,}\PY{l+m+mi}{20}\PY{p}{,}\PY{l+m+mi}{2}\PY{p}{)}\PY{p}{:}
    \PY{n}{var} \PY{o}{=} \PY{n}{pathfil}\PY{p}{[}\PY{n}{i} \PY{p}{:} \PY{n}{i} \PY{o}{+} \PY{l+m+mi}{2}\PY{p}{]}
    \PY{n}{vr} \PY{o}{=} \PY{n}{variables\PYZus{}contaminacion}
    \PY{n}{ejx0} \PY{o}{=} \PY{l+s+sa}{f}\PY{l+s+s2}{\PYZdq{}}\PY{l+s+s2}{Tiempo (h)}\PY{l+s+s2}{\PYZdq{}}\PY{c+c1}{\PYZsh{}\PYZob{}vr[var[0]][abrev]\PYZcb{}(\PYZob{}vr[var[0]][unid]\PYZcb{})\PYZdq{}}
    \PY{n}{ejx1} \PY{o}{=} \PY{l+s+sa}{f}\PY{l+s+s2}{\PYZdq{}}\PY{l+s+s2}{Tiempo (h)}\PY{l+s+s2}{\PYZdq{}}  \PY{c+c1}{\PYZsh{} \PYZob{}vr[var[1]][abrev]\PYZcb{}(\PYZob{}vr[var[1]][unid]\PYZcb{})\PYZdq{}}
    \PY{n}{ejy0} \PY{o}{=} \PY{l+s+sa}{rf}\PY{l+s+s2}{\PYZdq{}}\PY{l+s+s2}{Densidad }\PY{l+s+si}{\PYZob{}}\PY{n}{vr}\PY{p}{[}\PY{n}{var}\PY{p}{[}\PY{l+m+mi}{0}\PY{p}{]}\PY{p}{]}\PY{p}{[}\PY{n}{abrev}\PY{p}{]}\PY{l+s+si}{\PYZcb{}}\PY{l+s+s2}{(}\PY{l+s+si}{\PYZob{}}\PY{n}{vr}\PY{p}{[}\PY{n}{var}\PY{p}{[}\PY{l+m+mi}{0}\PY{p}{]}\PY{p}{]}\PY{p}{[}\PY{n}{unid}\PY{p}{]}\PY{l+s+si}{\PYZcb{}}\PY{l+s+s2}{)}\PY{l+s+s2}{\PYZdq{}}
    \PY{n}{ejy1} \PY{o}{=} \PY{l+s+sa}{rf}\PY{l+s+s2}{\PYZdq{}}\PY{l+s+s2}{Densidad }\PY{l+s+si}{\PYZob{}}\PY{n}{vr}\PY{p}{[}\PY{n}{var}\PY{p}{[}\PY{l+m+mi}{1}\PY{p}{]}\PY{p}{]}\PY{p}{[}\PY{n}{abrev}\PY{p}{]}\PY{l+s+si}{\PYZcb{}}\PY{l+s+s2}{(}\PY{l+s+si}{\PYZob{}}\PY{n}{vr}\PY{p}{[}\PY{n}{var}\PY{p}{[}\PY{l+m+mi}{1}\PY{p}{]}\PY{p}{]}\PY{p}{[}\PY{n}{unid}\PY{p}{]}\PY{l+s+si}{\PYZcb{}}\PY{l+s+s2}{)}\PY{l+s+s2}{\PYZdq{}}
    \PY{n}{tit0} \PY{o}{=} \PY{l+s+sa}{rf}\PY{l+s+s2}{\PYZdq{}}\PY{l+s+s2}{Densidad de }\PY{l+s+si}{\PYZob{}}\PY{n}{vr}\PY{p}{[}\PY{n}{var}\PY{p}{[}\PY{l+m+mi}{0}\PY{p}{]}\PY{p}{]}\PY{p}{[}\PY{n}{abrev}\PY{p}{]}\PY{l+s+si}{\PYZcb{}}\PY{l+s+s2}{ vs. Tiempo}\PY{l+s+s2}{\PYZdq{}}
    \PY{n}{tit1} \PY{o}{=} \PY{l+s+sa}{f}\PY{l+s+s2}{\PYZdq{}}\PY{l+s+s2}{Densidad de }\PY{l+s+si}{\PYZob{}}\PY{n}{vr}\PY{p}{[}\PY{n}{var}\PY{p}{[}\PY{l+m+mi}{1}\PY{p}{]}\PY{p}{]}\PY{p}{[}\PY{n}{abrev}\PY{p}{]}\PY{l+s+si}{\PYZcb{}}\PY{l+s+s2}{ vs. Tiempo}\PY{l+s+s2}{\PYZdq{}}
    \PY{n}{figura} \PY{o}{=} \PY{n}{figure\PYZus{}plots}\PY{p}{(}
        \PY{n}{i}\PY{p}{,}
        \PY{n}{mx\PYZus{}df}\PY{p}{[}\PY{n}{i} \PY{p}{:} \PY{n}{i} \PY{o}{+} \PY{l+m+mi}{2}\PY{p}{]}\PY{p}{,}
        \PY{n}{list\PYZus{}bn}\PY{p}{[}\PY{n}{i} \PY{p}{:} \PY{n}{i} \PY{o}{+} \PY{l+m+mi}{2}\PY{p}{]}\PY{p}{,}
        \PY{n}{densidad}\PY{p}{,}
        \PY{n}{eje\PYZus{}x}\PY{o}{=}\PY{p}{[}\PY{n}{ejx0}\PY{p}{,} \PY{n}{ejx1}\PY{p}{]}\PY{p}{,}
        \PY{n}{eje\PYZus{}y}\PY{o}{=}\PY{p}{[}\PY{n}{ejy0}\PY{p}{,} \PY{n}{ejy1}\PY{p}{]}\PY{p}{,}
        \PY{n}{titulo}\PY{o}{=}\PY{p}{[}\PY{n}{tit0}\PY{p}{,} \PY{n}{tit1}\PY{p}{]}\PY{p}{,}
        \PY{n}{labelsize}\PY{o}{=}\PY{l+m+mi}{50}\PY{p}{,}
        \PY{n}{fontsize}\PY{o}{=}\PY{l+m+mi}{50}\PY{p}{,}
        \PY{n}{figsize}\PY{o}{=}\PY{p}{(}\PY{l+m+mi}{30}\PY{p}{,} \PY{l+m+mi}{12}\PY{p}{)}\PY{p}{,}
        \PY{n}{wspace}\PY{o}{=}\PY{l+m+mf}{0.25}\PY{p}{,}
    \PY{p}{)}
\end{Verbatim}
\end{tcolorbox}

    \hypertarget{representaciuxf3n-gruxe1fica-de-los-conjuntos-de-datos-en-funciuxf3n-del-tiempo}{%
\subsubsection{Representación gráfica de los conjuntos de datos en
función del
tiempo}\label{representaciuxf3n-gruxe1fica-de-los-conjuntos-de-datos-en-funciuxf3n-del-tiempo}}

    Las gráficas presentadas muestran algunos de los conjuntos de datos
correspondientes a diversas variables climáticas, donde se han excluido
cuidadosamente los datos atípicos. Esta exclusión asegura una
representación más precisa de las condiciones climáticas típicas en las
diferentes localidades analizadas. Cada color en las gráficas
corresponde a los datos de un municipio distinto, lo que facilita la
comparación entre las variaciones climáticas de cada región. Todas las
variables de estudio se grafican en función del tiempo, lo que permite
observar cómo evolucionan en cada localidad a lo largo del periodo
analizado. Al observar el comportamiento disperso en los datos, se puede
apreciar que, a pesar de ciertas similitudes generales, cada municipio
presenta patrones climáticos únicos, reflejando la diversidad geográfica
y meteorológica de las áreas estudiadas.

    \begin{tcolorbox}[breakable, size=fbox, boxrule=1pt, pad at break*=1mm,colback=cellbackground, colframe=cellborder]
\prompt{In}{incolor}{5}{\boxspacing}
\begin{Verbatim}[commandchars=\\\{\}]
\PY{n}{imshow\PYZus{}plots}\PY{p}{(}\PY{l+s+s2}{\PYZdq{}}\PY{l+s+s2}{plots/dispers18.png}\PY{l+s+s2}{\PYZdq{}}\PY{p}{)}\PY{p}{;} 
\end{Verbatim}
\end{tcolorbox}

    \begin{center}
    \adjustimage{max size={0.9\linewidth}{0.9\paperheight}}{datosaire_files/datosaire_56_0.png}
    \end{center}
    { \hspace*{\fill} \\}
    
    Los gráficos en función del tiempo muestran datos en diferentes escalas,
lo que refleja el comportamiento variado de las variables climáticas en
las distintas regiones geográficas. Este fenómeno se debe a las
diferencias geográficas y a las condiciones ambientales específicas de
cada zona. Por ejemplo, el gráfico de temperatura ilustra claramente
cómo los datos varían en escalas diferentes, ya que las temperaturas
registradas en las distintas regiones de Colombia son muy diversas.
Mientras que algunas zonas, como las regiones costeras, presentan
temperaturas más altas de manera consistente, otras áreas, como las
regiones montañosas, muestran temperaturas significativamente más bajas.
Esta variación en las escalas es un reflejo directo de la complejidad y
diversidad climática del país, donde factores como la altitud, la
proximidad al mar y las corrientes atmosféricas influyen de manera
notable en los patrones climáticos observados.

    \begin{tcolorbox}[breakable, size=fbox, boxrule=1pt, pad at break*=1mm,colback=cellbackground, colframe=cellborder]
\prompt{In}{incolor}{6}{\boxspacing}
\begin{Verbatim}[commandchars=\\\{\}]
\PY{n}{imshow\PYZus{}plots}\PY{p}{(}\PY{l+s+s2}{\PYZdq{}}\PY{l+s+s2}{plots/dispers16.png}\PY{l+s+s2}{\PYZdq{}}\PY{p}{)}
\end{Verbatim}
\end{tcolorbox}

    \begin{center}
    \adjustimage{max size={0.9\linewidth}{0.9\paperheight}}{datosaire_files/datosaire_58_0.png}
    \end{center}
    { \hspace*{\fill} \\}
    
    Los gráficos de humedad relativa y concentración de NO (óxido nítrico)
también muestran variaciones en la escala de los datos debido a las
diferencias geográficas y ambientales de las distintas regiones. La
\textbf{humedad relativa} es la cantidad de vapor de agua en el aire en
relación con la cantidad máxima que el aire puede retener a una
determinada temperatura. En zonas cercanas a cuerpos de agua o áreas con
alta vegetación, como selvas o costas, la humedad relativa suele ser
mayor, mientras que en regiones áridas o montañosas, la humedad tiende a
ser más baja. Estas diferencias en los niveles de humedad se reflejan
claramente en los gráficos, mostrando variaciones a lo largo del tiempo
dependiendo de las condiciones locales.

Por otro lado, la concentración de \textbf{NO (óxido nítrico)} en el
aire varía según factores como la actividad industrial, el tráfico
vehicular y la combustión de combustibles fósiles, que tienden a ser más
intensos en áreas urbanas. En las zonas rurales o con menos actividad
industrial, los niveles de NO suelen ser más bajos. Esto se refleja en
los gráficos de NO, donde las regiones urbanas presentan concentraciones
más altas en comparación con las rurales.

Ambas variables, la humedad relativa y la concentración de NO, están
influenciadas por la geografía y las condiciones ambientales locales.
Los gráficos muestran cómo las regiones con climas húmedos o cercanas a
áreas industriales tienden a tener escalas diferentes a las de zonas más
secas o con menor actividad humana. Estas variaciones en las condiciones
climáticas y de contaminación explican las diferencias observadas en los
datos de las diferentes regiones a lo largo del tiempo.

    \begin{tcolorbox}[breakable, size=fbox, boxrule=1pt, pad at break*=1mm,colback=cellbackground, colframe=cellborder]
\prompt{In}{incolor}{7}{\boxspacing}
\begin{Verbatim}[commandchars=\\\{\}]
\PY{n}{imshow\PYZus{}plots}\PY{p}{(}\PY{l+s+s2}{\PYZdq{}}\PY{l+s+s2}{plots/dispers4.png}\PY{l+s+s2}{\PYZdq{}}\PY{p}{)}
\end{Verbatim}
\end{tcolorbox}

    \begin{center}
    \adjustimage{max size={0.9\linewidth}{0.9\paperheight}}{datosaire_files/datosaire_60_0.png}
    \end{center}
    { \hspace*{\fill} \\}
    
    Los gráficos de presión atmosférica muestran variaciones en la escala de
los datos debido a diferencias geográficas y ambientales. La presión
atmosférica es mayor en regiones bajas, como las zonas costeras, y menor
en áreas montañosas, debido a que la cantidad de aire sobre una región
es mayor a nivel del mar y menor a altitudes elevadas. Además, factores
meteorológicos como los sistemas de alta y baja presión también influyen
en estas fluctuaciones.

El aire puede ser más denso por dos razones principales: la temperatura
y la altitud. El aire frío es más denso porque las moléculas se mueven
más lentamente y están más compactas, mientras que el aire caliente es
menos denso. A nivel del mar, la mayor cantidad de aire apilado sobre la
superficie genera mayor presión. En un \textbf{sistema de alta presión},
el aire frío y denso desciende, lo que provoca condiciones estables y
cielos despejados. En contraste, un \textbf{sistema de baja presión} se
forma cuando el aire más cálido y menos denso asciende, generando nubes
y lluvias.

Estas diferencias en la densidad del aire y en los sistemas de presión
atmosférica se reflejan en los gráficos. Las regiones de baja altitud y
con aire más frío tienden a mostrar presiones más altas, mientras que
las áreas montañosas o más cálidas presentan presiones más bajas. Estas
variaciones en los patrones atmosféricos explican la diversidad de los
datos observados en función del tiempo en las distintas regiones.

    \begin{tcolorbox}[breakable, size=fbox, boxrule=1pt, pad at break*=1mm,colback=cellbackground, colframe=cellborder]
\prompt{In}{incolor}{8}{\boxspacing}
\begin{Verbatim}[commandchars=\\\{\}]
\PY{n}{imshow\PYZus{}plots}\PY{p}{(}\PY{l+s+s2}{\PYZdq{}}\PY{l+s+s2}{plots/dispers12.png}\PY{l+s+s2}{\PYZdq{}}\PY{p}{)}
\end{Verbatim}
\end{tcolorbox}

    \begin{center}
    \adjustimage{max size={0.9\linewidth}{0.9\paperheight}}{datosaire_files/datosaire_62_0.png}
    \end{center}
    { \hspace*{\fill} \\}
    
    \hypertarget{histogramas-comparativos-por-municipios-de-las-variables-climuxe1ticas}{%
\subsubsection{Histogramas comparativos por municipios de las variables
climáticas}\label{histogramas-comparativos-por-municipios-de-las-variables-climuxe1ticas}}

    Los gráficos que se muestran a continuación corresponden a los
histogramas de las variables climáticas, clasificados por municipio.
Cada histograma ofrece una representación de la distribución de los
valores climáticos observados en cada localidad, proporcionando una
visión clara de la frecuencia con la que ocurren ciertos rangos de
valores. Estos gráficos permiten identificar patrones específicos dentro
de cada municipio, destacando la variabilidad y el comportamiento típico
de cada variable climática. Al comparar los histogramas entre
municipios, se pueden observar diferencias en la dispersión y
concentración de los datos, lo que evidencia la diversidad climática
entre las distintas regiones.

    \begin{tcolorbox}[breakable, size=fbox, boxrule=1pt, pad at break*=1mm,colback=cellbackground, colframe=cellborder]
\prompt{In}{incolor}{9}{\boxspacing}
\begin{Verbatim}[commandchars=\\\{\}]
\PY{n}{imshow\PYZus{}plots}\PY{p}{(}\PY{l+s+s2}{\PYZdq{}}\PY{l+s+s2}{plots/Graf19.png}\PY{l+s+s2}{\PYZdq{}}\PY{p}{)}\PY{p}{;} 
\end{Verbatim}
\end{tcolorbox}

    \begin{center}
    \adjustimage{max size={0.9\linewidth}{0.9\paperheight}}{datosaire_files/datosaire_65_0.png}
    \end{center}
    { \hspace*{\fill} \\}
    
    Los histogramas de la temperatura son notablemente más diversos en
comparación con los histogramas de PM (partículas en suspensión), lo
cual se debe a la mayor variabilidad de la temperatura en función de
factores geográficos y estacionales. La temperatura varía
significativamente entre regiones montañosas, costeras, urbanas y
rurales, así como entre estaciones del año, lo que genera una amplia
gama de distribuciones en los datos. Esto se refleja en los histogramas,
donde se pueden observar desde temperaturas cálidas y estables en zonas
costeras hasta fluctuaciones extremas en áreas montañosas o durante
cambios estacionales.

Por otro lado, los histogramas de \textbf{PM (partículas en suspensión)}
tienden a ser menos diversos porque la concentración de partículas
depende más de fuentes puntuales, como la actividad industrial, el
tráfico vehicular y las quemas agrícolas, que pueden ser constantes en
ciertas áreas urbanas o rurales. Aunque las condiciones climáticas como
el viento y la lluvia pueden influir en los niveles de PM, su
variabilidad es más limitada en comparación con la temperatura, que está
directamente relacionada con fenómenos geográficos y climáticos de mayor
magnitud. Por ello, los histogramas de PM muestran distribuciones más
homogéneas, especialmente en áreas donde las fuentes de contaminación
son similares a lo largo del tiempo.

    \begin{tcolorbox}[breakable, size=fbox, boxrule=1pt, pad at break*=1mm,colback=cellbackground, colframe=cellborder]
\prompt{In}{incolor}{10}{\boxspacing}
\begin{Verbatim}[commandchars=\\\{\}]
\PY{n}{imshow\PYZus{}plots}\PY{p}{(}\PY{l+s+s2}{\PYZdq{}}\PY{l+s+s2}{plots/Graf8.png}\PY{l+s+s2}{\PYZdq{}}\PY{p}{)}
\end{Verbatim}
\end{tcolorbox}

    \begin{center}
    \adjustimage{max size={0.9\linewidth}{0.9\paperheight}}{datosaire_files/datosaire_67_0.png}
    \end{center}
    { \hspace*{\fill} \\}
    
    \begin{tcolorbox}[breakable, size=fbox, boxrule=1pt, pad at break*=1mm,colback=cellbackground, colframe=cellborder]
\prompt{In}{incolor}{11}{\boxspacing}
\begin{Verbatim}[commandchars=\\\{\}]
\PY{n}{imshow\PYZus{}plots}\PY{p}{(}\PY{l+s+s2}{\PYZdq{}}\PY{l+s+s2}{plots/Graf16.png}\PY{l+s+s2}{\PYZdq{}}\PY{p}{)}
\end{Verbatim}
\end{tcolorbox}

    \begin{center}
    \adjustimage{max size={0.9\linewidth}{0.9\paperheight}}{datosaire_files/datosaire_68_0.png}
    \end{center}
    { \hspace*{\fill} \\}
    
    El siguiente código calcula las correlaciones entre la misma variable
climática, como la temperatura, en diferentes municipios, con el
objetivo de visualizar si los cambios en una variable en un lugar pueden
estar relacionados con los de otro municipio. Los resultados se muestran
utilizando gráficos con la función \texttt{imshow}, donde cada color
indica el nivel de correlación entre las variables climáticas de los
distintos municipios. Los colores más claros representan una mayor
correlación, mientras que los tonos más oscuros indican una correlación
más baja. Este análisis permite identificar si, por ejemplo, las
variaciones de temperatura en un municipio están relacionadas con las de
otro, lo que puede ser útil para estudiar patrones regionales o
influencias climáticas compartidas entre localidades.

    \begin{tcolorbox}[breakable, size=fbox, boxrule=1pt, pad at break*=1mm,colback=cellbackground, colframe=cellborder]
\prompt{In}{incolor}{25}{\boxspacing}
\begin{Verbatim}[commandchars=\\\{\}]
\PY{k}{def} \PY{n+nf}{func\PYZus{}array}\PY{p}{(}\PY{n}{lista\PYZus{}df}\PY{p}{,}\PY{n}{n\PYZus{}bins}\PY{p}{)}\PY{p}{:}
    \PY{n}{array\PYZus{}freq} \PY{o}{=} \PY{p}{[}\PY{p}{]}
    \PY{k}{if} \PY{n+nb}{len}\PY{p}{(}\PY{n}{lista\PYZus{}df}\PY{p}{)} \PY{o}{!=} \PY{l+m+mi}{0}\PY{p}{:}
        \PY{k}{for} \PY{n}{df} \PY{o+ow}{in} \PY{n}{lista\PYZus{}df}\PY{p}{:}
            \PY{n}{fr}\PY{p}{,} \PY{n}{nb} \PY{o}{=} \PY{n}{np}\PY{o}{.}\PY{n}{histogram}\PY{p}{(}\PY{n}{df}\PY{p}{[}\PY{n}{densidad}\PY{p}{]}\PY{p}{,} \PY{n}{bins}\PY{o}{=}\PY{n}{n\PYZus{}bins}\PY{p}{)}
            \PY{n}{fr} \PY{o}{=} \PY{n}{fr} \PY{o}{/} \PY{n+nb}{sum}\PY{p}{(}\PY{n}{fr}\PY{p}{)}
            \PY{n}{array\PYZus{}freq}\PY{o}{.}\PY{n}{append}\PY{p}{(}\PY{n}{fr}\PY{p}{)}
    \PY{k}{else}\PY{p}{:}
        \PY{n+nb}{print}\PY{p}{(}\PY{l+s+s2}{\PYZdq{}}\PY{l+s+s2}{Dataframe vacío}\PY{l+s+s2}{\PYZdq{}}\PY{p}{,}\PY{n+nb}{len}\PY{p}{(}\PY{n}{lista\PYZus{}df}\PY{p}{[}\PY{n}{densidad}\PY{p}{]}\PY{p}{)}\PY{p}{)}
    \PY{k}{return} \PY{n}{array\PYZus{}freq}
\end{Verbatim}
\end{tcolorbox}

    \begin{tcolorbox}[breakable, size=fbox, boxrule=1pt, pad at break*=1mm,colback=cellbackground, colframe=cellborder]
\prompt{In}{incolor}{24}{\boxspacing}
\begin{Verbatim}[commandchars=\\\{\}]
\PY{k}{for} \PY{n}{u}\PY{p}{,}\PY{n}{list\PYZus{}dataf} \PY{o+ow}{in} \PY{n+nb}{enumerate}\PY{p}{(}\PY{n}{mx\PYZus{}df}\PY{p}{)}\PY{p}{:}
    \PY{n}{array\PYZus{}freq} \PY{o}{=} \PY{n}{func\PYZus{}array}\PY{p}{(}\PY{n}{list\PYZus{}dataf}\PY{p}{,}\PY{n}{list\PYZus{}bn}\PY{p}{[}\PY{n}{u}\PY{p}{]}\PY{p}{)}
    \PY{n}{corr\PYZus{}pears}\PY{p}{,} \PY{n}{valores\PYZus{}p} \PY{o}{=} \PY{n}{corr\PYZus{}pearsonr}\PY{p}{(}\PY{n}{np}\PY{o}{.}\PY{n}{array}\PY{p}{(}\PY{n}{array\PYZus{}freq}\PY{p}{)}\PY{p}{)}
    \PY{n}{titulos} \PY{o}{=} \PY{p}{[}
        \PY{l+s+sa}{f}\PY{l+s+s2}{\PYZdq{}}\PY{l+s+s2}{Correlaciones de Pearson entre }\PY{l+s+s2}{\PYZdq{}}
        \PY{l+s+sa}{rf}\PY{l+s+s2}{\PYZdq{}}\PY{l+s+s2}{\PYZdl{}}\PY{l+s+s2}{\PYZbs{}}\PY{l+s+s2}{rho(\PYZdl{}}\PY{l+s+si}{\PYZob{}}\PY{n}{abreviatura\PYZus{}variable}\PY{l+s+si}{\PYZcb{}}\PY{l+s+s2}{\PYZdl{})\PYZdl{}}\PY{l+s+s2}{\PYZdq{}}
        \PY{l+s+sa}{f}\PY{l+s+s2}{\PYZdq{}}\PY{l+s+se}{\PYZbs{}n}\PY{l+s+s2}{ de }\PY{l+s+si}{\PYZob{}}\PY{n}{abreviatura\PYZus{}variable}\PY{l+s+si}{\PYZcb{}}\PY{l+s+s2}{ por municipio}\PY{l+s+s2}{\PYZdq{}}\PY{p}{,}
        \PY{l+s+s2}{\PYZdq{}}\PY{l+s+s2}{Valores p asociados para cada alternativa}\PY{l+s+s2}{\PYZdq{}}\PY{p}{,}
    \PY{p}{]}

    \PY{n}{dscrip} \PY{o}{=} \PY{p}{[}
        \PY{l+s+sa}{rf}\PY{l+s+s2}{\PYZdq{}}\PY{l+s+s2}{Correlaciones de Pearson de \PYZdl{}}\PY{l+s+s2}{\PYZbs{}}\PY{l+s+s2}{rho(\PYZdl{}}\PY{l+s+si}{\PYZob{}}\PY{n}{abreviatura\PYZus{}variable}\PY{l+s+si}{\PYZcb{}}\PY{l+s+s2}{\PYZdl{})\PYZdl{}}\PY{l+s+s2}{\PYZdq{}}\PY{p}{,}
        \PY{l+s+s2}{\PYZdq{}}\PY{l+s+s2}{Valores p}\PY{l+s+s2}{\PYZdq{}}\PY{p}{,}
    \PY{p}{]}

    \PY{n}{farmers} \PY{o}{=} \PY{p}{[}\PY{n}{lista\PYZus{}municipal}\PY{p}{,} \PY{n}{lista\PYZus{}municipal}\PY{p}{]}
    \PY{n}{harvest} \PY{o}{=} \PY{p}{[}\PY{n}{corr\PYZus{}pears}\PY{p}{,} \PY{n}{valores\PYZus{}p}\PY{p}{]}
    \PY{n}{plt\PYZus{}corr\PYZus{}pears}\PY{p}{(}
        \PY{n}{harvest}\PY{p}{,}
        \PY{n}{titulos}\PY{p}{,}
        \PY{n}{farmers}\PY{p}{,}
        \PY{n}{figsize}\PY{o}{=}\PY{p}{(}\PY{l+m+mi}{45}\PY{p}{,} \PY{l+m+mi}{35}\PY{p}{)}\PY{p}{,}
        \PY{n}{labelsize}\PY{o}{=}\PY{l+m+mi}{35}\PY{p}{,}
        \PY{n}{fontsize}\PY{o}{=}\PY{l+m+mi}{55}\PY{p}{,}
        \PY{n}{tight\PYZus{}layout}\PY{o}{=}\PY{k+kc}{True}\PY{p}{,}
        \PY{n}{padd}\PY{o}{=}\PY{l+m+mi}{35}\PY{p}{,}
        \PY{n}{formato}\PY{o}{=}\PY{l+s+sa}{r}\PY{l+s+s2}{\PYZdq{}}\PY{l+s+s2}{\PYZdl{}}\PY{l+s+si}{\PYZob{}:.2f\PYZcb{}}\PY{l+s+s2}{\PYZdl{}}\PY{l+s+s2}{\PYZdq{}}\PY{p}{,}
        \PY{n}{descript}\PY{o}{=}\PY{n}{dscrip}\PY{p}{,}
        \PY{n}{nNums}\PY{o}{=}\PY{l+m+mi}{10}\PY{p}{,}
        \PY{n}{size}\PY{o}{=}\PY{l+s+s2}{\PYZdq{}}\PY{l+s+s2}{5}\PY{l+s+s2}{\PYZpc{}}\PY{l+s+s2}{\PYZdq{}}\PY{p}{,}
        \PY{n}{padr}\PY{o}{=}\PY{l+s+s2}{\PYZdq{}}\PY{l+s+s2}{2}\PY{l+s+s2}{\PYZpc{}}\PY{l+s+s2}{\PYZdq{}}\PY{p}{,}
        \PY{n}{descrip\PYZus{}bar}\PY{o}{=}\PY{l+m+mi}{50}\PY{p}{,}
        \PY{n}{var\PYZus{}clima}\PY{o}{=}\PY{n}{var\PYZus{}clima}\PY{p}{,}
        \PY{n}{wspace}\PY{o}{=}\PY{l+m+mf}{0.5}\PY{p}{,}
    \PY{p}{)}
\end{Verbatim}
\end{tcolorbox}

    \hypertarget{patrones-regionales-o-influencias-climuxe1ticas-compartidas-entre-localidades}{%
\subsubsection{Patrones regionales o influencias climáticas compartidas
entre
localidades}\label{patrones-regionales-o-influencias-climuxe1ticas-compartidas-entre-localidades}}

    Los mapas de calor generados con \texttt{imshow()} muestran los
diferentes niveles de correlación entre una variable climática
específica en distintos municipios. En estos gráficos, cada celda
representa la correlación entre las mediciones de la misma variable
climática, como la temperatura, en dos municipios diferentes. Los
colores en el mapa indican el grado de correlación: los tonos más claros
reflejan una alta correlación, mientras que los tonos más oscuros
indican una baja correlación.

    \begin{tcolorbox}[breakable, size=fbox, boxrule=1pt, pad at break*=1mm,colback=cellbackground, colframe=cellborder]
\prompt{In}{incolor}{19}{\boxspacing}
\begin{Verbatim}[commandchars=\\\{\}]
\PY{n}{imshow\PYZus{}plots}\PY{p}{(}\PY{l+s+s2}{\PYZdq{}}\PY{l+s+s2}{plots/temperatura\PYZus{}2m\PYZus{}corr.png}\PY{l+s+s2}{\PYZdq{}}\PY{p}{)}\PY{p}{;}
\end{Verbatim}
\end{tcolorbox}

    \begin{center}
    \adjustimage{max size={0.9\linewidth}{0.9\paperheight}}{datosaire_files/datosaire_74_0.png}
    \end{center}
    { \hspace*{\fill} \\}
    
    Estos mapas permiten visualizar cómo, a lo largo del tiempo, las
variables climáticas en diferentes municipios pueden estar relacionadas
entre sí. Por ejemplo, si los mapas muestran una alta correlación entre
la temperatura en dos municipios, esto puede sugerir que los patrones de
temperatura en esos lugares son similares o que están influenciados por
factores climáticos compartidos, como corrientes atmosféricas o
condiciones estacionales similares. La correlación entre variables
climáticas puede variar dependiendo de la proximidad geográfica, las
condiciones meteorológicas regionales y las características locales.

    \begin{tcolorbox}[breakable, size=fbox, boxrule=1pt, pad at break*=1mm,colback=cellbackground, colframe=cellborder]
\prompt{In}{incolor}{20}{\boxspacing}
\begin{Verbatim}[commandchars=\\\{\}]
\PY{n}{imshow\PYZus{}plots}\PY{p}{(}\PY{l+s+s2}{\PYZdq{}}\PY{l+s+s2}{plots/pm10\PYZus{}concentracion\PYZus{}corr.png}\PY{l+s+s2}{\PYZdq{}}\PY{p}{)}
\end{Verbatim}
\end{tcolorbox}

    \begin{center}
    \adjustimage{max size={0.9\linewidth}{0.9\paperheight}}{datosaire_files/datosaire_76_0.png}
    \end{center}
    { \hspace*{\fill} \\}
    
    Un \textbf{mapa de calor} es una representación gráfica de datos en una
matriz, donde los valores numéricos se visualizan mediante una escala de
colores. Este tipo de gráfico es útil para identificar patrones y
relaciones en grandes conjuntos de datos, ya que permite observar de
manera rápida y clara las áreas con alta o baja correlación entre
variables. La visualización en colores facilita la comprensión de cómo
las variables climáticas en diferentes municipios están
interrelacionadas a lo largo del tiempo.

    \begin{tcolorbox}[breakable, size=fbox, boxrule=1pt, pad at break*=1mm,colback=cellbackground, colframe=cellborder]
\prompt{In}{incolor}{21}{\boxspacing}
\begin{Verbatim}[commandchars=\\\{\}]
\PY{n}{imshow\PYZus{}plots}\PY{p}{(}\PY{l+s+s2}{\PYZdq{}}\PY{l+s+s2}{plots/pm25\PYZus{}concentracion\PYZus{}corr.png}\PY{l+s+s2}{\PYZdq{}}\PY{p}{)}
\end{Verbatim}
\end{tcolorbox}

    \begin{center}
    \adjustimage{max size={0.9\linewidth}{0.9\paperheight}}{datosaire_files/datosaire_78_0.png}
    \end{center}
    { \hspace*{\fill} \\}
    
    Además de los mapas de calor que muestran los niveles de correlación
entre las variables climáticas en distintos municipios, también es común
generar una \textbf{matriz de p-valores} para complementar el análisis.
Los p-valores indican la significancia estadística de las correlaciones
calculadas. Un p-valor bajo (generalmente menor a 0.05) sugiere que la
correlación observada es significativa y no se debe al azar, mientras
que un p-valor alto indica que la correlación no es estadísticamente
significativa.

    \begin{tcolorbox}[breakable, size=fbox, boxrule=1pt, pad at break*=1mm,colback=cellbackground, colframe=cellborder]
\prompt{In}{incolor}{24}{\boxspacing}
\begin{Verbatim}[commandchars=\\\{\}]
\PY{n}{imshow\PYZus{}plots}\PY{p}{(}\PY{l+s+s2}{\PYZdq{}}\PY{l+s+s2}{plots/humedad\PYZus{}relativa\PYZus{}2m\PYZus{}corr.png}\PY{l+s+s2}{\PYZdq{}}\PY{p}{)}
\end{Verbatim}
\end{tcolorbox}

    \begin{center}
    \adjustimage{max size={0.9\linewidth}{0.9\paperheight}}{datosaire_files/datosaire_80_0.png}
    \end{center}
    { \hspace*{\fill} \\}
    
    La \textbf{matriz de p-valores} se presenta de manera similar a un mapa
de calor, donde cada celda corresponde a un p-valor asociado con la
correlación entre una variable climática en dos municipios. Esta matriz
es fundamental para asegurar que las correlaciones que se observan en
los mapas de calor son estadísticamente relevantes, lo que brinda mayor
confianza en los patrones detectados. En resumen, mientras el mapa de
calor muestra el nivel de correlación, la matriz de p-valores ayuda a
determinar la fiabilidad de esas correlaciones.

    \begin{tcolorbox}[breakable, size=fbox, boxrule=1pt, pad at break*=1mm,colback=cellbackground, colframe=cellborder]
\prompt{In}{incolor}{26}{\boxspacing}
\begin{Verbatim}[commandchars=\\\{\}]
\PY{n}{imshow\PYZus{}plots}\PY{p}{(}\PY{l+s+s2}{\PYZdq{}}\PY{l+s+s2}{plots/o3\PYZus{}concentracion\PYZus{}corr.png}\PY{l+s+s2}{\PYZdq{}}\PY{p}{)}
\end{Verbatim}
\end{tcolorbox}

    \begin{center}
    \adjustimage{max size={0.9\linewidth}{0.9\paperheight}}{datosaire_files/datosaire_82_0.png}
    \end{center}
    { \hspace*{\fill} \\}
    
    \begin{tcolorbox}[breakable, size=fbox, boxrule=1pt, pad at break*=1mm,colback=cellbackground, colframe=cellborder]
\prompt{In}{incolor}{28}{\boxspacing}
\begin{Verbatim}[commandchars=\\\{\}]
\PY{n}{imshow\PYZus{}plots}\PY{p}{(}\PY{l+s+s2}{\PYZdq{}}\PY{l+s+s2}{plots/no\PYZus{}concentracion\PYZus{}corr.png}\PY{l+s+s2}{\PYZdq{}}\PY{p}{)}
\end{Verbatim}
\end{tcolorbox}

    \begin{center}
    \adjustimage{max size={0.9\linewidth}{0.9\paperheight}}{datosaire_files/datosaire_83_0.png}
    \end{center}
    { \hspace*{\fill} \\}
    
    \begin{tcolorbox}[breakable, size=fbox, boxrule=1pt, pad at break*=1mm,colback=cellbackground, colframe=cellborder]
\prompt{In}{incolor}{29}{\boxspacing}
\begin{Verbatim}[commandchars=\\\{\}]
\PY{n}{imshow\PYZus{}plots}\PY{p}{(}\PY{l+s+s2}{\PYZdq{}}\PY{l+s+s2}{plots/no2\PYZus{}concentracion\PYZus{}corr.png}\PY{l+s+s2}{\PYZdq{}}\PY{p}{)}
\end{Verbatim}
\end{tcolorbox}

    \begin{center}
    \adjustimage{max size={0.9\linewidth}{0.9\paperheight}}{datosaire_files/datosaire_84_0.png}
    \end{center}
    { \hspace*{\fill} \\}
    
    \begin{tcolorbox}[breakable, size=fbox, boxrule=1pt, pad at break*=1mm,colback=cellbackground, colframe=cellborder]
\prompt{In}{incolor}{30}{\boxspacing}
\begin{Verbatim}[commandchars=\\\{\}]
\PY{n}{imshow\PYZus{}plots}\PY{p}{(}\PY{l+s+s2}{\PYZdq{}}\PY{l+s+s2}{plots/direccion\PYZus{}viento\PYZus{}corr.png}\PY{l+s+s2}{\PYZdq{}}\PY{p}{)}
\end{Verbatim}
\end{tcolorbox}

    \begin{center}
    \adjustimage{max size={0.9\linewidth}{0.9\paperheight}}{datosaire_files/datosaire_85_0.png}
    \end{center}
    { \hspace*{\fill} \\}
    
    \begin{tcolorbox}[breakable, size=fbox, boxrule=1pt, pad at break*=1mm,colback=cellbackground, colframe=cellborder]
\prompt{In}{incolor}{31}{\boxspacing}
\begin{Verbatim}[commandchars=\\\{\}]
\PY{n}{imshow\PYZus{}plots}\PY{p}{(}\PY{l+s+s2}{\PYZdq{}}\PY{l+s+s2}{plots/presion\PYZus{}atmosferica\PYZus{}corr.png}\PY{l+s+s2}{\PYZdq{}}\PY{p}{)}
\end{Verbatim}
\end{tcolorbox}

    \begin{center}
    \adjustimage{max size={0.9\linewidth}{0.9\paperheight}}{datosaire_files/datosaire_86_0.png}
    \end{center}
    { \hspace*{\fill} \\}
    
    Como trabajo adicional, se grafican las variables climáticas cuyos
coeficientes de correlación de Pearson son mayores a 0.7. Este umbral
indica una correlación fuerte entre las variables en diferentes
municipios, lo que sugiere que los patrones de variación en una
localidad están estrechamente relacionados con los de otra. Estas
variables se visualizan en un gráfico de dispersión utilizando la
función \texttt{plot()}, donde el eje X representa la \textbf{PDF
(función de densidad de probabilidad)} de la variable climática en un
municipio y el eje Y muestra la \textbf{PDF} correspondiente de otro
municipio.

El propósito de este análisis es observar si los datos de ambos
municipios se ajustan a un \textbf{modelo lineal común}. Si los puntos
en el gráfico se distribuyen cerca de una línea recta, esto indicaría
una relación lineal entre las PDF de las dos localidades, lo que sugiere
que los datos siguen un patrón común a pesar de las diferencias
geográficas o ambientales. Este enfoque permite identificar tendencias
compartidas y construir modelos predictivos que relacionen las
condiciones climáticas de distintas regiones mediante un ajuste lineal.

    \begin{tcolorbox}[breakable, size=fbox, boxrule=1pt, pad at break*=1mm,colback=cellbackground, colframe=cellborder]
\prompt{In}{incolor}{ }{\boxspacing}
\begin{Verbatim}[commandchars=\\\{\}]
\PY{n}{list\PYZus{}arr} \PY{o}{=} \PY{p}{[}\PY{p}{]}\PY{p}{;} \PY{n}{name\PYZus{}var} \PY{o}{=} \PY{p}{[}\PY{p}{]}\PY{p}{;} 
\PY{n}{titX} \PY{o}{=} \PY{p}{[}\PY{p}{]}\PY{p}{;} \PY{n}{titY} \PY{o}{=} \PY{p}{[}\PY{p}{]}\PY{p}{;} \PY{n}{uni\PYZus{}var}\PY{o}{=}\PY{p}{[}\PY{p}{]}

\PY{k}{for} \PY{n}{u} \PY{o+ow}{in} \PY{p}{[}\PY{l+m+mi}{10}\PY{p}{,}\PY{l+m+mi}{12}\PY{p}{]}\PY{p}{:}
    \PY{n+nb}{print}\PY{p}{(}\PY{n}{pathfil}\PY{p}{[}\PY{n}{u}\PY{p}{]}\PY{p}{)}
    \PY{n}{unidad\PYZus{}variable} \PY{o}{=} \PY{n}{variables\PYZus{}contaminacion}\PY{p}{[}\PY{n}{pathfil}\PY{p}{[}\PY{n}{u}\PY{p}{]}\PY{p}{]}\PY{p}{[}\PY{l+s+s2}{\PYZdq{}}\PY{l+s+s2}{unidad}\PY{l+s+s2}{\PYZdq{}}\PY{p}{]}
    \PY{n}{nombre\PYZus{}variable} \PY{o}{=} \PY{n}{variables\PYZus{}contaminacion}\PY{p}{[}\PY{n}{pathfil}\PY{p}{[}\PY{n}{u}\PY{p}{]}\PY{p}{]}\PY{p}{[}\PY{l+s+s2}{\PYZdq{}}\PY{l+s+s2}{variable}\PY{l+s+s2}{\PYZdq{}}\PY{p}{]}
    \PY{n}{abreviatura\PYZus{}variable} \PY{o}{=} \PY{n}{variables\PYZus{}contaminacion}\PY{p}{[}\PY{n}{pathfil}\PY{p}{[}\PY{n}{u}\PY{p}{]}\PY{p}{]}\PY{p}{[}\PY{l+s+s2}{\PYZdq{}}\PY{l+s+s2}{abreviatura}\PY{l+s+s2}{\PYZdq{}}\PY{p}{]}
    \PY{n}{uni\PYZus{}var}\PY{o}{.}\PY{n}{append}\PY{p}{(}\PY{n}{pathfil}\PY{p}{[}\PY{n}{u}\PY{p}{]}\PY{p}{)}
    \PY{n}{name\PYZus{}var}\PY{o}{.}\PY{n}{append}\PY{p}{(}\PY{l+s+sa}{f}\PY{l+s+s2}{\PYZdq{}}\PY{l+s+s2}{Relación de }\PY{l+s+si}{\PYZob{}}\PY{n}{abreviatura\PYZus{}variable}\PY{l+s+si}{\PYZcb{}}\PY{l+s+s2}{ entre los municipios \PYZdl{}i, j\PYZdl{}}\PY{l+s+s2}{\PYZdq{}}\PY{p}{)}
    \PY{n}{titX}\PY{o}{.}\PY{n}{append}\PY{p}{(}\PY{l+s+sa}{rf}\PY{l+s+s2}{\PYZdq{}}\PY{l+s+s2}{\PYZdl{}}\PY{l+s+s2}{\PYZbs{}}\PY{l+s+s2}{rho[\PYZdl{}}\PY{l+s+si}{\PYZob{}}\PY{n}{abreviatura\PYZus{}variable}\PY{l+s+si}{\PYZcb{}}\PY{l+s+s2}{(}\PY{l+s+si}{\PYZob{}}\PY{n}{unidad\PYZus{}variable}\PY{l+s+si}{\PYZcb{}}\PY{l+s+s2}{)\PYZdl{}]\PYZdl{} del Municipio \PYZdl{}j\PYZdl{}}\PY{l+s+s2}{\PYZdq{}}\PY{p}{)}
    \PY{n}{titY}\PY{o}{.}\PY{n}{append}\PY{p}{(}\PY{l+s+sa}{rf}\PY{l+s+s2}{\PYZdq{}}\PY{l+s+s2}{\PYZdl{}}\PY{l+s+s2}{\PYZbs{}}\PY{l+s+s2}{rho[\PYZdl{}}\PY{l+s+si}{\PYZob{}}\PY{n}{abreviatura\PYZus{}variable}\PY{l+s+si}{\PYZcb{}}\PY{l+s+s2}{(}\PY{l+s+si}{\PYZob{}}\PY{n}{unidad\PYZus{}variable}\PY{l+s+si}{\PYZcb{}}\PY{l+s+s2}{)\PYZdl{}]\PYZdl{} del Municipio \PYZdl{}i\PYZdl{}}\PY{l+s+s2}{\PYZdq{}}\PY{p}{)}
    \PY{n}{list\PYZus{}arr}\PY{o}{.}\PY{n}{append}\PY{p}{(}\PY{n}{func\PYZus{}array}\PY{p}{(}\PY{n}{mx\PYZus{}df}\PY{p}{[}\PY{n}{u}\PY{p}{]}\PY{p}{,} \PY{n}{list\PYZus{}bn}\PY{p}{[}\PY{n}{u}\PY{p}{]}\PY{p}{)}\PY{p}{)}
\PY{n}{plots\PYZus{}lineal\PYZus{}model}\PY{p}{(}\PY{l+m+mf}{0.12} \PY{p}{,}\PY{n}{list\PYZus{}arr}\PY{p}{,} \PY{n}{name\PYZus{}var}\PY{p}{,} \PY{n}{titX}\PY{p}{,} \PY{n}{titY}\PY{p}{,}\PY{p}{[}\PY{l+m+mf}{0.12}\PY{p}{,}\PY{l+m+mf}{0.1}\PY{p}{]}\PY{p}{,}\PY{p}{[}\PY{l+m+mf}{0.12}\PY{p}{,}\PY{l+m+mf}{0.1}\PY{p}{]}\PY{p}{,}\PY{n}{u}\PY{p}{)}
\end{Verbatim}
\end{tcolorbox}

    Los resultados son prometedores, ya que los gráficos muestran que un
modelo lineal puede ajustarse a la relación entre la misma variable
climática en diferentes municipios. En estos gráficos, donde el eje X
representa la \textbf{PDF} de la variable climática en un municipio y el
eje Y la \textbf{PDF} de la misma variable en otro municipio, se observa
una tendencia lineal en muchos casos. Esto sugiere que, bajo ciertas
condiciones, existe una relación lineal significativa entre las
variables climáticas de los municipios comparados, lo que indica que los
cambios en un municipio están estrechamente correlacionados con los
cambios en otro.

El análisis revela que las variables climáticas entre ellos sí pueden
seguir una tendencia lineal. Este ajuste es posible considerando los
niveles de error y dispersión presentes en los datos. Este hallazgo
sugiere que, a pesar de las diferencias geográficas y climáticas, es
posible predecir el comportamiento de una variable climática en un
municipio a partir de los datos de otro, utilizando un modelo lineal
adecuado.

    \begin{tcolorbox}[breakable, size=fbox, boxrule=1pt, pad at break*=1mm,colback=cellbackground, colframe=cellborder]
\prompt{In}{incolor}{12}{\boxspacing}
\begin{Verbatim}[commandchars=\\\{\}]
\PY{n}{imshow\PYZus{}plots}\PY{p}{(}\PY{l+s+s2}{\PYZdq{}}\PY{l+s+s2}{plots/gxGraf13.png}\PY{l+s+s2}{\PYZdq{}}\PY{p}{)}
\end{Verbatim}
\end{tcolorbox}

    \begin{center}
    \adjustimage{max size={0.9\linewidth}{0.9\paperheight}}{datosaire_files/datosaire_90_0.png}
    \end{center}
    { \hspace*{\fill} \\}
    
    \begin{tcolorbox}[breakable, size=fbox, boxrule=1pt, pad at break*=1mm,colback=cellbackground, colframe=cellborder]
\prompt{In}{incolor}{13}{\boxspacing}
\begin{Verbatim}[commandchars=\\\{\}]
\PY{n}{imshow\PYZus{}plots}\PY{p}{(}\PY{l+s+s2}{\PYZdq{}}\PY{l+s+s2}{plots/gxGraf17.png}\PY{l+s+s2}{\PYZdq{}}\PY{p}{)}
\end{Verbatim}
\end{tcolorbox}

    \begin{center}
    \adjustimage{max size={0.9\linewidth}{0.9\paperheight}}{datosaire_files/datosaire_91_0.png}
    \end{center}
    { \hspace*{\fill} \\}
    
    \begin{tcolorbox}[breakable, size=fbox, boxrule=1pt, pad at break*=1mm,colback=cellbackground, colframe=cellborder]
\prompt{In}{incolor}{14}{\boxspacing}
\begin{Verbatim}[commandchars=\\\{\}]
\PY{n}{imshow\PYZus{}plots}\PY{p}{(}\PY{l+s+s2}{\PYZdq{}}\PY{l+s+s2}{plots/gxGraf9.png}\PY{l+s+s2}{\PYZdq{}}\PY{p}{)}
\end{Verbatim}
\end{tcolorbox}

    \begin{center}
    \adjustimage{max size={0.9\linewidth}{0.9\paperheight}}{datosaire_files/datosaire_92_0.png}
    \end{center}
    { \hspace*{\fill} \\}
    
    \begin{tcolorbox}[breakable, size=fbox, boxrule=1pt, pad at break*=1mm,colback=cellbackground, colframe=cellborder]
\prompt{In}{incolor}{15}{\boxspacing}
\begin{Verbatim}[commandchars=\\\{\}]
\PY{n}{imshow\PYZus{}plots}\PY{p}{(}\PY{l+s+s2}{\PYZdq{}}\PY{l+s+s2}{plots/gxGraf7.png}\PY{l+s+s2}{\PYZdq{}}\PY{p}{)}
\end{Verbatim}
\end{tcolorbox}

    \begin{center}
    \adjustimage{max size={0.9\linewidth}{0.9\paperheight}}{datosaire_files/datosaire_93_0.png}
    \end{center}
    { \hspace*{\fill} \\}
    
    \begin{tcolorbox}[breakable, size=fbox, boxrule=1pt, pad at break*=1mm,colback=cellbackground, colframe=cellborder]
\prompt{In}{incolor}{16}{\boxspacing}
\begin{Verbatim}[commandchars=\\\{\}]
\PY{n}{imshow\PYZus{}plots}\PY{p}{(}\PY{l+s+s2}{\PYZdq{}}\PY{l+s+s2}{plots/gxGraf3.png}\PY{l+s+s2}{\PYZdq{}}\PY{p}{)}
\end{Verbatim}
\end{tcolorbox}

    \begin{center}
    \adjustimage{max size={0.9\linewidth}{0.9\paperheight}}{datosaire_files/datosaire_94_0.png}
    \end{center}
    { \hspace*{\fill} \\}
    
    Para calcular las correlaciones entre las variables climáticas de los
diferentes municipios, se utilizó el \textbf{coeficiente de correlación
de Pearson}. Esta medida evalúa la relación lineal entre dos variables y
se define matemáticamente como:

\[
r = \frac{\sum{(x_i - \bar{x})(y_i - \bar{y})}}{\sqrt{\sum{(x_i - \bar{x})^2} \sum{(y_i - \bar{y})^2}}} 
\]

donde \(x_i\) y \(y_i\) son los valores individuales de las variables en
los diferentes municipios, y \(\bar{x}\) y \(\bar{y}\) son las medias de
esas variables. Un valor de \(r\) cercano a 1 indica una fuerte
correlación positiva, mientras que un valor cercano a -1 indica una
fuerte correlación negativa.

Para los ajustes lineales, se utilizó el método de \textbf{mínimos
cuadrados}, que busca minimizar la suma de los cuadrados de las
diferencias entre los valores observados y los valores predichos por el
modelo. El ajuste lineal básico sigue la forma:

\[ 
y = mx + b 
\]

donde \(m\) es la pendiente y \(b\) es la intersección con el eje Y.
Para encontrar los mejores valores de \(m\) y \(b\), se minimiza la suma
de los cuadrados de los residuos, \(S\):

\[
S = \sum{(y_i - (mx_i + b))^2} 
\]

Además, para asegurar la precisión del ajuste, se minimizó la
\textbf{función chi-cuadrado} (\(\chi^2\)) considerando incertidumbres
constantes en todos los casos. La función \(\chi^2\) se define como:

\[ 
\chi^2 = \sum{\frac{(y_i - f(x_i))^2}{\sigma^2}} 
\]

donde \(y_i\) son los valores observados, \(f(x_i)\) los valores
ajustados, y \(\sigma\) es la incertidumbre constante asociada con los
datos. Este enfoque permite realizar ajustes robustos para identificar
relaciones lineales entre las variables climáticas de distintos
municipios, tomando en cuenta la variabilidad de los datos.

    La tabla presentada a continuación muestra los municipios en los cuales
una variable climática específica tiene correlaciones mayores a 0.7
entre los datos de dicha variable para diferentes municipios. Es decir,
se está analizando cómo varía una misma variable climática, pero en
diferentes municipios, encontrando fuertes relaciones (correlaciones)
entre estos municipios. La tabla está organizada de la siguiente manera:

\begin{enumerate}
\def\labelenumi{\arabic{enumi}.}
\item
  \textbf{Variable climática:} Esta columna contiene el nombre de la
  variable climática que se está analizando, como temperatura,
  precipitación o concentración de CO.
\item
  \textbf{Municipios:} Esta columna lista los municipios donde la
  correlación de la misma variable climática entre diferentes municipios
  es mayor a 0.7. En cada fila, la variable es la misma, pero los
  municipios varían.
\item
  \textbf{Correlación promedio:} Esta columna muestra el valor promedio
  de la correlación de la variable climática entre los municipios
  seleccionados. Un valor alto indica que los datos de la variable
  climática entre esos municipios presentan una relación fuerte.
\item
  \textbf{Pendiente:} Esta columna presenta la pendiente del ajuste de
  los datos en las PDF (Funciones de Densidad de Probabilidad) de la
  variable climática en los municipios seleccionados. La pendiente
  refleja la relación entre los valores de la variable conforme varían
  entre los municipios.
\item
  \textbf{Intercepto:} Esta columna muestra el intercepto del ajuste de
  los datos en las PDF. El intercepto representa el valor base de la
  variable climática cuando la relación entre los municipios se ajusta
  al modelo lineal.
\end{enumerate}

    \begin{tcolorbox}[breakable, size=fbox, boxrule=1pt, pad at break*=1mm,colback=cellbackground, colframe=cellborder]
\prompt{In}{incolor}{315}{\boxspacing}
\begin{Verbatim}[commandchars=\\\{\}]
\PY{n}{imshow\PYZus{}plots}\PY{p}{(}\PY{l+s+s2}{\PYZdq{}}\PY{l+s+s2}{dataf.png}\PY{l+s+s2}{\PYZdq{}}\PY{p}{)}
\end{Verbatim}
\end{tcolorbox}

    \begin{center}
    \adjustimage{max size={0.9\linewidth}{0.9\paperheight}}{datosaire_files/datosaire_97_0.png}
    \end{center}
    { \hspace*{\fill} \\}
    
    \hypertarget{conclusiones}{%
\section{Conclusiones}\label{conclusiones}}

    En conclusión, el análisis realizado hasta ahora muestra resultados
prometedores al identificar correlaciones lineales significativas entre
variables climáticas en diferentes municipios. Se ha logrado ajustar un
modelo lineal bajo ciertas condiciones en casi todos los casos, con
mínimas excepciones, lo que sugiere que las variables climáticas entre
municipios tienden a seguir patrones lineales bajo ciertas
circunstancias. Sin embargo, aún queda mucho por explorar. Debido a las
limitaciones de tiempo, no se ha abordado el estudio de dinámicas no
lineales, las cuales podrían revelar comportamientos más complejos y
realistas de las variables climáticas.

Además, se aplicó el umbral del IQR (rango intercuartílico) a todas las
variables climáticas, pero solo los datos de precipitación fueron
completamente excluidos debido a este filtro. Esto sugiere que los datos
de precipitación presentaron una variabilidad significativa o valores
atípicos que el IQR descartó, lo que hace necesario explorar métodos
alternativos para analizar estos datos y entender mejor su
comportamiento. Las precipitaciones extremas pueden ser clave para
comprender fenómenos climáticos abruptos, por lo que es fundamental
revisarlas con mayor detalle.

En resumen, aunque se ha avanzado significativamente en el análisis
inicial, es crucial continuar investigando, particularmente en dinámicas
no lineales y en la revisión de datos extremos, para obtener una visión
más completa y precisa del comportamiento climático en diferentes
municipios.

    \begin{tcolorbox}[breakable, size=fbox, boxrule=1pt, pad at break*=1mm,colback=cellbackground, colframe=cellborder]
\prompt{In}{incolor}{316}{\boxspacing}
\begin{Verbatim}[commandchars=\\\{\}]
\PY{o}{!}jupyter\PY{+w}{ }nbconvert\PY{+w}{ }\PYZhy{}\PYZhy{}to\PY{+w}{ }latex\PY{+w}{ }datosaire.ipynb
\end{Verbatim}
\end{tcolorbox}

    \begin{Verbatim}[commandchars=\\\{\}]
[NbConvertApp] Converting notebook datosaire.ipynb to latex
/home/alejandro/.pyenv/versions/djworkspace/lib/python3.8/site-
packages/nbconvert/utils/pandoc.py:50: RuntimeWarning: You are using an
unsupported version of pandoc (2.5).
Your version must be at least (2.9.2) but less than (4.0.0).
Refer to https://pandoc.org/installing.html.
Continuing with doubts{\ldots}
  check\_pandoc\_version()
\^{}C
Traceback (most recent call last):
  File "/home/alejandro/.pyenv/versions/djworkspace/bin/jupyter-nbconvert", line
8, in <module>
    sys.exit(main())
  File "/home/alejandro/.pyenv/versions/djworkspace/lib/python3.8/site-
packages/jupyter\_core/application.py", line 283, in launch\_instance
    super().launch\_instance(argv=argv, **kwargs)
  File "/home/alejandro/.pyenv/versions/djworkspace/lib/python3.8/site-
packages/traitlets/config/application.py", line 1075, in launch\_instance
    app.start()
  File "/home/alejandro/.pyenv/versions/djworkspace/lib/python3.8/site-
packages/nbconvert/nbconvertapp.py", line 420, in start
    self.convert\_notebooks()
  File "/home/alejandro/.pyenv/versions/djworkspace/lib/python3.8/site-
packages/nbconvert/nbconvertapp.py", line 597, in convert\_notebooks
    self.convert\_single\_notebook(notebook\_filename)
  File "/home/alejandro/.pyenv/versions/djworkspace/lib/python3.8/site-
packages/nbconvert/nbconvertapp.py", line 563, in convert\_single\_notebook
    output, resources = self.export\_single\_notebook(
  File "/home/alejandro/.pyenv/versions/djworkspace/lib/python3.8/site-
packages/nbconvert/nbconvertapp.py", line 487, in export\_single\_notebook
    output, resources = self.exporter.from\_filename(
  File "/home/alejandro/.pyenv/versions/djworkspace/lib/python3.8/site-
packages/nbconvert/exporters/templateexporter.py", line 386, in from\_filename
    return super().from\_filename(filename, resources, **kw)  \#
type:ignore[return-value]
  File "/home/alejandro/.pyenv/versions/djworkspace/lib/python3.8/site-
packages/nbconvert/exporters/exporter.py", line 201, in from\_filename
    return self.from\_file(f, resources=resources, **kw)
  File "/home/alejandro/.pyenv/versions/djworkspace/lib/python3.8/site-
packages/nbconvert/exporters/templateexporter.py", line 392, in from\_file
    return super().from\_file(file\_stream, resources, **kw)  \#
type:ignore[return-value]
  File "/home/alejandro/.pyenv/versions/djworkspace/lib/python3.8/site-
packages/nbconvert/exporters/exporter.py", line 220, in from\_file
    return self.from\_notebook\_node(
  File "/home/alejandro/.pyenv/versions/djworkspace/lib/python3.8/site-
packages/nbconvert/exporters/latex.py", line 92, in from\_notebook\_node
    return super().from\_notebook\_node(nb, resources, **kw)
  File "/home/alejandro/.pyenv/versions/djworkspace/lib/python3.8/site-
packages/nbconvert/exporters/templateexporter.py", line 424, in
from\_notebook\_node
    output = self.template.render(nb=nb\_copy, resources=resources)
  File "/home/alejandro/.pyenv/versions/djworkspace/lib/python3.8/site-
packages/jinja2/environment.py", line 1302, in render
    return self.environment.concat(self.root\_render\_func(ctx))  \# type: ignore
  File "/home/alejandro/.pyenv/versions/djworkspace/share/jupyter/nbconvert/temp
lates/latex/index.tex.j2", line 27, in root
  File "/home/alejandro/.pyenv/versions/djworkspace/share/jupyter/nbconvert/temp
lates/latex/style\_jupyter.tex.j2", line 104, in root
    ((*- block style\_prompt *))
  File "/home/alejandro/.pyenv/versions/djworkspace/share/jupyter/nbconvert/temp
lates/latex/base.tex.j2", line 15, in root

  File "/home/alejandro/.pyenv/versions/djworkspace/share/jupyter/nbconvert/temp
lates/latex/document\_contents.tex.j2", line 33, in root
    ((* block data\_latex -*))
  File "/home/alejandro/.pyenv/versions/djworkspace/share/jupyter/nbconvert/temp
lates/latex/display\_priority.j2", line 15, in root
    ((*- elif type == 'image/svg+xml' -*))
  File "/home/alejandro/.pyenv/versions/djworkspace/share/jupyter/nbconvert/temp
lates/latex/null.j2", line 12, in root
    To layout the different blocks of a notebook.
  File "/home/alejandro/.pyenv/versions/djworkspace/share/jupyter/nbconvert/temp
lates/latex/base.tex.j2", line 173, in block\_body
    \textbackslash{}def\textbackslash{}gt\{>\}
  File "/home/alejandro/.pyenv/versions/djworkspace/lib/python3.8/site-
packages/jinja2/runtime.py", line 303, in call
    return \_\_obj(*args, **kwargs)
  File "/home/alejandro/.pyenv/versions/djworkspace/lib/python3.8/site-
packages/jinja2/runtime.py", line 384, in \_\_call\_\_
    rv = concat(self.\_stack[self.\_depth](self.\_context))
  File "/home/alejandro/.pyenv/versions/djworkspace/share/jupyter/nbconvert/temp
lates/latex/null.j2", line 37, in block\_body
    ((*- block input\_group -*))
  File "/home/alejandro/.pyenv/versions/djworkspace/share/jupyter/nbconvert/temp
lates/latex/null.j2", line 54, in block\_any\_cell
    ((*- elif output.output\_type == 'stream' -*))
  File "/home/alejandro/.pyenv/versions/djworkspace/share/jupyter/nbconvert/temp
lates/latex/null.j2", line 83, in block\_codecell
    ((*- elif cell.cell\_type in ['markdown'] -*))
  File "/home/alejandro/.pyenv/versions/djworkspace/share/jupyter/nbconvert/temp
lates/latex/null.j2", line 100, in block\_input\_group
    ((*- endif -*))
  File "/home/alejandro/.pyenv/versions/djworkspace/share/jupyter/nbconvert/temp
lates/latex/style\_jupyter.tex.j2", line 173, in block\_input

  File "/home/alejandro/.pyenv/versions/djworkspace/lib/python3.8/site-
packages/nbconvert/filters/highlight.py", line 130, in \_\_call\_\_
    latex = \_pygments\_highlight(
  File "/home/alejandro/.pyenv/versions/djworkspace/lib/python3.8/site-
packages/nbconvert/filters/highlight.py", line 173, in \_pygments\_highlight
    from IPython.lib.lexers import IPython3Lexer
  File "/home/alejandro/.pyenv/versions/djworkspace/lib/python3.8/site-
packages/IPython/\_\_init\_\_.py", line 52, in <module>
    from .core.application import Application
  File "/home/alejandro/.pyenv/versions/djworkspace/lib/python3.8/site-
packages/IPython/core/application.py", line 26, in <module>
    from IPython.core import release, crashhandler
  File "/home/alejandro/.pyenv/versions/djworkspace/lib/python3.8/site-
packages/IPython/core/crashhandler.py", line 27, in <module>
    from IPython.core import ultratb
  File "/home/alejandro/.pyenv/versions/djworkspace/lib/python3.8/site-
packages/IPython/core/ultratb.py", line 95, in <module>
    import pydoc
  File "/usr/lib/python3.8/pydoc.py", line 471, in <module>
    class HTMLDoc(Doc):
  File "/usr/lib/python3.8/pydoc.py", line 1024, in HTMLDoc
    def index(self, dir, shadowed=None):
KeyboardInterrupt
    \end{Verbatim}


    % Add a bibliography block to the postdoc
    
    
    
\end{document}
